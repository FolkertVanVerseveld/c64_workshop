\begin{frame}
\frametitle{Audio}

\begin{itemize}
\item Analoog of Digitaal
\end{itemize}

\begin{figure}
\includegraphics[width=0.6\linewidth]{images/waveforms.png}
\end{figure}

\end{frame}

%----------------------------------------------

\begin{frame}
\frametitle{Analoog Audio}

\begin{itemize}
\item Niet-elektronisch
\item Visualisatie met Oscilloscoop
\end{itemize}

\end{frame}

%----------------------------------------------

\begin{frame}
\frametitle{Digitaal Audio}

\begin{itemize}
\item Benadering van Analoog
\item Stapsgewijs
\item Duidelijk zichtbaar op oude PCs
\end{itemize}

\end{frame}

%----------------------------------------------

\begin{frame}
\frametitle{Digitaal Audio - Speaker}

\begin{itemize}
\item Squaker
\item Zeer ruw
\item Aan of uit
\item Frequentie
\end{itemize}

\end{frame}

%----------------------------------------------

\begin{frame}
\frametitle{Digitaal Audio - Speaker Voorbeelden}

\begin{itemize}
\item Loop
\item Single
\item Arpeggio
\end{itemize}

\end{frame}

%----------------------------------------------

\begin{frame}
\frametitle{Digitaal Audio - NES}

\begin{itemize}
\item 5 kanalen
\item 2 pulse, 1 noise, 1 triangle
\item 1 voor geluidseffecten
\end{itemize}

\end{frame}

%----------------------------------------------

\begin{frame}
\frametitle{Digitaal Audio - NES voorbeeld}

\begin{itemize}
\item The Legend of Zelda theme
\end{itemize}

\end{frame}

%----------------------------------------------

\begin{frame}
\frametitle{Digitaal Audio}

\begin{figure}
\includegraphics[width=0.9\linewidth]{images/rc3nes.png}
\end{figure}

\end{frame}

%----------------------------------------------

\begin{frame}
\frametitle{C64: SID chip}

\begin{itemize}
\item 's werelds eerste audio synthesizer on chip
\item Analoog \emph{en} Digitaal
\item Verschillende onderscheidingen
\item 6581 en 8580
\end{itemize}

\end{frame}

%----------------------------------------------

\begin{frame}
\frametitle{Ter Vergelijking}

\begin{tabular}{|l|l|l|l|}
\hline Model & Geluid & Type & Voices \\ \hline
PET & CPU PIO & 1-bit beeper & 0 \\
VIC-20 & VIC I & 8-bit digital & 4 \\
C64 & SID & A/D 8-bit synth. & 3 \\ \hline
Apple ][ & Speaker & 1-bit beeper & 0 \\
IBM PC & Speaker & 1-bit beeper & 0 \\
ZX Spectrum & AY-3-8912 & 8-bit digitaal & 4 \\ \hline
VCS & TIA & 5-bit digitaal & 2 \\
Vectrex & AY-3-8192 & 8-bit digitaal & 4 \\
NES & RP2A03 & 8-bit digitaal & 5 \\ \hline
\end{tabular}

\end{frame}

%----------------------------------------------

\begin{frame}
\frametitle{ADSR}

\begin{figure}
\includegraphics[width=0.7\linewidth]{images/adsr.png}
\end{figure}

\begin{tabular}{l|l}
Attack & Toetsaanslag \\
Decay & Amplitude zwakt af \\
Sustain & Amplitude constant \\
Release & Geluid `dooft' uit \\
\end{tabular}

\end{frame}

%----------------------------------------------

\begin{frame}
\frametitle{Waveforms}

\begin{figure}
\includegraphics[width=0.75\linewidth]{images/waveforms.png}
\end{figure}

\end{frame}

%----------------------------------------------

\begin{frame}
\frametitle{SID - Technieken}

\begin{itemize}
\item Pulse Width Modulation (PWM)
\item Ring modulation (ring mod)
\end{itemize}

\end{frame}

%----------------------------------------------

\begin{frame}
\frametitle{SID - Voorbeelden}

\begin{itemize}
\item Driller (PWM)
\item Commando
\item Zoids (ring mod)
\end{itemize}

\end{frame}

%----------------------------------------------

\begin{frame}
\frametitle{Intermezzo}

\begin{block}{Muziek}
Raad welk muziekje bij welke computer/console hoort!
\end{block}

\begin{itemize}
\item IBM PC
\item NES
\item C64
\end{itemize}

\end{frame}

%----------------------------------------------

\begin{frame}
\frametitle{Intermezzo - Voorbeelden}

\begin{itemize}
\item Grand Prix Circuit (PC, C64)
\item Castlevania (PC, NES, C64)
\end{itemize}

\end{frame}

%----------------------------------------------

\begin{frame}
\frametitle{Intermezzo}

\begin{itemize}
\item IBM PC
\item NES
\item C64
\end{itemize}

\end{frame}

%----------------------------------------------

\begin{frame}
\frametitle{SID hacking - Arkanoid}

\begin{itemize}
\item Click/pop-bug in 6581
\item 4-bit PCM digitaal
\end{itemize}

\begin{figure}
\includegraphics[width=0.9\linewidth]{images/farts.png}
\end{figure}

\end{frame}

%----------------------------------------------

\begin{frame}
\frametitle{SID hacking - Digitized sound}

\begin{itemize}
%\item mahoney, lft, lman, thcm
\item Vortex, Hi Fi Sky
\item Uniek: klinkt totaal niet als C64 muziek
\end{itemize} 

\end{frame}
