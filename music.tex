\begin{frame}
\frametitle{Audio}

\begin{itemize}
\item Analoog of Digitaal
\end{itemize}

\begin{figure}
\includegraphics[width=0.6\linewidth]{images/waveforms.png}
\end{figure}

\end{frame}

%----------------------------------------------

\begin{frame}
\frametitle{Analoog Audio}

\begin{itemize}
\item Niet-elektronisch
\item Visualisatie met Oscilloscoop
\end{itemize}

\end{frame}

%----------------------------------------------

\begin{frame}
\frametitle{Digitaal Audio}

\begin{itemize}
\item Benadering van Analoog
\item Stapsgewijs
\item Duidelijk zichtbaar op oude PCs
\end{itemize}

\end{frame}

%----------------------------------------------

\begin{frame}
\frametitle[fragile]{Digitaal Audio}

\begin{figure}
	\begin{subfigure}[b]{0.4\textwidth}
		\includegraphics[width=\linewidth]{images/waveforms.png}
	\end{subfigure}
	\begin{subfigure}[b]{0.5\textwidth}
		\includegraphics[width=\linewidth]{images/rc3nes.png}
	\end{subfigure}
\end{figure}

\end{frame}

%----------------------------------------------

\begin{frame}
\frametitle{Digitaal Audio - Speaker}

\begin{itemize}
\item Squaker
\item Zeer ruw
\item Aan of uit
\item Frequentie
\end{itemize}

\end{frame}

%----------------------------------------------

\begin{frame}
\frametitle{Digitaal Audio - Speaker Voorbeelden}

\begin{itemize}
\item Loop
\item Single
\item Arpeggio
\end{itemize}

\end{frame}

%----------------------------------------------

\begin{frame}
\frametitle{Digitaal Audio - NES}

\begin{tabular}{|l|l|l|l|}
\hline \# & type & volumes & frequenties \\ \hline
2 & Pulse wave & 16 & 54Hz-28kHz \\
% XXX er stond 447kHz als upper bound, maar kan toch niet kloppen?
1 & White noise & 16 & 29.3Hz-47kHz \\
1 & Triangle wave & 1 & 27Hz-56kHz \\
1 & Delta PCM & 16 & 4kHz-33kHz \\ \hline
\end{tabular}

\end{frame}
%----------------------------------------------

\begin{frame}
\frametitle{Digitaal Audio - NES}

\begin{itemize}
\item 5 kanalen
\item 8 bit voor pulse, noise en triangle
\item 1 bit voor DPCM
\item hardware pitch bending
\end{itemize}

\end{frame}

%----------------------------------------------

\begin{frame}
\frametitle{Digitaal Audio - NES voorbeeld}

\begin{itemize}
\item The Legend of Zelda theme
\end{itemize}

\end{frame}

%----------------------------------------------

\begin{frame}
\frametitle{Digitaal Audio}

\begin{figure}
\includegraphics[width=0.9\linewidth]{images/rc3nes.png}
\end{figure}

\end{frame}

%----------------------------------------------

\begin{frame}
\frametitle{C64: SID chip}

\begin{itemize}
\item 's werelds eerste audio synthesizer
\item Analoog \emph{en} Digitaal
\item Verschillende onderscheidingen
\item 6581 en 8580
\item Timbaland plagiaat
\end{itemize}

\end{frame}

%----------------------------------------------

\begin{frame}
\frametitle{Ter Vergelijking}

\begin{tabular}{|l|l|l|l|}
\hline Model & Geluid & Type & Voices \\ \hline
PET & CPU PIO & 1-bit beeper & 0 \\
VIC-20 & VIC I & 8-bit digital & 4 \\
C64 & SID & A/D 8-bit synthesizer & 3 \\ \hline
Apple ][ & Speaker & 1-bit beeper & 0 \\
IBM PC & Speaker & 1-bit beeper & 0 \\
NES & RP2A03 & 8-bit digitaal & 5 \\ \hline
\end{tabular}

\end{frame}

%----------------------------------------------

\begin{frame}
\frametitle{Audio}

\begin{itemize}
\item 3 kanalen (voices)
\item ADSR envelope
\end{itemize}

\end{frame}

%----------------------------------------------

\begin{frame}
\frametitle{ADSR}

\begin{figure}
\includegraphics[width=0.8\linewidth]{images/adsr.png}
\end{figure}

\end{frame}

%----------------------------------------------

\begin{frame}
\frametitle{Waveforms}

\begin{itemize}
\item Sawtooth
\item Square
\item Sinuso\"ide
\item Triangle
\item Noise
\end{itemize}

\end{frame}

%----------------------------------------------

\begin{frame}
\frametitle{Waveforms}

\begin{figure}
\includegraphics[width=0.75\linewidth]{images/waveforms.png}
\end{figure}

\end{frame}

%----------------------------------------------

\begin{frame}
\frametitle{SID - Technieken}

\begin{itemize}
\item Pulse Width Modulation (PWM)
\item Ring modulation (ring mod)
\end{itemize}

\end{frame}

%----------------------------------------------

\begin{frame}
\frametitle{SID - Voorbeelden}

\begin{itemize}
\item Driller (PWM)
\item Commando
\item Zoids (ring mod)
\end{itemize}

\end{frame}

%----------------------------------------------

\begin{frame}
\frametitle{Intermezzo}

\begin{block}{Muziek}
Raad welk muziekje bij welke computer/console hoort!
\end{block}

\begin{itemize}
\item IBM PC
\item NES
\item C64
\end{itemize}

\end{frame}

%----------------------------------------------

\begin{frame}
\frametitle{Intermezzo - Voorbeelden}

\begin{itemize}
\item Grand Prix Circuit (PC, C64)
\item Castlevania (PC, NES, C64)
\end{itemize}

\end{frame}

%----------------------------------------------

\begin{frame}
\frametitle{Intermezzo}

\begin{itemize}
\item IBM PC
\item NES
\item C64
\end{itemize}

\end{frame}

%----------------------------------------------

\begin{frame}
\frametitle{SID hacking}

\begin{itemize}
\item Click/pop-bug in 6581
\item 4-bit PCM digitaal
\end{itemize}

\end{frame}

%----------------------------------------------

\begin{frame}
\frametitle{SID hacking - Arkanoid}

\begin{figure}
\includegraphics[width=\linewidth]{images/farts.png}
\end{figure}

\end{frame}

%----------------------------------------------

\begin{frame}
\frametitle{SID hacking - 6581 examples}

\begin{tabular}{|l|l|l|}
\hline Year & Artist & Title \\ \hline
1987 & Martin Galway & Arkanoid \\
1987 & Chris H\"ulsbeck & Dulcedo Cogitationis \\
1988 & Jeroen Tel & Savage \\
1989 & Jeroen Tel & Stormlord \\
1989 & Jeroen Tel & Turbo Outrun \\
1990 & J. Tel and C. H\"ulsbeck & Turrican \\ \hline
\end{tabular}

\end{frame}
