%
% Naam: Folkert van Verseveld
% UvAnetID: 11839341
% Contact:
%   @: folkert.van.verseveld@gmail.com
%   @: folkert.vanverseveld@students.uva.nl
%

%%%%%%%%%%%%%%%%%%%%%%%%%%%%%%
% LATEX-TEMPLATE PRESENTATIE
%-------------------------------------------------------------------------------
% Voor informatie over presenterenn, zie
% http://practicumav.nl/presenteren/presenteren.html
% Voor readme en meest recente versie van het template, zie
% https://gitlab-fnwi.uva.nl/informatica/LaTeX-template.git
% Gebaseerd op een template van: http://www.LaTeXTemplates.com
% Licentie: CC BY-NC-SA 3.0
% (http://creativecommons.org/licenses/by-nc-sa/3.0/)
%%%%%%%%%%%%%%%%%%%%%%%%%%%%%%

%-------------------------------------------------------------------------------
%	PACKAGES EN CONFIGURATIE
%-------------------------------------------------------------------------------

\documentclass[aspectratio=43]{uva-inf-presentation}
\usepackage[dutch]{babel}
%\usepackage{qtree}
\usepackage{multicol}
\usepackage{amssymb}
\usepackage{listings}

\title{Programmeren in Beperkte Omgevingen}
% Easter egg: Commodore Business Machines logo
\course{C=}
\assignment{Just 4 Fun}
\assignmenttype{Hacking}
\authors{Folkert van Verseveld}
\uvanetids{11839341}
% XXX Tutor, mentor en docent niet echt van toepassing denk ik?
\tutor{Teun Mathijssen}
\mentor{}
\docent{Robert van Wijk}
\group{Assembly A2}

\begin{document}

\begin{titelframe}
\titlepage

\end{titelframe}

\begin{frame}
\frametitle{Inhoudsopgave}
\tableofcontents
\end{frame}

%-------------------------------------------------------------------------------
%	PRESENTATIE SLIDES
%-------------------------------------------------------------------------------

% Secties zijn er om structuur in je presentatie aan te brengen
% en worden direct in de inhoudsopgave opgenomen
\section{Introductie}
% Maak een subsection voor een serie slides met een gedeeld thema
\section{Programmeeromgeving}
\section{Intermezzo's en Opdrachten}

\section{Conclusies}

%----------------------------------------------

\begin{frame}
\frametitle{Beperkte Omgevingen}

\begin{itemize}
\item Allereerste `computers'
\item Militaire toepassingen
\item Kamersgrote machines en torenhoge energierekeningen
\end{itemize}

\end{frame}

%----------------------------------------------

\begin{frame}
\frametitle{Allereerste `computers'}

\begin{itemize}
\item Rekentabellen (sin, cos)
\item Administratie (statistiek, belasting)
\end{itemize}

\end{frame}

%----------------------------------------------

\begin{frame}
\frametitle{Rekentabellen}

\begin{figure}
\includegraphics[width=0.75\linewidth]{images/rekentabellen.jpg}
\end{figure}

\end{frame}

%----------------------------------------------

\begin{frame}
\frametitle{Militaire toepassingen}

\begin{itemize}
\item Artillerie (boog van kogel)
\item Cryptografie (enigma)
\item OXO
\end{itemize}

\end{frame}

%----------------------------------------------

\begin{frame}
\frametitle{Bombe}

\begin{figure}
\includegraphics[width=0.75\linewidth]{images/bombe.jpg}
\end{figure}

\end{frame}

%----------------------------------------------

\begin{frame}
\frametitle{EDSAC}

\begin{figure}
\includegraphics[width=0.75\linewidth]{images/edsac.jpg}
\end{figure}

\end{frame}

%----------------------------------------------

\begin{frame}
\frametitle{EDSAC I/O}

\begin{figure}
\includegraphics[width=0.75\linewidth]{images/edsac_crts.jpg}
\end{figure}

\end{frame}

%----------------------------------------------

\begin{frame}
\frametitle{OXO}

\begin{figure}
\includegraphics[width=0.75\linewidth]{images/oxo.png}
\end{figure}
% http://www.dcs.warwick.ac.uk/~edsac/
% eventueel OXO in emulator laten zien.

\end{frame}

%----------------------------------------------

% XXX add edsac geschiedenis & specs?

%----------------------------------------------

\begin{frame}
\frametitle{Mainframes}

\begin{itemize}
\item Multi-user omgeving
\item Batch
\item Reserveren
\item Onpraktisch
\item Heel duur: \$100,000+ / \$1,000,000+
\item Gebruikersonvriendelijk
\end{itemize}

\end{frame}

%----------------------------------------------

\begin{frame}
\frametitle{Mini-mainframes}

\begin{figure}
\includegraphics[width=0.45\linewidth]{images/pdp8.jpg}
\end{figure}

\end{frame}

%----------------------------------------------

\begin{frame}
\frametitle{Mini-mainframes}

\begin{itemize}
\item Duur, maar niet onbetaalbaar: \$10000+
\item Acceptabele performance
\item `Gewone' klanten
\end{itemize}

\end{frame}

%----------------------------------------------

\begin{frame}
\frametitle{Moore's law}

\begin{itemize}
\item Rekenmachines steeds sneller en goedkoper
\item Hoe goedkoper, des te meer toepassingen
\end{itemize}

\end{frame}

%----------------------------------------------

\begin{frame}
\frametitle{Introductie van de Personal Computer}

\begin{itemize}
\item Mini mainframe
\item `Goedkoop'
\item `Gebruikersvriendelijk'
\item Eerste echte PCs rond 1977-1982
\end{itemize}

\end{frame}

%----------------------------------------------

\begin{frame}
\frametitle{Eerste computers}

\begin{figure}
\includegraphics[width=0.75\linewidth]{images/trs80.jpg}
\end{figure}

\end{frame}

%----------------------------------------------

\begin{frame}
\frametitle{Eerste computers}

\begin{figure}
\includegraphics[width=0.6\linewidth]{images/appleii.jpg}
\end{figure}

\end{frame}

%----------------------------------------------

\begin{frame}
\frametitle{Eerste computers}

\begin{figure}
\includegraphics[width=0.6\linewidth]{images/vic20.jpg}
\end{figure}

\end{frame}

%----------------------------------------------

\begin{frame}
\frametitle{Eerste computers}

\begin{figure}
\includegraphics[width=0.6\linewidth]{images/ibm5150.jpg}
\end{figure}

\end{frame}

%----------------------------------------------

\begin{frame}
\frametitle{Eerste computers}

\begin{figure}
\includegraphics[width=0.6\linewidth]{images/zxspectrum.jpg}
\end{figure}

\end{frame}

%----------------------------------------------

\begin{frame}
\frametitle{Eerste Game Consoles}

\begin{figure}
\includegraphics[width=0.75\linewidth]{images/vcs.jpg}
\end{figure}

\end{frame}

%----------------------------------------------

\begin{frame}
\frametitle{Eerste Game Consoles}

\begin{figure}
\includegraphics[width=0.75\linewidth]{images/nes.jpg}
\end{figure}

\end{frame}

%----------------------------------------------

\begin{frame}
\frametitle{Toepassingen van PCs}

\begin{itemize}
\item Kantoor gerelateerd (administratie, spreadsheets, \dots)
\item Consumentgericht
\item (Games)
\end{itemize}

\end{frame}

%----------------------------------------------

\begin{frame}
\frametitle{Ter Vergelijking}

\begin{tabular}{|l|l|l|l|l|}
\hline Jaar & Model & Prijs toen & Prijs omg. & Prijs nu \\ \hline
1977 & PET & \$795 & \$3150 & \$120-\$200 \\
1980 & VIC-20 & \$299.95 & \$874 & \$80-\$110 \\
1982 & C64 & \$595 & \$1480 & \$10-\$15 \\ \hline
1977 & Apple ][ & \$1296 & \$5130 & \$200-\$350 \\
1981 & IBM PC & \$1565 & \$4130 & \$1200-\$2000 \\
%IBM PC & 16-640 & Beeper & Ja & \$1565 \\
1982 & ZX Spec. & \pounds 175 & \pounds 567 & \pounds 100 \\ \hline
1983 & NES & \$179 & \$433 & \$40-\$80 \\ \hline
\end{tabular}

%\begin{tabular}{|l|l|l|l|l|}
%\hline Model & RAM & Geluid & \$ toen & \$ nu \\
%%C64 & 64K & AD Synth. & \$595 - \$135 & \$595 \\
%C64 & 64K & AD Synth. & \$595 & \$595 \\
%Apple ][ & 8-16K & Beeper & \$1296 & \$1296 \\
%IBM PC & 16K & Beeper & \$1565 & \$1565 \\
%%IBM PC & 16-640 & Beeper & Ja & \$1565 \\
%%ZX Spec. & 16 & Beeper & Nee & \pound 175 \\ \hline
%ZX Spec. & 16K & Beeper & \pounds 175 & \pounds 175 pound \\ \hline
%NES & 2K & Z80? & \$179 & \$433 \\ \hline
%\end{tabular}

\end{frame}

%----------------------------------------------

\begin{frame}
\frametitle{En harddisk opslag?}

\begin{itemize}
\item Niet nodig
\item Te duur (+\$500/+\$2000)
\end{itemize}

\end{frame}

%----------------------------------------------

\begin{frame}
\frametitle{Data-opslag}

\begin{itemize}
\item programma's in RAM, firmware in ROM
\item power loss
\end{itemize}

\end{frame}

%----------------------------------------------

\begin{frame}
\frametitle{Tapes!}

\begin{itemize}
\item Sequentieel magnetisch geheugen
\item Enkele MBs
\item Enkel-/dubbelzijdig
\end{itemize}

\end{frame}

%----------------------------------------------

\begin{frame}
\frametitle{Programma opzoeken}

\begin{itemize}
\item Onhandig
\item Handmatig heen en weer zoeken
\item Analoge file table
\end{itemize}

\end{frame}

%----------------------------------------------

\begin{frame}
\frametitle{Floppies!}

\begin{itemize}
\item Echt floppy
\item 8 inch, 5.25 inch
\item SD/DD
\item 300-600 RPM (5-10 revs/sec)
\item 0.1MB-2.0MB
\end{itemize}

\end{frame}

%----------------------------------------------

\begin{frame}
\frametitle{Ontmoet de Succesvolste PC ooit}

\begin{figure}
\includegraphics[width=0.75\linewidth]{images/desk.jpg}
\end{figure}

\end{frame}

%----------------------------------------------

\begin{frame}
\frametitle{De Enige Echte}

\begin{figure}
\includegraphics[width=0.6\linewidth]{images/c64.jpg}
\end{figure}

\end{frame}

%----------------------------------------------

\begin{frame}
\frametitle{De Enige Echte}

\begin{figure}
\includegraphics[width=0.6\linewidth]{images/c64c.jpg}
\end{figure}

\end{frame}

%----------------------------------------------

\begin{frame}
\frametitle{Moderne Variant}

\begin{figure}
\includegraphics[width=0.5\linewidth]{images/boot.png}
\end{figure}

\end{frame}

%----------------------------------------------

\begin{frame}
\frametitle{De Specificaties}

\begin{itemize}
\item Ge\"introduceerd in 1982
\item 64 kilobytes
\item MOS 6510 met 1 MHz kristalklok
\item 2 joystick poorten
\item cartridge port (games!), serial I/O, user port
\item 16 kleuren
\end{itemize}

\end{frame}

%----------------------------------------------

\begin{frame}
\frametitle{Ter Vergelijking}

\begin{tabular}{|l|l|l|l|}
\hline Model & CPU & MHz \\ \hline
PET & MOS 6502 & 1 \\ 
VIC-20 & MOS 6502 & 1 \\ 
C64 & MOS 6510 & 1 \\ \hline
Apple ][ & MOS 6502 & 1 \\
IBM PC & i8086 & 4.04 \\
ZX Spec. & Z80 & 3.5 \\ \hline
NES & Ricoh 2A03 & 1 \\ \hline
\end{tabular}

\end{frame}

%----------------------------------------------

\begin{frame}
\frametitle{MOS 6502}

\begin{itemize}
\item Zeer goedkoop en succesvol
\item 6502, 6510, ricoh 2a03, \dots
\item PET, VIC-20, C64, NES, \dots
\item Embedded systems
\end{itemize}

\end{frame}

%----------------------------------------------

\begin{frame}
\frametitle{MOS 6502}

\begin{figure}
\includegraphics[width=\linewidth]{images/terminator.jpg}
\end{figure}

\end{frame}

%----------------------------------------------

\begin{frame}
\frametitle{MOS 6502}

\begin{figure}
\includegraphics[width=0.6\linewidth]{images/bender.jpeg}
\end{figure}

\end{frame}

%----------------------------------------------

\begin{frame}
\frametitle{Wat kan een C64?}

\begin{itemize}
\item Administratie (spreadsheets, word, \dots)
\item Scannen, printen
\item Games!
\item Internet (via 50Kbps modem)
\item Hardware upgrades
\end{itemize}

\end{frame}

%----------------------------------------------

\begin{frame}
\frametitle{Waarom werken met Vintage PCs?}

\begin{itemize}
\item Enorme uitdaging
\item Simpel ontwerp, simpel te begrijpen
\item Literatuur
\item Programmeerwedstrijden (ja, 30 jaar lang!)
\end{itemize}

\end{frame}

%----------------------------------------------

\begin{frame}
\frametitle{Programmeertalen}

\begin{itemize}
\item Microsoft BASIC
\item Machine code (assembly)
\end{itemize}

\end{frame}

%----------------------------------------------

\begin{frame}
\frametitle{Hoe Programmeerden we vroeger?}

\begin{itemize}
\item Geen internet
\item Literatuur (tijdschriften, boeken, \dots)
\item Overlevering (bijeenkomsten, wedstrijden, \dots)
\end{itemize}

\end{frame}

%----------------------------------------------

\begin{frame}
\frametitle{Wat krijg je bij een C64?}

\begin{itemize}
\item Handleiding van ca. 300 pagina's
\item Programmeerhandleiding voor BASIC en assembly
\item Verwijzingen naar programma's, clubs, tijdschriften
\end{itemize}

\end{frame}

%----------------------------------------------

\begin{frame}
\frametitle{Gebruikershandleiding}

\begin{itemize}
\item Randapparaten aansluiten
\item Configuratie en commando's
\item BASIC programmeren (ca. 60 commando's)
\item Voorbeelden
\item Tips and tricks
\end{itemize}

\end{frame}

%----------------------------------------------

\begin{frame}
\frametitle{Programmeerhandleiding}

\begin{itemize}
\item Geheugenlayout
\item Assembly instructie set (151 documented)
\item Graphics modes
\item Sound effecten
\item Hardware uitbreidingen
\end{itemize}

\end{frame}

%----------------------------------------------

\begin{frame}
\frametitle{Literatuur}

\begin{figure}
\includegraphics[width=0.75\linewidth]{images/c64schema.jpg}
\end{figure}

\end{frame}

%----------------------------------------------

\begin{frame}
\frametitle{Literatuur}

\begin{figure}
\includegraphics[width=0.75\linewidth]{images/c64schema2.jpg}
\end{figure}

\end{frame}

%----------------------------------------------

\begin{frame}[fragile]{Simpel Programma - BASIC}

\begin{lstlisting}
10 PRINT "HALLO VIA"
20 GOTO 10
\end{lstlisting}

\end{frame}

%----------------------------------------------

\begin{frame}[fragile]{Simpel Programma - BASIC}

\begin{lstlisting}
10 POKE 1024,8
20 POKE 1025,5
30 POKE 1026,25
\end{lstlisting}

\end{frame}

%----------------------------------------------

\begin{frame}[fragile]{Simpel Programma - assembly}

\begin{lstlisting}
lda #$08
sta $400
lda #$05
sta $401
lda #$19
sta $402
rts
\end{lstlisting}

\end{frame}

%----------------------------------------------

\begin{frame}{Assembly}

\begin{itemize}
\item Simpel, maar diepere leercurve
\item 151 offici\"ele instructies
\item Kleine voorbeelden
\end{itemize}

\end{frame}

%----------------------------------------------

\begin{frame}{Assembly}

\begin{itemize}
\item lda \#\$10 Zet waarde \$10 in A (A = 16)
\item sta \$400 Schrijf A naar \$400 (mem[1000] = A)
\item rts return;
\end{itemize}

\end{frame}

%----------------------------------------------

\begin{frame}{Simpel Programma}

\begin{itemize}
\item Ga naar: https://skilldrick.github.io/easy6502/
\item Bekijk de voorbeelden.
\item Probeer zelf een programma te maken!
\end{itemize}


\end{frame}

%----------------------------------------------

\begin{frame}
\frametitle{Audio}

\begin{itemize}
\item Analoog of Digitaal
\end{itemize}

\end{frame}

%----------------------------------------------

\begin{frame}
\frametitle{Analoog Audio}

\begin{itemize}
\item Niet-elektronisch
\item Visualisatie met Oscilloscoop
\end{itemize}

\end{frame}

%----------------------------------------------

\begin{frame}
\frametitle{Digitaal Audio}

\begin{itemize}
\item Benadering van Analoog
\item Stapsgewijs
\item Duidelijk zichtbaar op oude PCs
\end{itemize}

\end{frame}

%----------------------------------------------

\begin{frame}
\frametitle{Digitaal Audio - Speaker}

\begin{itemize}
\item Zeer ruw
\item Aan of uit
\item Frequentie
\end{itemize}

\end{frame}

%----------------------------------------------

\begin{frame}
\frametitle{Digitaal Audio}

\begin{figure}
\includegraphics[width=0.9\linewidth]{images/rc3nes.png}
\end{figure}

\end{frame}

%----------------------------------------------

\begin{frame}
\frametitle{C64: SID chip}

\begin{itemize}
\item 's werelds eerste audio synthesizer
\item Analoog \emph{en} Digitaal
\item Verschillende onderscheidingen
\item Timbaland plagiaat
\end{itemize}

\end{frame}

%----------------------------------------------

\begin{frame}
\frametitle{Ter Vergelijking}

\begin{tabular}{|l|l|l|l|}
\hline Model & Geluid & Type & Voices \\ \hline
PET & Speaker & 1-bit beeper & 0 \\
%VIC-20 & Speaker? & 1-bit beeper & 0 \\
C64 & SID & A/D 8-bit synthesizer & 3 \\ \hline
Apple ][ & Speaker & 1-bit beeper & 0 \\
IBM PC & Speaker & 1-bit beeper & 0 \\
%ZX Spec. & 16K & Beeper & 175 pound & 175 pound \\ \hline
NES & custom & 8-bit digital & 5 \\ \hline
\end{tabular}

\end{frame}

%----------------------------------------------

\begin{frame}
\frametitle{Audio}

\begin{itemize}
\item 3 kanalen (voices)
\item ADSR envelope
\end{itemize}

\end{frame}

%----------------------------------------------

\begin{frame}
\frametitle{ADSR}

\begin{figure}
\includegraphics[width=0.8\linewidth]{images/adsr.png}
\end{figure}

\end{frame}

%----------------------------------------------

\begin{frame}
\frametitle{Waveforms}

\begin{itemize}
\item Sawtooth
\item Square
\item Sinuso\"ide
\item Triangle
\item Noise
\end{itemize}

\end{frame}

% TODO laat voorbeelden horen van elke waveform

%----------------------------------------------

\begin{frame}
\frametitle{Intermezzo}

\begin{block}{Muziek}
Raad welk muziekje bij welke computer/console hoort!
\end{block}

\end{frame}

%----------------------------------------------

\begin{frame}
\frametitle{Intermezzo}

\begin{itemize}
\item IBM PC
\item C64
\item NES
\end{itemize}

\end{frame}

%----------------------------------------------

\begin{frame}
\frametitle{Graphics}

\begin{figure}
\includegraphics[width=0.5\linewidth]{images/palette.png}
\end{figure}

\end{frame}

%----------------------------------------------

\begin{frame}
\frametitle{Graphics}

\begin{itemize}
\item 320x200 pixels
\item 16 kleuren palette
\item Maximaal 2 of 4 kleuren per 8x8 blok.
\end{itemize}

\end{frame}

%----------------------------------------------

\begin{frame}
\frametitle{Ter Vergelijking}

\begin{tabular}{|l|l|l|}
\hline Model & Video & Kleuren \\
VIC-20 & VIC I & 4-16??? \\
C64 & VIC II & 16 \\
Apple ][ & ??? & 1-4 \\
IBM PC & mono of CGA & 1-4 \\
ZX Spec. & geen (CPU) & 8 \\ \hline
NES & Ricoh 2C02 PPU & 52 \\ \hline
\end{tabular}

\end{frame}

%----------------------------------------------

\begin{frame}
\frametitle{Graphics}

\begin{figure}
\includegraphics[width=0.5\linewidth]{images/gfx.png}
\end{figure}

\end{frame}

%----------------------------------------------

\begin{frame}
\frametitle{Graphic modes}

\begin{itemize}
\item Standard Character (text) Mode
\item Multicolor Character (text) Mode
\item Standard Bitmap Mode
\item Multicolor Bitmap Mode
\item Extended Background Color Mode
\end{itemize}

\end{frame}

%----------------------------------------------

\begin{frame}
\frametitle{Multicolor Character Mode}

\begin{figure}
\includegraphics[width=0.5\linewidth]{images/gary.png}
\end{figure}

\end{frame}

%----------------------------------------------

\begin{frame}
\frametitle{Standard Bitmap}

\begin{figure}
\includegraphics[width=0.5\linewidth]{images/sbm_obey.png}
\end{figure}

\begin{center}
Source: Archmage
\end{center}

\end{frame}

%----------------------------------------------

\begin{frame}
\frametitle{Standard Bitmap}

\begin{figure}
\includegraphics[width=0.5\linewidth]{images/sbm_tuksu.png}
\end{figure}

\begin{center}
Source: Duce
\end{center}

\end{frame}

%----------------------------------------------

\begin{frame}
\frametitle{Programmeerwedstrijden}

\begin{itemize}
\item X (in Nederland!)
\item Revision (grootste ter wereld!)
\item Nordlicht
\item Under Construction
\item \dots
\end{itemize}

\end{frame}

%----------------------------------------------

\begin{frame}
\frametitle{X}

\begin{figure}
\includegraphics[width=0.8\linewidth]{images/x.jpg}
\end{figure}

\end{frame}

%----------------------------------------------

\begin{frame}
\frametitle{X}

\begin{itemize}
\item sinds 1995
\item Vanaf 2004 \'e\'en keer per 2 jaar
\item Vorig jaar ong. 400 bezoekers
\end{itemize}

\end{frame}

%----------------------------------------------

\begin{frame}
\frametitle{Revision}

\begin{figure}
\includegraphics[width=0.8\linewidth]{images/rev2017.jpg}
\end{figure}

\end{frame}

%----------------------------------------------

\begin{frame}
\frametitle{Revision}

\begin{itemize}
\item Sinds 2011 (daarvoor: Breakpoint)
\item Ong. 700 bezoekers
\item Grote bedrijven bieden banen aan!
\item Seminars
\item 30+ wedstrijden (demo, gfx, fotografie, \dots)
\end{itemize}

\end{frame}

%----------------------------------------------

\begin{frame}
\frametitle{Opdracht}

\begin{itemize}
\item Vorm groepen van maximaal 5
\item Maak een demo!
\end{itemize}

\end{frame}

%----------------------------------------------

\begin{frame}
\frametitle{Wil je een C64?}

\begin{itemize}
\item Dat kan!
\item Duur op ebay en marktplaats
\item Goedkoop bij rommelmarkt
\item Toetsenbord
\item HCC
\end{itemize}

\end{frame}

%----------------------------------------------

\begin{frame}
\frametitle{HCC}

\begin{itemize}
\item Nederlandse Hobby Computer Club
\item Sinds 27 april 1977
\item Diverse interessegroepen
\item Elke twee maanden in Maarssen
\item Retro (8/16 bit) computers welkom
\item Ultimate II+
\item Swappen
\end{itemize}

\end{frame}

%------------------------------------------------

\begin{frame}
\frametitle{Vervolg}

\begin{itemize}
\item Demoscene
\item Oldskool computers
\item Programmeerwedstrijden
\item Cracking 'n Hacking
\end{itemize}

\end{frame}

%------------------------------------------------

\begin{frame}
\frametitle{Conclusies}

\begin{itemize}
\item PC Geschiedenis
\item Hacken op C64
\item Programmeren tot het uiterste
\item Programmeerwedstrijden
\end{itemize}

\end{frame}

%------------------------------------------------

\begin{frame}
\frametitle{Bronvermeldingen}

\begin{multicols}{2}
\begin{itemize}
\item archive.org
\item c64-wiki.com
\item codebase64.org
\item commodore.hcc.nl
\item csdb.dk
\item dcs.warwick.ac.uk
\item demoparty.net
\item derbian.webs.com
\item devil.iki.fi
\item freeinfosociety.com
\item id.scene.org
\item ieee.org
\end{itemize}
\end{multicols}

\end{frame}

%------------------------------------------------

\begin{frame}
\frametitle{Bronvermeldingen}

\begin{multicols}{2}
\begin{itemize}
\item musictechstudent.co.uk
\item oldcomputers.net
\item pagetable.com
\item pouet.net
\item revision-party.net
\item scs-trc.net
\item sta.c64.org
\item vice.sourceforge.net
\item wikipedia.org
\end{itemize}
\end{multicols}

\end{frame}

%------------------------------------------------

%------------------------------------------------

\begin{frame}[noframenumbering]
\frametitle{Bronvermeldingen}

\begin{multicols}{2}
\begin{itemize}
\item archive.org
\item bolo.ch
\item c64-wiki.com
\item codebase64.org
%http://www.commodore.ca/manuals/c64_programmers_reference/c64-programmers_reference.htm
\item commodore.ca
\item commodore.hcc.nl
\item csdb.dk
\item dcs.warwick.ac.uk
\item demoparty.net
\item derbian.webs.com
\item devil.iki.fi
\item dmshas.de
\item famitracker.com
\item freeinfosociety.com
\item granishmusicproduction. com
\item history-computer.com
\item id.scene.org
\item ieee.org
\end{itemize}
\end{multicols}

\end{frame}

%------------------------------------------------

\begin{frame}[noframenumbering]
\frametitle{Bronvermeldingen}

\begin{multicols}{2}
\begin{itemize}
\item musictechstudent.co.uk
\item oldcomputers.net
\item pagetable.com
\item pouet.net
\item revision-party.net
\item scs-trc.net
\item sta.c64.org
\item vice.sourceforge.net
\item wikipedia.org
\end{itemize}
\end{multicols}

\end{frame}
%------------------------------------------------

\begin{frame}[noframenumbering]
\frametitle{Met dank aan}

\begin{itemize}
\item Lezingencommissie
\item VIA
\item CCC
\item HCC, lemon64, csdb, VICE
\item Hobbyisten en C64 fans
\item Scene
\item \'Ecole Polytechnique F\'ed\'erale de Lausanne
% trs80 image:
\item Mus\'ee Bolo
\end{itemize}

\end{frame}

%------------------------------------------------

\begin{frame}[noframenumbering]
\frametitle{Met dank aan - HCC en Sceners}

\begin{itemize}
\item Duncan
\item fgenesis
\item fieserWolf
\item Ron van Schaik
\item Wilfred Bos
\end{itemize}

\end{frame}

%------------------------------------------------

\begin{frame}[noframenumbering]
\frametitle{Met dank aan - Voorproef en Feedback}

\begin{itemize}
\item Bert
\item Boaz
\item Brian
\item Dennis
\item Jelle
\item Kees
\item Otto
\item Sander
\item Teun
\item Tjaard
\end{itemize}

\end{frame}

%------------------------------------------------

\begin{frame}[noframenumbering]
\frametitle{Met dank aan - gfx}

\begin{multicols}{2}
\begin{itemize}
% breadbin comic
\item Bart van Tieghem
% vectrex
\item Evan-Amos
\item Bill Bertram
% CRT raster scan
\item Ian Harvey
\item Jozef Galanda
% bleeding MSX
\item Marco L
% C1541 drive
\item Nathan Beach
% waveforms
\item Omegatron
% trs80 image author:
\item Rama
\item Ruben de Rijcke
% C64 back size
\item Sammler
% great giana sisters - sprite vergroesserung
\item Werner
\end{itemize}
\end{multicols}

\end{frame}

%------------------------------------------------

\begin{frame}[noframenumbering]
\frametitle{Met dank aan de Sceners}

\begin{multicols}{3}
\begin{itemize}
\item Abyss Connection
\item Abnormal
\item Albion Crew
\item Alpha Flight
\item Algotech
\item Arkanix Labs
\item Artline Designs
\item Arise
\item Artstate
\item Arsenic
\item Atlantic
\item Atlantis
\item Atw
\item Bauknecht
\item Black Sun
\item Bluez Muz
\item Bonzai
\item Booze Design
\item Byterapers
\item Beyond Force
\end{itemize}
\end{multicols}

\end{frame}

%------------------------------------------------

\begin{frame}[noframenumbering]
\frametitle{Met dank aan de Sceners}

\begin{multicols}{3}
\begin{itemize}
\item Camelot
\item Cascade
\item Censor
\item Chorus
\item Chrome
\item Cosine
\item Covert Bitops
\item Creators
\item Crescent
\item Crest
\item Darklite
\item Delysid
\item Dekadence
\item Desire
\item Dmagic
\item Dual Crew
\item Elysium
\item Extend
\item Excess
\item Exon
\item Fairlight
\item Fatzone
\item Focus
\item Fossil
\end{itemize}
\end{multicols}

\end{frame}

%------------------------------------------------

\begin{frame}[noframenumbering]
\frametitle{Met dank aan de Sceners}

\begin{multicols}{3}
\begin{itemize}
\item Genesis Project
\item Glance
\item Hack 'n 'trade
\item Hitman
\item Hoaxers
\item Holger
\item Hokuto Force
\item Horizon
\item House Designs
\item HVSC Crew
\item Judas
\item Laxity
\item Lepsi De
\item Level 64
\item LFT
\item Mahoney
\item Maniacs of Noise
\item Mayday
\item MDG
\end{itemize}
\end{multicols}

\end{frame}

%------------------------------------------------

\begin{frame}[noframenumbering]
\frametitle{Met dank aan de Sceners}

\begin{multicols}{3}
\begin{itemize}
\item Metalvotze
\item Miracles
\item Multistyle Labs
\item The New Dimension
\item Nah-Kolor
\item Noice
\item No Name
\item Nostalgia
\item Nuance
\item Offence
\item Onslaught
\item Oxyron
\item Padua
\item Panda Design
\item Plush
\item Poo-brain
\item Prosonix
\item Radwar
\item Razor
\item Reflex
\item Resource
\item Samar
\item Scene sat
\item Shape
\item Sidrip
\end{itemize}
\end{multicols}

\end{frame}

%------------------------------------------------

\begin{frame}[noframenumbering]
\frametitle{Met dank aan de Sceners}

\begin{multicols}{3}
\begin{itemize}
\item Singular
\item Slay Radio
\item Smash Designs
\item Starion
\item Style
\item SCS*TRC
\item TPUG
\item Triad
\item Tropyx
\item TRSI
\item Unicess
\item Up Rough
\item Vandalism News
\item Vibrants
\item VICE team
\item Vision
\item Wow
\item Wrath Designs
\item Xenon
\item You Lazy Bastards
\end{itemize}
\end{multicols}

\end{frame}


%------------------------------------------------

\begin{frame}
\frametitle{Bedankt voor jullie aandacht}
\Large{\centerline{Zijn er nog vragen?}}
\end{frame}

\end{document}
