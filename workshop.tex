%
% Naam: Folkert van Verseveld
% UvAnetID: 11839341
% Contact:
%   @: folkert.van.verseveld@gmail.com
%   @: folkert.vanverseveld@students.uva.nl
%

\documentclass[aspectratio=43]{uva-inf-presentation}
\usepackage[dutch]{babel}
\usepackage{multicol}
\usepackage{amssymb}
\usepackage{listings}
\usepackage{subcaption}

\title{Programmeren in Beperkte Omgevingen}
% Easter egg: Commodore Business Machines logo
\course{C=}
\assignment{Just 4 Fun}
\assignmenttype{Hacking}
\authors{Folkert van Verseveld}
\uvanetids{11839341}
% XXX Tutor, mentor en docent niet echt van toepassing denk ik?
\tutor{Teun Mathijssen}
\mentor{}
\docent{Robert van Wijk}
\group{Assembly A2}

\begin{document}

\begin{titelframe}
\titlepage

\end{titelframe}

\begin{frame}[noframenumbering]
\frametitle{Inhoudsopgave}
\tableofcontents
\end{frame}

\section{Introductie}
\section{Programmeeromgeving}
\section{Programmeren in BASIC}
\section{Graphics}
\section{Muziek}
\section{Intermezzo's en Opdrachten}

\section{Conclusies}

%----------------------------------------------

\begin{frame}[noframenumbering]
\frametitle{Beperkte Omgevingen}

\begin{itemize}
\item Allereerste `computers'
\item Militaire toepassingen
\item Kamersgrote machines en torenhoge energierekeningen
\end{itemize}

\end{frame}

%----------------------------------------------

\begin{frame}
\frametitle{Allereerste `computers'}

\begin{itemize}
\item Rekentabellen (sin, cos)
\item Administratie (statistiek, belasting)
\end{itemize}

\end{frame}

%----------------------------------------------

\begin{frame}
\frametitle{Rekentabellen}

\begin{figure}
\includegraphics[width=0.75\linewidth]{images/rekentabellen.jpg}
\end{figure}

\end{frame}

%----------------------------------------------

\begin{frame}
\frametitle{Militaire toepassingen}

\begin{itemize}
\item Artillerie (boog van kogel)
\item Cryptografie (enigma)
\item OXO
\end{itemize}

\end{frame}

%----------------------------------------------

\begin{frame}
\frametitle{Bombe}

\begin{figure}
\includegraphics[width=0.75\linewidth]{images/bombe.jpg}
\end{figure}

\end{frame}

%----------------------------------------------

\begin{frame}
\frametitle{EDSAC}

\begin{figure}
\includegraphics[width=0.75\linewidth]{images/edsac.jpg}
\end{figure}

\end{frame}

%----------------------------------------------

\begin{frame}
\frametitle{EDSAC Debugging}

\begin{figure}
\includegraphics[width=0.75\linewidth]{images/edsac_crts.jpg}
\end{figure}

\end{frame}

%----------------------------------------------

\begin{frame}
\frametitle{OXO}

\begin{figure}
\includegraphics[width=0.75\linewidth]{images/oxo.jpg}
\end{figure}
% http://www.dcs.warwick.ac.uk/~edsac/
% eventueel OXO in emulator laten zien.

\end{frame}

%----------------------------------------------

\begin{frame}
\frametitle{Mainframes}

\begin{itemize}
\item Multi-user omgeving
\item Batch
\item Reserveren
\item Onpraktisch
\item Heel duur: \$100,000+ / \$1,000,000+
\item Gebruikersonvriendelijk
\end{itemize}

\end{frame}

%----------------------------------------------

\begin{frame}
\frametitle{Mini-mainframes}

\begin{figure}
\includegraphics[width=0.45\linewidth]{images/pdp8.jpg}
\end{figure}

\end{frame}

%----------------------------------------------

\begin{frame}
\frametitle{Mini-mainframes}

\begin{itemize}
\item Duur, maar niet onbetaalbaar: \$10,000+
\item Acceptabele performance
\item `Gewone' klanten
\end{itemize}

\end{frame}

%----------------------------------------------

\begin{frame}
\frametitle{Introductie van de Personal Computer}

\begin{itemize}
\item Mini mainframe
\item `Goedkoop'
\item `Gebruikersvriendelijk'
\item Eerste echte PCs rond 1977-1982
\end{itemize}

\end{frame}

%----------------------------------------------

\begin{frame}
\frametitle{Eerste Personal Computers}

\begin{figure}
\includegraphics[width=0.6\linewidth]{images/appleii.jpg}
\end{figure}

\end{frame}

%----------------------------------------------

\begin{frame}
\frametitle{Eerste Personal Computers}

\begin{figure}
\includegraphics[width=0.75\linewidth]{images/trs80.jpg}
\end{figure}

\end{frame}

%----------------------------------------------

\begin{frame}
\frametitle{Eerste Personal Computers}

\begin{figure}
\includegraphics[width=0.6\linewidth]{images/pet.jpg}
\end{figure}

\end{frame}

%----------------------------------------------

\begin{frame}
\frametitle{Eerste Personal Computers}

\begin{figure}
\includegraphics[width=0.6\linewidth]{images/vic20.jpg}
\end{figure}

\end{frame}

%----------------------------------------------

\begin{frame}
\frametitle{Eerste Personal Computers}

\begin{figure}
\includegraphics[width=0.6\linewidth]{images/ibm5150.jpg}
\end{figure}

\end{frame}

%----------------------------------------------

\begin{frame}
\frametitle{Eerste Personal Computers}

\begin{figure}
\includegraphics[width=0.6\linewidth]{images/zxspectrum.jpg}
\end{figure}

\end{frame}

%----------------------------------------------

\begin{frame}
\frametitle{Eerste Game Consoles}

\begin{figure}
\includegraphics[width=0.75\linewidth]{images/vcs.jpg}
\end{figure}

\end{frame}

%----------------------------------------------

\begin{frame}
\frametitle{Eerste Game Consoles}

\begin{figure}
\includegraphics[width=0.5\linewidth]{images/vectrex.jpg}
\end{figure}

\end{frame}

%----------------------------------------------

\begin{frame}
\frametitle{Eerste Game Consoles}

\begin{figure}
\includegraphics[width=0.75\linewidth]{images/nes.jpg}
\end{figure}

\end{frame}

%----------------------------------------------

\begin{frame}
\frametitle{Ter Vergelijking}

\begin{tabular}{|l|l|l|l|l|}
\hline Jaar & Model & Toen & Omg. & Nu (ebay) \\ \hline
1977 & PET & \$795 & \$3150 & \$600-\$800 \\
1980 & VIC-20 & \$299.95 & \$874 & \$30-\$110 \\
1982 & C64 & \$595 & \$1480 & \$10-\$50 \\ \hline
1977 & Apple ][ & \$1296 & \$5130 & \$200-\$750 \\
1981 & IBM PC & \$1565 & \$4130 & \$150-\$1100 \\
1982 & ZX Spec. & \pounds 175 & \pounds 567 & \pounds 80-\pounds 120 \\ \hline
1977 & VCS & \$199 & \$768 & \$40-\$80 \\
1982 & Vectrex & \$199 & \$495 & \$400-\$600 \\
1983 & NES & \$179 & \$433 & \$25-\$120 \\ \hline
\end{tabular}

\end{frame}

%----------------------------------------------

\begin{frame}
\frametitle{En harddisk opslag?}

\begin{itemize}
\item Niet nodig
\item Te duur (+\$500/+\$2,000)
\item Programma's in RAM, firmware in ROM
\item Power loss: programma weg!
\end{itemize}

\end{frame}

%----------------------------------------------

\begin{frame}
\frametitle{Tapes!}

\begin{figure}
\includegraphics[width=0.6\linewidth]{images/datasette.jpg}
\end{figure}

\end{frame}

%----------------------------------------------

\begin{frame}
\frametitle{Tapes!}

\begin{itemize}
\item Sequentieel magnetisch geheugen
\item 450KB-2000KB
\item Enkel-/dubbelzijdig
\item Onhandig
\item Handmatig heen en weer zoeken
\item Analoge file table
\end{itemize}

\end{frame}

%----------------------------------------------

\begin{frame}
\frametitle{Floppies!}

\begin{figure}
\includegraphics[width=0.7\linewidth]{images/c1541.jpg}
\end{figure}

\end{frame}

%----------------------------------------------

\begin{frame}
\frametitle{Floppies!}

\begin{itemize}
\item Echt floppy
\item 5.25 inch
\item SD
\item 300 RPM (5 revs/sec)
\item 186KB
\end{itemize}

\end{frame}

%----------------------------------------------

\begin{frame}
\frametitle{Ontmoet de Succesvolste PC ooit}

\begin{figure}
\includegraphics[width=0.75\linewidth]{images/desk.jpg}
\end{figure}

\end{frame}

%----------------------------------------------

\begin{frame}
\frametitle{De Enige Echte - Broodtrommel}

\begin{figure}
\includegraphics[width=0.6\linewidth]{images/c64.jpg}
\end{figure}

\end{frame}

%----------------------------------------------

\begin{frame}
\frametitle{De Enige Echte - Broodtrommel}

\begin{figure}
\includegraphics[width=0.75\linewidth]{images/breadbin.jpg}
\end{figure}

\end{frame}

%----------------------------------------------

\begin{frame}
\frametitle{De Enige Echte - C64C}

\begin{figure}
\includegraphics[width=0.6\linewidth]{images/c64c.jpg}
\end{figure}

\end{frame}

%----------------------------------------------

\begin{frame}
\frametitle{Moderne Variant}

\begin{figure}
\includegraphics[width=0.5\linewidth]{images/boot.png}
\end{figure}

\end{frame}

%----------------------------------------------

\begin{frame}
\frametitle{Ter Vergelijking}

\begin{tabular}{|l|l|l|l|l|}
\hline Model & RAM (bytes) & CPU & MHz \\ \hline
PET & 8192 & MOS 6502 & 1 \\
VIC-20 & 5120 & MOS 6502 & 1.11 \\
C64 & 65536 & MOS 6510 & 0.985 \\ \hline
Apple ][ & 4096 & MOS 6502 & 1 \\
IBM PC & 16384 & i8086 & 4.04 \\
ZX Spectrum & 16384 & Z80 & 3.5 \\ \hline
VCS & 128 & MOS 6507 & 1.19 \\
Vectrex & 1024 & MC68A09 & 1.5 \\
NES & 2048 & Ricoh 2A03 & 1.6 \\ \hline
\end{tabular}

\end{frame}

%----------------------------------------------

\begin{frame}
\frametitle{MOS 6502}

\begin{itemize}
\item Zeer goedkoop en succesvol
\item 6502, 6507, 6510, ricoh 2a03, \dots
\item PET, VIC-20, C64, VCS, NES, \dots
\item Embedded systems
\end{itemize}

\end{frame}

%----------------------------------------------

\begin{frame}
\frametitle{MOS 6502}

\begin{figure}
\includegraphics[width=\linewidth]{images/terminator.jpg}
\end{figure}

\end{frame}

%----------------------------------------------

\begin{frame}
\frametitle{MOS 6502}

\begin{figure}
\includegraphics[width=0.6\linewidth]{images/bender.jpeg}
\end{figure}

\end{frame}

%----------------------------------------------

\begin{frame}
\frametitle{Wat kan een C64?}

\begin{figure}
\includegraphics[width=0.8\linewidth]{images/c64back.png}
\end{figure}

\end{frame}

%----------------------------------------------

\begin{frame}
\frametitle{Wat kan een C64?}

\begin{figure}
\includegraphics[width=0.8\linewidth]{images/performance.png}
\end{figure}

\end{frame}

%----------------------------------------------

\begin{frame}
\frametitle{Waarom werken met Vintage PCs?}

\begin{itemize}
\item Enorme uitdaging
\item Simpel ontwerp, simpel te begrijpen
\item Literatuur
\item Programmeerwedstrijden (ja, 30 jaar lang!)
\end{itemize}

\end{frame}

%----------------------------------------------

\begin{frame}
\frametitle{C64 wordt nog steeds gebruikt!!}

\begin{figure}
\includegraphics[width=0.75\linewidth]{images/c64c_garage.jpg}
\end{figure}

\end{frame}

%----------------------------------------------

\begin{frame}
\frametitle{Hoe Programmeerden we vroeger?}

\begin{itemize}
\item Geen internet
\item Literatuur (tijdschriften, boeken, \dots)
\item Overlevering (bijeenkomsten, wedstrijden, \dots)
\item `Download' programma's van radio naar tape!
\end{itemize}

\end{frame}

%----------------------------------------------

\begin{frame}[fragile]{Simpel Programma - BASIC}

\begin{lstlisting}
10 PRINT "HALLO VIA"
20 GOTO 10
\end{lstlisting}

\end{frame}

%----------------------------------------------

\begin{frame}[fragile]{Simpel Programma - BASIC}

\begin{lstlisting}
10 FOR I=0 TO 16
20 PRINT 2^I, I
30 NEXT I
\end{lstlisting}

\end{frame}

%----------------------------------------------

% is the same as
\begin{frame}[fragile]{Simpel Programma - BASIC}

\begin{lstlisting}
10 FORI=0TO16
20 PRINT2^I,I
30 NEXTI
\end{lstlisting}

\end{frame}

%----------------------------------------------

% as well as
\begin{frame}[fragile]{Simpel Programma - BASIC}

\begin{lstlisting}
10FORI=0TO16:PRINT2^I,I:NEXTI
\end{lstlisting}

\end{frame}

%----------------------------------------------

\begin{frame}{Intermezzo}

\begin{itemize}
\item Ga naar https://www.github.com/FolkertVanVerseveld
\item Klik op `workshop' en dan op `releases'
\item Download het release archief en pak het uit
\item Probeer wat opdrachten te maken
\item Pas de code aan of probeer zelf een BASIC programma te schrijven
\end{itemize}

\end{frame}

%----------------------------------------------

\begin{frame}{Assembly}

\begin{itemize}
\item Simpel, maar diepere leercurve
\item 151 offici\"ele instructies
\item Kleine voorbeelden
\end{itemize}

\end{frame}

%----------------------------------------------

\begin{frame}[fragile]{Simpel Programma - BASIC}

\begin{lstlisting}
10 POKE 1024,8
20 POKE 1025,5
30 POKE 1026,25
40 END
\end{lstlisting}

\end{frame}

%----------------------------------------------

\begin{frame}[fragile]{Simpel Programma - assembly}

\begin{lstlisting}
* = $0810
lda #8
sta 1024
lda #5
sta 1025
lda #25
sta 1026
rts
\end{lstlisting}

\end{frame}

%----------------------------------------------

\begin{frame}{Simpel Programma - assembly}

\begin{tabular}{l|l|l|l}
Addr. & Mach. Code & Dissassembly & Pseudo Code \\
 & & * = \$0810 & \\
0810 & A9 08 & LDA \#\$08 & A = `H' \\
0812 & 8D 00 04 & STA \$400 & mem[1024] = A \\
0815 & A9 05 & LDA \#\$05 & A = `E' \\
0817 & 8D 01 04 & STA \$401 & mem[1025] = A \\
081A & A9 19 & LDA \#\$19 & A = `Y' \\
081C & 8D 02 04 & STA \$402 & mem[1026] = A \\
081F & 60 & RTS & return \\
\end{tabular}

\end{frame}

%----------------------------------------------

\begin{frame}
\frametitle[fragile]{Graphics}

\begin{figure}
	\begin{subfigure}[b]{0.4\textwidth}
		\includegraphics[width=\linewidth]{images/sbm_obey.png}
	\end{subfigure}
	\begin{subfigure}[b]{0.5\textwidth}
		\includegraphics[width=\linewidth]{images/nyan.png}
	\end{subfigure}
\end{figure}

\end{frame}

%----------------------------------------------

\begin{frame}
\frametitle{Graphics - Analoog}

\begin{itemize}
\item Vectrex
\item Geen pixels, maar SVG
\item Monochroom: helderheid
\end{itemize}

\end{frame}

%----------------------------------------------

\begin{frame}
\frametitle{Ter Vergelijking}

\begin{tabular}{|l|l|l|}
\hline Model & Video & Kleuren \\
PET & geen (CPU) & 1 \\
VIC-20 & VIC & 16 \\
C64 & VIC II & 16 \\
Apple ][ & geen (CPU, TTL) & 1 of 4 \\
IBM PC & mono of CGA & 1 of 4 \\
ZX Spec. & geen (CPU) & 8 \\ \hline
VCS & TIA & verschilt \\
Vectrex & VIA & geen: helderheid \\
NES & Ricoh 2C02 PPU & 52 \\ \hline
\end{tabular}

\end{frame}

%----------------------------------------------

\begin{frame}
\frametitle{Graphics - C64}

\begin{figure}
\includegraphics[width=0.5\linewidth]{images/palette.png}
\end{figure}

\end{frame}

%----------------------------------------------

\begin{frame}
\frametitle{Graphics Memory}

\begin{itemize}
\item 320x200 = 64,000
\item Maar 65,536 - 64,000 = 1,536 bytes over met bitmap
\item Hoe lossen we dit op?
\end{itemize}

\end{frame}

%----------------------------------------------

\begin{frame}
\frametitle{Graphics Memory}

\begin{itemize}
\item Gebruik 8x8 blokken
\item Tekst modus
\item Maar 40 * 25 = 1,000 bytes nodig
\item Met 16 colors maar 1,000 / 2 = 500 bytes meer
\item 1,500 bytes
\end{itemize}

\end{frame}

%----------------------------------------------

\begin{frame}
\frametitle{PETSCII - Tekst Mode}

\begin{figure}
\includegraphics[width=0.5\linewidth]{images/gary.png}
\end{figure}

\end{frame}

%----------------------------------------------

\begin{frame}
\frametitle{Standard Bitmap}

\begin{figure}
\includegraphics[width=0.5\linewidth]{images/sbm_tuksu.png}
\end{figure}

\begin{center}
Source: Duce
\end{center}

\end{frame}

%----------------------------------------------

\begin{frame}[fragile]{Simpel Graphics Programma - BASIC}

\begin{lstlisting}
10 VIC=53248
20 C=PEEK(VIC+32)AND15
30 PRINT"KLEURCODE=",C
40 C=(C+3)AND15
50 POKEVIC+32,C
60 PRINT"KLEURCODE=",C
\end{lstlisting}

\end{frame}

%----------------------------------------------

\begin{frame}[fragile]{Simpel Graphics Programma - BASIC}

\begin{lstlisting}
10 VIC=53248
20 FORC=0TO15
30 POKEVIC+32,C
40 FORT=1TO128:NEXT
50 NEXTC
\end{lstlisting}

\end{frame}

%----------------------------------------------

\begin{frame}
\frametitle{Hoe plaatsen we individuele pixels?}

\begin{itemize}
\item Onhandig: Grid/blok-aligned
\item Color clash (bleeding)
\end{itemize}

\end{frame}

%----------------------------------------------

\begin{frame}
\frametitle{Color clash bij MSX}

\begin{figure}
\includegraphics[width=0.5\linewidth]{images/bleeding.png}
\end{figure}

\end{frame}

%----------------------------------------------

\begin{frame}
\frametitle{Welkom in de wereld van Sprites!}

\begin{figure}
\includegraphics[width=0.5\linewidth]{images/sprdwh.png}
\end{figure}

\begin{itemize}
\item Movable Objects (MOBs)
\item 8 sprites bestaande uit 24*21 pixels
\item 63 bytes per sprite
\item X en Y co\"ordinaat
\item Great Giana Sisters
\end{itemize}

\end{frame}

%----------------------------------------------

\begin{frame}
\frametitle{Graphics Hacking}

\begin{figure}
\includegraphics[width=0.75\linewidth]{images/raster.png}
\end{figure}

\end{frame}

%----------------------------------------------

\begin{frame}
\frametitle{Graphics Hacking - Raster bars}

\begin{itemize}
\item Horizontaal: Double Hotline - R-Type+
\item Verticaal \& Ronddraaiend: Booze Design - Uncensored
\item Wiebelen: Censor \& Oxyron - Comalight X14
\end{itemize}

\end{frame}

%----------------------------------------------

\begin{frame}
\frametitle{Graphics Hacking - Effects/hacks/VIC bugs}

\begin{itemize}
\item Borderless graphics
\item Raster split
\item Parallax scrolling
\item Andere graphic modes
\item Plasma
\item Meer dan 8 sprites
\item \dots
\end{itemize}

\end{frame}

%----------------------------------------------

\begin{frame}{Intermezzo - Eigen intro maken}

\begin{itemize}
\item Download roms.zip van de github repository
\item Autostart de disk \verb:intro\_maker.d64:
\item Attach \verb:voorbeelden.d64:
\item Maak je eigen intro!
\end{itemize}

\end{frame}


%----------------------------------------------

\begin{frame}
\frametitle{Audio}

\begin{itemize}
\item Analoog of Digitaal
\end{itemize}

\begin{figure}
\includegraphics[width=0.6\linewidth]{images/waveforms.png}
\end{figure}

\end{frame}

%----------------------------------------------

\begin{frame}
\frametitle{Analoog Audio}

\begin{itemize}
\item Niet-elektronisch
\item Visualisatie met Oscilloscoop
\end{itemize}

\end{frame}

%----------------------------------------------

\begin{frame}
\frametitle{Digitaal Audio}

\begin{itemize}
\item Benadering van Analoog
\item Stapsgewijs
\item Duidelijk zichtbaar op oude PCs
\end{itemize}

\end{frame}

%----------------------------------------------

\begin{frame}
\frametitle{Digitaal Audio - Speaker}

\begin{itemize}
\item Squaker
\item Zeer ruw
\item Aan of uit
\item Frequentie
\end{itemize}

\end{frame}

%----------------------------------------------

\begin{frame}
\frametitle{Digitaal Audio - Speaker Voorbeelden}

\begin{itemize}
\item Loop
\item Single
\item Arpeggio
\end{itemize}

\end{frame}

%----------------------------------------------

\begin{frame}
\frametitle{Digitaal Audio - NES}

\begin{itemize}
\item 5 kanalen
\item 2 pulse, 1 noise, 1 triangle
\item 1 voor geluidseffecten
\end{itemize}

\end{frame}

%----------------------------------------------

\begin{frame}
\frametitle{Digitaal Audio - NES voorbeeld}

\begin{itemize}
\item The Legend of Zelda theme
\end{itemize}

\end{frame}

%----------------------------------------------

\begin{frame}
\frametitle{Digitaal Audio}

\begin{figure}
\includegraphics[width=0.9\linewidth]{images/rc3nes.png}
\end{figure}

\end{frame}

%----------------------------------------------

\begin{frame}
\frametitle{C64: SID chip}

\begin{itemize}
\item 's werelds eerste audio synthesizer on chip
\item Analoog \emph{en} Digitaal
\item Verschillende onderscheidingen
\item 6581 en 8580
\end{itemize}

\end{frame}

%----------------------------------------------

\begin{frame}
\frametitle{Ter Vergelijking}

\begin{tabular}{|l|l|l|l|}
\hline Model & Geluid & Type & Voices \\ \hline
PET & CPU PIO & 1-bit beeper & 0 \\
VIC-20 & VIC I & 8-bit digital & 4 \\
C64 & SID & A/D 8-bit synth. & 3 \\ \hline
Apple ][ & Speaker & 1-bit beeper & 0 \\
IBM PC & Speaker & 1-bit beeper & 0 \\
ZX Spectrum & AY-3-8912 & 8-bit digitaal & 4 \\ \hline
VCS & TIA & 5-bit digitaal & 2 \\
Vectrex & AY-3-8192 & 8-bit digitaal & 4 \\
NES & RP2A03 & 8-bit digitaal & 5 \\ \hline
\end{tabular}

\end{frame}

%----------------------------------------------

\begin{frame}
\frametitle{ADSR}

\begin{figure}
\includegraphics[width=0.7\linewidth]{images/adsr.png}
\end{figure}

\begin{tabular}{l|l}
Attack & Toetsaanslag \\
Decay & Amplitude zwakt af \\
Sustain & Amplitude constant \\
Release & Geluid `dooft' uit \\
\end{tabular}

\end{frame}

%----------------------------------------------

\begin{frame}
\frametitle{Waveforms}

\begin{figure}
\includegraphics[width=0.75\linewidth]{images/waveforms.png}
\end{figure}

\end{frame}

%----------------------------------------------

\begin{frame}
\frametitle{SID - Technieken}

\begin{itemize}
\item Pulse Width Modulation (PWM)
\item Ring modulation (ring mod)
\end{itemize}

\end{frame}

%----------------------------------------------

\begin{frame}
\frametitle{SID - Voorbeelden}

\begin{itemize}
\item Driller (PWM)
\item Commando
\item Zoids (ring mod)
\end{itemize}

\end{frame}

%----------------------------------------------

\begin{frame}
\frametitle{Intermezzo}

\begin{block}{Muziek}
Raad welk muziekje bij welke computer/console hoort!
\end{block}

\begin{itemize}
\item IBM PC
\item NES
\item C64
\end{itemize}

\end{frame}

%----------------------------------------------

\begin{frame}
\frametitle{Intermezzo - Voorbeelden}

\begin{itemize}
\item Grand Prix Circuit (PC, C64)
\item Castlevania (PC, NES, C64)
\end{itemize}

\end{frame}

%----------------------------------------------

\begin{frame}
\frametitle{Intermezzo}

\begin{itemize}
\item IBM PC
\item NES
\item C64
\end{itemize}

\end{frame}

%----------------------------------------------

\begin{frame}
\frametitle{SID hacking - Arkanoid}

\begin{itemize}
\item Click/pop-bug in 6581
\item 4-bit PCM digitaal
\end{itemize}

\begin{figure}
\includegraphics[width=0.9\linewidth]{images/farts.png}
\end{figure}

\end{frame}

%----------------------------------------------

\begin{frame}
\frametitle{SID hacking - Digitized sound}

\begin{itemize}
%\item mahoney, lft, lman, thcm
\item Vortex, Hi Fi Sky
\item Uniek: klinkt totaal niet als C64 muziek
\end{itemize} 

\end{frame}


%----------------------------------------------

\begin{frame}
\frametitle{Programmeerwedstrijden}

\begin{itemize}
\item X (in Nederland!)
\item Revision (grootste ter wereld!)
\item Nordlicht
\item Under Construction
\item \dots
\end{itemize}

\end{frame}

%----------------------------------------------

\begin{frame}
\frametitle{X}

\begin{figure}
\includegraphics[width=0.8\linewidth]{images/x.jpg}
\end{figure}

\end{frame}

%----------------------------------------------

\begin{frame}[noframenumbering]
\frametitle{X}

\begin{itemize}
\item sinds 1995
\item Vanaf 2004 \'e\'en keer per 2 jaar
\item Vorig jaar ong. 400 bezoekers
\end{itemize}

\end{frame}

%----------------------------------------------

\begin{frame}
\frametitle{Revision}

\begin{figure}
\includegraphics[width=0.8\linewidth]{images/rev2017.jpg}
\end{figure}

\end{frame}

%----------------------------------------------

\begin{frame}[noframenumbering]
\frametitle{Revision}

\begin{itemize}
\item Je krijgt er 2 colloqiumpunten voor!!
\item Sinds 2011 (daarvoor: Breakpoint)
\item Ong. 700 bezoekers
\item Grote bedrijven bieden banen aan!
\item Seminars
\item 30+ wedstrijden (demo, gfx, fotografie, \dots)
\end{itemize}
\end{frame}

%----------------------------------------------

\begin{frame}
\frametitle{Opdracht}

\begin{itemize}
\item Probeer de gegeven voorbeelden
\item Speel met de voorbeelden en probeer dingen uit!
\end{itemize}

\end{frame}

%----------------------------------------------

\begin{frame}
\frametitle{Wil je een C64?}

\begin{itemize}
\item Dat kan!
\item Duur op ebay en marktplaats
\item Goedkoop bij rommelmarkt
\item Toetsenbord
\item HCC
\end{itemize}

\end{frame}

%----------------------------------------------

\begin{frame}
\frametitle{HCC}

\begin{itemize}
\item Nederlandse Hobby Computer Club
\item Sinds 27 april 1977
\item Diverse interessegroepen
\item Elke twee maanden in Maarssen
\item Retro (8/16 bit) computers welkom
\end{itemize}

\end{frame}

%------------------------------------------------

\begin{frame}
\frametitle{Vervolg}

\begin{itemize}
\item Demoscene
\item Oldskool computers
\item Programmeerwedstrijden
\item Cracking 'n Hacking
\end{itemize}

\end{frame}

%------------------------------------------------

\begin{frame}
\frametitle{HVSC}

\begin{figure}
\includegraphics[width=0.6\linewidth]{images/hvsc.jpg}
\end{figure}

\begin{itemize}
\item 49542 SIDs
\item Ongeveer 1500 artiesten
\item Ongeveer 300MB
\item Dus ongeveer 6KB per SID!
\item Meer dan 10000 uur muziek!
\end{itemize}

\end{frame}

%------------------------------------------------

\begin{frame}
\frametitle{Pouet}

\begin{figure}
\includegraphics[width=0.6\linewidth]{images/pouet.jpg}
\end{figure}

\begin{itemize}
\item 92 platforms
\item 36 categorie\"en
\item Ongeveer 70,000 prods!
\end{itemize}

\end{frame}

%------------------------------------------------

\begin{frame}
\frametitle{Conclusies}

\begin{itemize}
\item PC Geschiedenis
\item Hacken op C64
\item Programmeren tot het uiterste
\item Programmeerwedstrijden
\end{itemize}

\end{frame}

%------------------------------------------------

%------------------------------------------------

\begin{frame}[noframenumbering]
\frametitle{Bronvermeldingen}

\begin{multicols}{2}
\begin{itemize}
\item archive.org
\item bolo.ch
\item c64-wiki.com
\item codebase64.org
%http://www.commodore.ca/manuals/c64_programmers_reference/c64-programmers_reference.htm
\item commodore.ca
\item commodore.hcc.nl
\item csdb.dk
\item dcs.warwick.ac.uk
\item demoparty.net
\item derbian.webs.com
\item devil.iki.fi
\item dmshas.de
\item famitracker.com
\item freeinfosociety.com
\item granishmusicproduction. com
\item history-computer.com
\item id.scene.org
\item ieee.org
\end{itemize}
\end{multicols}

\end{frame}

%------------------------------------------------

\begin{frame}[noframenumbering]
\frametitle{Bronvermeldingen}

\begin{multicols}{2}
\begin{itemize}
\item musictechstudent.co.uk
\item oldcomputers.net
\item pagetable.com
\item pouet.net
\item revision-party.net
\item scs-trc.net
\item sta.c64.org
\item vice.sourceforge.net
\item wikipedia.org
\end{itemize}
\end{multicols}

\end{frame}
%------------------------------------------------

\begin{frame}[noframenumbering]
\frametitle{Met dank aan}

\begin{itemize}
\item Lezingencommissie
\item VIA
\item CCC
\item HCC, lemon64, csdb, VICE
\item Hobbyisten en C64 fans
\item Scene
\item \'Ecole Polytechnique F\'ed\'erale de Lausanne
% trs80 image:
\item Mus\'ee Bolo
\end{itemize}

\end{frame}

%------------------------------------------------

\begin{frame}[noframenumbering]
\frametitle{Met dank aan - HCC en Sceners}

\begin{itemize}
\item Duncan
\item fgenesis
\item fieserWolf
\item Ron van Schaik
\item Wilfred Bos
\end{itemize}

\end{frame}

%------------------------------------------------

\begin{frame}[noframenumbering]
\frametitle{Met dank aan - Voorproef en Feedback}

\begin{itemize}
\item Bert
\item Boaz
\item Brian
\item Dennis
\item Jelle
\item Kees
\item Otto
\item Sander
\item Teun
\item Tjaard
\end{itemize}

\end{frame}

%------------------------------------------------

\begin{frame}[noframenumbering]
\frametitle{Met dank aan - gfx}

\begin{multicols}{2}
\begin{itemize}
% breadbin comic
\item Bart van Tieghem
% vectrex
\item Evan-Amos
\item Bill Bertram
% CRT raster scan
\item Ian Harvey
\item Jozef Galanda
% bleeding MSX
\item Marco L
% C1541 drive
\item Nathan Beach
% waveforms
\item Omegatron
% trs80 image author:
\item Rama
\item Ruben de Rijcke
% C64 back size
\item Sammler
% great giana sisters - sprite vergroesserung
\item Werner
\end{itemize}
\end{multicols}

\end{frame}

%------------------------------------------------

\begin{frame}[noframenumbering]
\frametitle{Met dank aan de Sceners}

\begin{multicols}{3}
\begin{itemize}
\item Abyss Connection
\item Abnormal
\item Albion Crew
\item Alpha Flight
\item Algotech
\item Arkanix Labs
\item Artline Designs
\item Arise
\item Artstate
\item Arsenic
\item Atlantic
\item Atlantis
\item Atw
\item Bauknecht
\item Black Sun
\item Bluez Muz
\item Bonzai
\item Booze Design
\item Byterapers
\item Beyond Force
\end{itemize}
\end{multicols}

\end{frame}

%------------------------------------------------

\begin{frame}[noframenumbering]
\frametitle{Met dank aan de Sceners}

\begin{multicols}{3}
\begin{itemize}
\item Camelot
\item Cascade
\item Censor
\item Chorus
\item Chrome
\item Cosine
\item Covert Bitops
\item Creators
\item Crescent
\item Crest
\item Darklite
\item Delysid
\item Dekadence
\item Desire
\item Dmagic
\item Dual Crew
\item Elysium
\item Extend
\item Excess
\item Exon
\item Fairlight
\item Fatzone
\item Focus
\item Fossil
\end{itemize}
\end{multicols}

\end{frame}

%------------------------------------------------

\begin{frame}[noframenumbering]
\frametitle{Met dank aan de Sceners}

\begin{multicols}{3}
\begin{itemize}
\item Genesis Project
\item Glance
\item Hack 'n 'trade
\item Hitman
\item Hoaxers
\item Holger
\item Hokuto Force
\item Horizon
\item House Designs
\item HVSC Crew
\item Judas
\item Laxity
\item Lepsi De
\item Level 64
\item LFT
\item Mahoney
\item Maniacs of Noise
\item Mayday
\item MDG
\end{itemize}
\end{multicols}

\end{frame}

%------------------------------------------------

\begin{frame}[noframenumbering]
\frametitle{Met dank aan de Sceners}

\begin{multicols}{3}
\begin{itemize}
\item Metalvotze
\item Miracles
\item Multistyle Labs
\item The New Dimension
\item Nah-Kolor
\item Noice
\item No Name
\item Nostalgia
\item Nuance
\item Offence
\item Onslaught
\item Oxyron
\item Padua
\item Panda Design
\item Plush
\item Poo-brain
\item Prosonix
\item Radwar
\item Razor
\item Reflex
\item Resource
\item Samar
\item Scene sat
\item Shape
\item Sidrip
\end{itemize}
\end{multicols}

\end{frame}

%------------------------------------------------

\begin{frame}[noframenumbering]
\frametitle{Met dank aan de Sceners}

\begin{multicols}{3}
\begin{itemize}
\item Singular
\item Slay Radio
\item Smash Designs
\item Starion
\item Style
\item SCS*TRC
\item TPUG
\item Triad
\item Tropyx
\item TRSI
\item Unicess
\item Up Rough
\item Vandalism News
\item Vibrants
\item VICE team
\item Vision
\item Wow
\item Wrath Designs
\item Xenon
\item You Lazy Bastards
\end{itemize}
\end{multicols}

\end{frame}


%------------------------------------------------

\begin{frame}[noframenumbering]
\frametitle{Bedankt voor jullie aandacht}
\Large{\centerline{Zijn er nog vragen?}}
\end{frame}

\end{document}
