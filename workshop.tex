%%%%%%%%%%%%%%%%%%%%%%%%%%%%%%
% LATEX-TEMPLATE PRESENTATIE
%-------------------------------------------------------------------------------
% Voor informatie over presenterenn, zie
% http://practicumav.nl/presenteren/presenteren.html
% Voor readme en meest recente versie van het template, zie
% https://gitlab-fnwi.uva.nl/informatica/LaTeX-template.git
% Gebaseerd op een template van: http://www.LaTeXTemplates.com
% Licentie: CC BY-NC-SA 3.0
% (http://creativecommons.org/licenses/by-nc-sa/3.0/)
%%%%%%%%%%%%%%%%%%%%%%%%%%%%%%

%-------------------------------------------------------------------------------
%	PACKAGES EN CONFIGURATIE
%-------------------------------------------------------------------------------

\documentclass[aspectratio=43]{uva-inf-presentation}
\usepackage[dutch]{babel}
\usepackage{qtree}
\usepackage{listings}

\title{Programmeren in Beperkte Omgevingen}
\course{C=}
\assignment{Just 4 Fun}
\assignmenttype{Hacking}
\authors{Folkert van Verseveld}
\uvanetids{11839341}
\tutor{Teun Mathijssen}
\mentor{}
\docent{Robert van Wijk}
\group{Assembly A2}

\begin{document}

\begin{titelframe}
% De eerste slide is de titelpagina
\titlepage

\end{titelframe}

\begin{frame}
% Geeft een inhoudsopgave voor de presentatie
\frametitle{Inhoudsopgave}
\tableofcontents
\end{frame}

%-------------------------------------------------------------------------------
%	PRESENTATIE SLIDES
%-------------------------------------------------------------------------------

% Secties zijn er om structuur in je presentatie aan te brengen
% en worden direct in de inhoudsopgave opgenomen
\section{Introductie}
% Maak een subsection voor een serie slides met een gedeeld thema
\section{Programmeeromgeving}
\section{Opdrachten}

\section{Conclusies}

%----------------------------------------------

\begin{frame}
\frametitle{Beperkte Omgevingen}

\begin{itemize}
\item Allereerste `computers'
\item Militaire toepassingen
\item Kamersgrote machines en torenhoge energierekeningen
\end{itemize}

\end{frame}

%----------------------------------------------

\begin{frame}
\frametitle{Allereerste `computers'}

\begin{itemize}
\item Rekentabellen (sin, cos)
\item Administratie (statistiek, belasting)
\end{itemize}

\end{frame}

%----------------------------------------------

\begin{frame}
\frametitle{Rekentabellen}

\begin{figure}
\includegraphics[width=0.75\linewidth]{images/rekentabellen.jpg}
\end{figure}

\end{frame}

%----------------------------------------------

\begin{frame}
\frametitle{Militaire toepassingen}

\begin{itemize}
\item Artillerie (boog van kogel)
\item Cryptografie (enigma)
\item OXO
\end{itemize}

\end{frame}

%----------------------------------------------

\begin{frame}
\frametitle{Bombe}

\begin{figure}
\includegraphics[width=0.75\linewidth]{images/bombe.jpg}
\end{figure}

\end{frame}

%----------------------------------------------

\begin{frame}
\frametitle{EDSAC}

\begin{figure}
\includegraphics[width=0.75\linewidth]{images/edsac.jpg}
\end{figure}

\end{frame}

%----------------------------------------------

\begin{frame}
\frametitle{EDSAC I/O}

\begin{figure}
\includegraphics[width=0.75\linewidth]{images/edsac_crts.jpg}
\end{figure}

\end{frame}

%----------------------------------------------

\begin{frame}
\frametitle{OXO}

\begin{figure}
\includegraphics[width=0.75\linewidth]{images/oxo.jpg}
\end{figure}

\end{frame}

%----------------------------------------------

% TODO add e.g. OXO, edsac met specs

%----------------------------------------------

\begin{frame}
\frametitle{Mainframes}

\begin{itemize}
\item Multi-user omgeving
\item Batch
\item Reserveren
\end{itemize}

\end{frame}

%----------------------------------------------

\begin{frame}
\frametitle{Mainframes}

\begin{itemize}
\item Onpraktisch
\item Heel duur: \$100,000+ / \$1,000,000+
\item Gebruikersonvriendelijk
\end{itemize}

\end{frame}

%----------------------------------------------

\begin{frame}
\frametitle{Mini-mainframes}

\begin{itemize}
\item Duur, maar niet onbetaalbaar: \$10000+
\item Acceptabele performance
\item `Gewone' klanten
\end{itemize}

\end{frame}

%----------------------------------------------

\begin{frame}
\frametitle{Moore's law}

\begin{itemize}
\item Rekenmachines steeds sneller en goedkoper
\item Hoe goedkoper, des te meer toepassingen
\end{itemize}

\end{frame}

%----------------------------------------------

\begin{frame}
\frametitle{Introductie van de Personal Computer}

\begin{itemize}
\item Mini mainframe
\item `Goedkoop'
\item `Gebruikersvriendelijk'
\end{itemize}

\end{frame}

%----------------------------------------------

\begin{frame}
\frametitle{Toepassingen van PCs}

\begin{itemize}
\item Kantoor gerelateerd (administratie, spreadsheets, \dots)
\item Consumentgericht
\item (Games)
\end{itemize}

\end{frame}

%----------------------------------------------

\begin{frame}
\frametitle{Ontmoet de Succesvolste PC ooit}

\begin{figure}
\includegraphics[width=0.75\linewidth]{images/desk.jpg}
\end{figure}

\end{frame}

%----------------------------------------------

\begin{frame}
\frametitle{De Enige Echte}

\begin{figure}
\includegraphics[width=0.6\linewidth]{images/c64.jpg}
\end{figure}

\end{frame}

%----------------------------------------------

\begin{frame}
\frametitle{Moderne Variant}

\begin{figure}
\includegraphics[width=0.5\linewidth]{images/boot.png}
\end{figure}

\end{frame}

%----------------------------------------------

\begin{frame}
\frametitle{De Specificaties}

\begin{itemize}
\item Ge\"introduceerd in 1982
\item 64 kilobytes
\item MOS 6510 met ~1 MHz kristalklok
\item 2 joystick poorten
\item cartridge port (games!), serial I/O, user port
\item 16 kleuren
\end{itemize}

\end{frame}

%----------------------------------------------

\begin{frame}
\frametitle{Ter Vergelijking}

\begin{tabular}{|l|l|l|l|}
\hline Jaar & Model & Prijs toen & Prijs nu \\
1977 & PET & \$795 & \$3150 \\
1980 & VIC-20 & \$299.95 & \$874 \\
1982 & C64 & \$595 & \$1480 \\ \hline
1977 & Apple ][ & \$1296 & \$5130 \\
1981 & IBM PC & \$1565 & \$4130 \\
%IBM PC & 16-640 & Beeper & Ja & \$1565 \\
%ZX Spec. & 16 & Beeper & Nee & \pound 175 \\ \hline
1982 & ZX Spec. & \pounds 175 & ??? \\ \hline
1983 & NES & \$179 & \$433 \\ \hline
\end{tabular}

%\begin{tabular}{|l|l|l|l|l|}
%\hline Model & RAM & Geluid & \$ toen & \$ nu \\
%%C64 & 64K & AD Synth. & \$595 - \$135 & \$595 \\
%C64 & 64K & AD Synth. & \$595 & \$595 \\
%Apple ][ & 8-16K & Beeper & \$1296 & \$1296 \\
%IBM PC & 16K & Beeper & \$1565 & \$1565 \\
%%IBM PC & 16-640 & Beeper & Ja & \$1565 \\
%%ZX Spec. & 16 & Beeper & Nee & \pound 175 \\ \hline
%ZX Spec. & 16K & Beeper & \pounds 175 & \pounds 175 pound \\ \hline
%NES & 2K & Z80? & \$179 & \$433 \\ \hline
%\end{tabular}

\end{frame}

%----------------------------------------------

\begin{frame}
\frametitle{En harddisk opslag?}

\begin{itemize}
\item Niet nodig
\item Te duur (+\$500/+\$2000)
\end{itemize}

\end{frame}

%----------------------------------------------

\begin{frame}
\frametitle{Ter Vergelijking}

\begin{tabular}{|l|l|l|l|l|l|}
\hline Model & CPU & MHz & Geluid & Voices \\
C64 & MOS 6510 & 1 & 8-bit SID & 3 \\
Apple ][ & MOS 6502 & 1 & 1-bit & 0 \\
IBM PC & i8086 & 4.04 & 1-bit & 0 \\
ZX Spec. & Z80 & 3.5 & 1-bit & 0 \\ \hline
NES & Ricoh 2A03 & 1 & 8-bit & 5 \\ \hline
\end{tabular}

\end{frame}

%----------------------------------------------

\begin{frame}
\frametitle{Ter Vergelijking}

\begin{tabular}{|l|l|l|l|}
\hline Model & Geluid & Type & Voices \\
C64 & SID & A/D 8-bit synth. & 3 \\
Apple ][ & Speaker & 1-bit beeper & 0 \\
IBM PC & Speaker & 1-bit beepr & 0 \\
%ZX Spec. & 16K & Beeper & 175 pound & 175 pound \\ \hline
NES & Z80? & 8-bit digital & 5 \\ \hline
\end{tabular}

\end{frame}

%----------------------------------------------

\begin{frame}
\frametitle{Ter Vergelijking}

\begin{tabular}{|l|l|l|}
\hline Model & Video & Kleuren \\
C64 & VIC II & 16 \\
Apple ][ & ??? & 1-4 \\
IBM PC & mono of CGA & 1-4 \\
ZX Spec. & geen (CPU) & 8 \\ \hline
NES & Ricoh 2C02 PPU & 52 \\ \hline
\end{tabular}

\end{frame}

%----------------------------------------------

\begin{frame}
\frametitle{Wat kan een C64?}

\begin{itemize}
\item Administratie (spreadsheets, word, \dots)
\item Scannen, printen
\item Games!
\item Internet (via 50Kbps modem)
\item Hardware upgrades
\end{itemize}

\end{frame}

%----------------------------------------------

\begin{frame}
\frametitle{Waarom werken met Vintage PCs?}

\begin{itemize}
\item Enorme uitdaging
\item Simpel ontwerp, simpel te begrijpen
\item Literatuur
\item Programmeerwedstrijden (ja, 30 jaar lang!)
\end{itemize}

\end{frame}

%----------------------------------------------

\begin{frame}
\frametitle{Programmeertalen}

\begin{itemize}
\item Microsoft BASIC
\item Machine code (assembly)
\end{itemize}

\end{frame}

%----------------------------------------------

\begin{frame}
\frametitle{Hoe Programmeerden we vroeger?}

\begin{itemize}
\item Geen internet
\item Literatuur (tijdschriften, boeken, \dots)
\item Overlevering (bijeenkomsten, wedstrijden, \dots)
\end{itemize}

\end{frame}

%----------------------------------------------

\begin{frame}[fragile]{Simpel Programma}

\begin{lstlisting}
10 PRINT "HELLO VIA"
20 GOTO 10
\end{lstlisting}

\end{frame}

%----------------------------------------------

\begin{frame}[fragile]{Simpel Programma}

\begin{lstlisting}
10 POKE 1024,8
20 POKE 1025,5
30 POKE 1026,25
\end{lstlisting}

\end{frame}

%----------------------------------------------

\begin{frame}
\frametitle{Audio}

\begin{itemize}
\item 3 kanalen (voices)
\item ADSR envelope
\end{itemize}

\end{frame}

%https://musictechstudent.co.uk/music-production-glossary/adsr/

%----------------------------------------------

\begin{frame}
\frametitle{ADSR}

\begin{figure}
\includegraphics[width=0.8\linewidth]{images/adsr.png}
\end{figure}

\end{frame}

%----------------------------------------------

\begin{frame}
\frametitle{Waveforms}

\begin{itemize}
\item Sawtooth
\item Square
\item Sinuso\"ide
\item Triangle
\item Noise
\end{itemize}

\end{frame}

%----------------------------------------------

\begin{frame}
\frametitle{Intermezzo}

\begin{block}{Muziek}
Raad welk muziekje bij welke computer hoort!
\end{block}

\end{frame}

%----------------------------------------------

\begin{frame}
\frametitle{Graphics}

\begin{figure}
\includegraphics[width=0.5\linewidth]{images/palette.png}
\end{figure}

\end{frame}

%----------------------------------------------

\begin{frame}
\frametitle{Graphics}

\begin{itemize}
\item 320x200 pixels
\item 16 kleuren palette
\item Maximaal 2 of 4 kleuren per 8x8 blok.
\end{itemize}

\end{frame}

%----------------------------------------------

\begin{frame}
\frametitle{Graphics}

\begin{figure}
\includegraphics[width=0.5\linewidth]{images/gfx.png}
\end{figure}

\end{frame}

%----------------------------------------------

\begin{frame}
\frametitle{Graphic modes}

\begin{itemize}
\item Standard Character (text) Mode
\item Multicolor Character (text) Mode
\item Standard Bitmap Mode
\item Multicolor Bitmap Mode
\item Extended Background Color Mode
\end{itemize}

\end{frame}

%----------------------------------------------

\begin{frame}
\frametitle{Multicolor Character Mode}

\begin{figure}
\includegraphics[width=0.5\linewidth]{images/gary.png}
\end{figure}

\end{frame}

%----------------------------------------------

\begin{frame}
\frametitle{Standard Bitmap}

\begin{figure}
\includegraphics[width=0.5\linewidth]{images/sbm_obey.png}
\end{figure}

\begin{center}
Source: Archmage
\end{center}

\end{frame}

%----------------------------------------------

\begin{frame}
\frametitle{Standard Bitmap}

\begin{figure}
\includegraphics[width=0.5\linewidth]{images/sbm_tuksu.png}
\end{figure}

\begin{center}
Source: Duce
\end{center}

\end{frame}

%----------------------------------------------

\begin{frame}
\frametitle{Programmeerwedstrijden}

\begin{itemize}
\item X (in Nederland!)
\item Revision (grootste ter wereld!)
\item Nordlicht
\item Under Construction
\item \dots
\end{itemize}

\end{frame}

%----------------------------------------------

\begin{frame}
\frametitle{X}

\begin{figure}
\includegraphics[width=0.8\linewidth]{images/x.jpg}
\end{figure}

\end{frame}

%----------------------------------------------

\begin{frame}
\frametitle{X}

\begin{itemize}
\item sinds 1995
\item Vanaf 2004 \'e\'en keer per 2 jaar
\item Vorig jaar ong. 400 bezoekers
\end{itemize}

\end{frame}

%----------------------------------------------

\begin{frame}
\frametitle{Revision}

\begin{figure}
\includegraphics[width=0.8\linewidth]{images/rev2017.jpg}
\end{figure}

\end{frame}

%----------------------------------------------

\begin{frame}
\frametitle{Revision}

\begin{itemize}
\item sinds 2011
\item ong. 700 bezoekers
\item grote bedrijven bieden banen aan!
\item seminars
\item 30+ wedstrijden (demo, gfx, fotografie, \dots)
\end{itemize}

\end{frame}

%----------------------------------------------

\begin{frame}
\frametitle{Opdracht}

\begin{itemize}
\item Vorm groepen van maximaal 5
\item Maak een demo!
\end{itemize}

\end{frame}

%----------------------------------------------
%\begin{frame}
%
%\begin{block}{Typen van binaire bomen}
%\begin{itemize}
%\item Geworteld
%\item Volledig
%\item Perfect
%\item Compleet
%%\item[Geworteld:] wortel en elke knoop niet meer dan 2 kinderen
%%\item[Volledig:] elke knoop heeft \'of 0 \'of 2 kinderen
%%\item[Perfect:] elke knoop heeft precies 2 kinderen of 2 bladeren
%%\item[Compleet:] elke level behalve laatste is kompleet gevuld
%\end{itemize}
%\end{block}
%
%\end{frame}

%------------------------------------------------

\begin{frame}
\frametitle{Conclusies}

%\begin{figure}
%\includegraphics[width=0.8\linewidth]{images/bigocheatsheet.png}
%\end{figure}
\end{frame}

%------------------------------------------------

\begin{frame}
\frametitle{Bedankt voor jullie aandacht}
\Large{\centerline{Zijn er nog vragen?}}
\end{frame}

\end{document}
