\begin{frame}
\frametitle[fragile]{Graphics}

\begin{figure}
	\begin{subfigure}[b]{0.4\textwidth}
		\includegraphics[width=\linewidth]{images/sbm_obey.png}
	\end{subfigure}
	\begin{subfigure}[b]{0.5\textwidth}
		\includegraphics[width=\linewidth]{images/nyan.png}
	\end{subfigure}
\end{figure}

\end{frame}

%----------------------------------------------

\begin{frame}
\frametitle{Graphics - Analoog}

\begin{itemize}
\item Vectrex
\item Geen pixels, maar SVG
\item Monochroom: helderheid
\end{itemize}

\end{frame}

%----------------------------------------------

\begin{frame}
\frametitle{Ter Vergelijking}

\begin{tabular}{|l|l|l|}
\hline Model & Video & Kleuren \\
PET & geen (CPU) & 1 \\
VIC-20 & VIC & 16 \\
C64 & VIC II & 16 \\
Apple ][ & geen (CPU, TTL) & 1 of 4 \\
IBM PC & mono of CGA & 1 of 4 \\
ZX Spec. & geen (CPU) & 8 \\ \hline
VCS & TIA & verschilt \\
Vectrex & VIA & geen: helderheid \\
NES & Ricoh 2C02 PPU & 52 \\ \hline
\end{tabular}

\end{frame}

%----------------------------------------------

\begin{frame}
\frametitle{Graphics - C64}

\begin{figure}
\includegraphics[width=0.5\linewidth]{images/palette.png}
\end{figure}

\end{frame}

%----------------------------------------------

\begin{frame}
\frametitle{Graphics Memory}

\begin{itemize}
\item 320x200 = 64,000
\item Maar 65,536 - 64,000 = 1,536 bytes over met bitmap
\item Hoe lossen we dit op?
\end{itemize}

\end{frame}

%----------------------------------------------

\begin{frame}
\frametitle{Graphics Memory}

\begin{itemize}
\item Gebruik 8x8 blokken
\item Tekst modus
\item Maar 40 * 25 = 1,000 bytes nodig
\item Met 16 colors maar 1,000 / 2 = 500 bytes meer
\item 1,500 bytes
\end{itemize}

\end{frame}

%----------------------------------------------

\begin{frame}
\frametitle{PETSCII - Tekst Mode}

\begin{figure}
\includegraphics[width=0.5\linewidth]{images/gary.png}
\end{figure}

\end{frame}

%----------------------------------------------

\begin{frame}
\frametitle{Standard Bitmap}

\begin{figure}
\includegraphics[width=0.5\linewidth]{images/sbm_tuksu.png}
\end{figure}

\begin{center}
Source: Duce
\end{center}

\end{frame}

%----------------------------------------------

\begin{frame}[fragile]{Simpel Graphics Programma - BASIC}

\begin{lstlisting}
10 VIC=53248
20 C=PEEK(VIC+32)AND15
30 PRINT"KLEURCODE=",C
40 C=(C+3)AND15
50 POKEVIC+32,C
60 PRINT"KLEURCODE=",C
\end{lstlisting}

\end{frame}

%----------------------------------------------

\begin{frame}[fragile]{Simpel Graphics Programma - BASIC}

\begin{lstlisting}
10 VIC=53248
20 FORC=0TO15
30 POKEVIC+32,C
40 FORT=1TO128:NEXT
50 NEXTC
\end{lstlisting}

\end{frame}

%----------------------------------------------

\begin{frame}
\frametitle{Hoe plaatsen we individuele pixels?}

\begin{itemize}
\item Onhandig: Grid/blok-aligned
\item Color clash (bleeding)
\end{itemize}

\end{frame}

%----------------------------------------------

\begin{frame}
\frametitle{Color clash bij MSX}

\begin{figure}
\includegraphics[width=0.5\linewidth]{images/bleeding.png}
\end{figure}

\end{frame}

%----------------------------------------------

\begin{frame}
\frametitle{Welkom in de wereld van Sprites!}

\begin{figure}
\includegraphics[width=0.5\linewidth]{images/sprdwh.png}
\end{figure}

\begin{itemize}
\item Movable Objects (MOBs)
\item 8 sprites bestaande uit 24*21 pixels
\item 63 bytes per sprite
\item X en Y co\"ordinaat
\item Great Giana Sisters
\end{itemize}

\end{frame}

%----------------------------------------------

\begin{frame}
\frametitle{Graphics Hacking}

\begin{figure}
\includegraphics[width=0.75\linewidth]{images/raster.png}
\end{figure}

\end{frame}

%----------------------------------------------

\begin{frame}
\frametitle{Graphics Hacking - Raster bars}

\begin{itemize}
\item Horizontaal: Double Hotline - R-Type+
\item Verticaal \& Ronddraaiend: Booze Design - Uncensored
\item Wiebelen: Censor \& Oxyron - Comalight X14
\end{itemize}

\end{frame}

%----------------------------------------------

\begin{frame}
\frametitle{Graphics Hacking - Effects/hacks/VIC bugs}

\begin{itemize}
\item Borderless graphics
\item Raster split
\item Parallax scrolling
\item Andere graphic modes
\item Plasma
\item Meer dan 8 sprites
\item \dots
\end{itemize}

\end{frame}

%----------------------------------------------

\begin{frame}{Intermezzo}

\begin{itemize}
\item RUN en speel met de programma's op roms/workshop/slides.d64
\item Ga verder met de opgaven
\item Of: schrijf een BASIC programma met wat grafische effecten
\end{itemize}

\end{frame}
