\documentclass{article}
\usepackage[dutch]{babel}
\usepackage{amsmath}
\usepackage{hyperref}
\usepackage{listings}
\usepackage{graphicx}
\usepackage{eurosym}

\title{Opgaven voor workshop Commodore 64}
\author{Folkert van Verseveld}

\newcounter{problem}
\newcounter{solution}

\newcommand\problem{%
  \stepcounter{problem}%
  \textbf{\theproblem.}~%
  \setcounter{solution}{0}%
}
\parindent 0in
\parskip 1em

\begin{document}
\maketitle

\section{Inleiding}

Deze opgavenbundel bevat een verzameling aan opgaven van langzaam toenemende moeilijkheidsgraad.
U hoeft niet de opgaven van begin tot eind door te werken, maar u kunt gewoon ergens beginnen en kijken hoever u komt.
Een deel van de opgaven zullen we ook tijdens de workshop klassikaal doornemen en variaties van laten zien.

\emph{Let op dat het document hyperlinks bevat waar u op kan klikken.
Als er geen links te zien zijn met een blauw of rood vierkant dan kunt u ook de pdf in uw browser openen.}

\subsection{VICE opzetten}
Voor deze workshop gebruiken we de Versatile Commodore Emulator (VICE).
Deze kunt u \href{http://vice-emu.sourceforge.net/}{hier} vinden.
Hieronder volgen kant-en-klare recepten voor verschillende besturingssytemen.

\subsubsection{Linux}
Het opzetten van de emulator verschilt per Linux distro.
Als u een distro hebt die afgeleid is van Debian (bijv. Ubuntu of Xubuntu) dan kunt u alles opzetten door het script \verb:vice.sh: als superuser te draaien:
\verb:sudo sh vice.sh:.
Als dit niet werkt of u heeft een distro die een andere package manager heeft dan \verb:apt:, dan kunt u in het document de variabelen aanpassen en script (nogmaals) draaien.

Als dit ook niet werkt maar u heeft Wine dan kunt u het recept van Windows ook volgen.
Voor de masochisten is er natuurlijk ook altijd de \href{http://sourceforge.net/projects/vice-emu/files/releases/vice-3.1.tar.gz/download}{source code}.

\subsubsection{Windows}
Download de \href{http://sourceforge.net/projects/vice-emu/files/releases/binaries/windows/WinVICE-3.1-x64.7z/download}{windows versie}.
Andere versies kunt u \href{http://vice-emu.sourceforge.net/windows.html}{hier} vinden.
U heeft \href{http://www.7-zip.org/}{7Zip} nodig om de emulator uit te pakken.
Pak vervolgens de emulator uit en start \verb:x64sc.exe: op.

\subsubsection{Mac}
Download de \href{http://sourceforge.net/projects/vice-emu/files/releases/binaries/macosx/vice-macosx-cocoa-x86_64-10.12-3.1.dmg/download}{mac versie}.
Om de emulator te gebruiken moet de .dmg uitgepakt worden.
Klik met de rechtermuisknop (of met CTRL+linkermuisknop) op x64sc en dan op \verb:Open:.

\subsection{Tips en Trucs}

Aarzel vooral niet om dingen te proberen.
Sommige opgaven zijn juist gemaakt met het idee dat u dingen moet proberen.

Als de emulator lijkt te zijn vastgelopen dan kunt u via \verb:File->Reset->Soft: de computer herstarten.
Als dit niet werkt kunt u via \verb:File->Reset->Hard: het forceren en als dat ook niet werkt de emulator sluiten en opnieuw openen.
\emph{Let wel op dat uw BASIC programma en alles wat niet op een floppydisk of tape is opgeslagen verloren gaat als u de emulator reset!
De floppydisk en/of tape moet u ook opnieuw aansluiten om uw programma weer van te laden!}

De emulator kan ook versneld worden met \verb:ALT+W: om zo de laadtijd van programma's te verkorten.
De originele laadtijd wordt namelijk ook nagebootst omdat veel software hiervan afhankelijk is!
Met \verb:ALT+W: kunt u de snelheid ook weer herstellen.

\section{Games}

Voor de C64 zijn ook talloze spelletjes te vinden die zowel via tape, floppy als cartridge verkocht werden.
Als u geen zin heeft om te programmeren in BASIC of gewoon even iets anders wil doen kunt u wat spellen laden.
Sommige spellen dienen uitsluitend met het toetsenbord of een joystick gespeeld te worden en dit wordt meestal in het spel aangegeven.
\emph{De standaardjoystick voor veel spellen moet in port 2!}

Tip: zie ook figuur \ref{fig:vicekbd} hoe het toetsenbord er uitziet en in de emulator werkt en zie appendix 6.1 voor het vinden van software, demo's en spellen.

\section{Opslaan/Laden van Programma's}

We gaan eerst kijken hoe we een floppydisk kunnen aanmaken en daarna kijken we naar het laden en opslaan van programma's, spellen en dergelijke.

Als eerste is het belangrijk dat we kijken of de emulator de echte trage laadtijden emuleert omdat veel programma's hiervan afhankelijk zijn.
Deze optie kunt u vinden in \verb:Settings->Drive settings->True drive emulation: (zie figuur \ref{fig:true_c1541}).
Als u deze optie niet kunt vinden dan is de drive emulation al goed ingesteld en kunt u dit negeren.

\begin{figure}
\centering
\includegraphics[width=0.75\linewidth]{images/true_c1541.png}
\caption{Aanzetten van echte floppy emulatie}
\label{fig:true_c1541}
\end{figure}

\subsection{Snel laden van programma's}

De emulator biedt gelukkig een manier om een floppy, disk of cartridge aan te sluiten en het direct te starten.
Dit kan met \verb:File->Smart-attach disk/tape: of met \verb:ALT+A:.
U kunt na het kiezen van een bestand dan op \verb:Open: of \verb:Autostart: klikken.
\verb:Autostart: zal het programma direct opstarten terwijl \verb:Open: alleen het opslagmedium aansluit.

\problem Autostart de demo \verb:roms/cracktros/hotline/htl-41.prg: om te kijken of de emulator goed werkt.

In de appendix kunt u verwijzingen vinden naar websites voor als u nog meer software wilt proberen (en ook in \verb:roms/:).
%Spellen kunt u vinden in \verb:roms/games: en demo's voor programmeerwedstrijden in \verb:roms/parties:.

\subsection{Cheatsheet opslaan/laden van programma's}

Hieronder volgt een beknopt overzicht van de BASIC commando's om programma's te laden en op te slaan:

\begin{tabular}{l|l}
Taak & Commando \\ \hline
Laad lijst van programma's & \verb:LOAD"$",8: \\
Laad programma genaamd \verb:TEST: & \verb:LOAD"TEST",8,1: \\
Laad eerste programma van disk & \verb:LOAD"*",8,1: \\
Sla programma op & \verb:SAVE"NAAM",8:
\end{tabular}

Hierbij is \verb:8: de drive letter.
Met \verb:RUN: kunt u een programma starten en met \verb:LIST: de lijst van programma's of de BASIC code van het geladen programma tonen.

In de volgende secties kijken we hoe we floppydisks kunnen maken en worden de \verb:SAVE: en \verb:LOAD: commando's uitgebreider besproken.

\subsection{Aanmaken en aansluiten van een floppydisk}

De makkelijkste manier om een floppydisk aan te maken is door deze in de werkbalk aan te maken met \verb:File->Create and attach an empty disk->Unit #8: en
vervolgens kiest u een naam voor het bestand en de \verb:Disk name: met het type \verb:d64: (zie figuur \ref{fig:create_d64}).

\begin{figure}
\centering
\includegraphics[width=0.75\linewidth]{images/create_d64.png}
\caption{Aanmaken van een floppydisk voor drive 8}
\label{fig:create_d64}
\end{figure}

Als het goed is kunt de floppydisk inladen met \verb:LOAD"$",8: waarna na een aantal seconden \verb:READY: zal verschijnen.
Nu kunt u met \verb:LIST: een directory listing tonen.
Het resultaat is vergelijkbaar met figuur \ref{fig:d64dummy}.

\begin{figure}
\centering
\includegraphics[width=0.75\linewidth]{images/d64dummy.png}
\caption{Lege floppydisk}
\label{fig:d64dummy}
\end{figure}

Als dit niet werkt, of u houdt van command-line tools, kunt u het onderstaande recept proberen.

\subsection{Handmatig aanmaken van floppydisk met c1541}

Met het floppy hulpprogramma c1541 kunt u een disk maken.
Als we bijvoorbeeld een disk genaamd `muziek' willen maken op linux kunnen we dit doen:

\begin{lstlisting}
c1541 -format diskname,id d64 muziek.d64
\end{lstlisting}

Of op windows:

\begin{lstlisting}
c1541.exe -format diskname,id d64 muziek.d64
\end{lstlisting}

Of op een mac:

% XXX make sure this is correct. It may be ./x64.app/.../c1541 but I'm not sure
\begin{lstlisting}
./c1541.app/Contents/MacOS/c1541 -format diskname,id d64 muziek.d64
\end{lstlisting}

Vervang `muziek' met de naam van de disk.

\subsection{Programma's opslaan}

Nadat u een BASIC programma heeft ingevoerd en de floppydisk aan drive 8 heeft aangesloten kunt u met \verb:SAVE"NAAM",8: een programma opslaan.
Als u het programma wilt overschrijven zult u dit in meerdere stappen moeten doen.
Als eerste moet het programma verwijderd worden:

\begin{lstlisting}
OPEN1,8,15,"I":CLOSE1
OPEN1,8,15,"S:NAAM":CLOSE1
\end{lstlisting}

Vervang \verb:NAAM: met het programma wat u wilt verwijderen.
Nadat het verwijderd is, kunt u het opnieuw bewaren met \verb:SAVE"NAAM",8:.

\subsection{Programma's laden}

Nadat u de floppydisk heeft aangesloten kunt u het programma van drive 8 laden met \verb:LOAD"$",8,1: waarna u \verb:LIST: kan intypen om de lijst van alle programma's te zien.
Een bestand laden kan met \verb:LOAD"NAAM",8,1:.

Als u het eerste programma van de disk wilt laden kunt u ook dit gebruiken: \verb:LOAD"*",8,1:.

Het is ook mogelijk om \verb:,1: weg te laten bij het \verb:LOAD: commando, maar sommige programma's zullen dan niet goed ingeladen worden.
Wat dit precies betekent valt buiten het domein van deze opgaven.


\section{Introductie BASIC}

De Commmodore 64 begrijpt naast machine code ook de high-level programmeertaal BASIC.
Als u de VICE emulator opstart zult u een vergelijkbaar venster zien zoals figuur \ref{fig:vice}.

\begin{figure}
\centering
\includegraphics[width=0.75\linewidth]{images/boot.png}
\caption{VICE emulator}
\label{fig:vice}
\end{figure}

Het opstartvenster vertelt dat het uitgerust is met BASIC versie 2 en dat het een 64KiB systeem is waarvan er 38911 bytes beschikbaar zijn voor BASIC.
Dit betekent ook dat een BASIC programma dus niet groter kan zijn dan 38.9KB!

Zoals u waarschijnlijk ook is opgevallen is dat er alleen hoofdletters in het opstartscherm gebruikt zijn.
Dit komt omdat toentertijd computerontwerpers het overbodig vonden om hoofdletters en kleine letters door mekaar te kunnen gebruiken!
Het is wel mogelijk om ze allebei tegelijk te gebruiken, maar dat laten we hier buiten beschouwing.

\subsection{Typen}

Met behulp van de spatie kunt u naast een spatie plaatsen ook een teken van het scherm `wegpoetsen'.

\problem Probeer verschillende stukken tekst van het scherm aan te passen of te verwijderen door er spaties overheen te plaatsen.

\problem Wat gebeurt er eigenlijk als u de cursor helemaal rechts plaats en vervolgens nog een keer pijltje naar rechts indrukt?
En wat gebeurt er als u de cursor helemaal links plaats en vervolgens nog een keer pijltje naar links indrukt?
En wat gebeurt er als u de cursor onderaan het scherm plaats en nogmaals pijltje omlaag indrukt?

\subsection{Toetsenbordconfiguratie}

Misschien is het u nog niet opgevallen, maar het toetsenbord van de C64 is best wel anders dan een QWERTY toetsenbord.
Als u \verb:SHIFT+2: en \verb:]: intypt zal er waarschijnlijk \verb:@]: op uw scherm staan.
Indien u \verb:"@: ziet, dan heeft de emulator de echte layout van het toetsenbord van de C64 gebruikt.
U de layout veranderen bij \verb:Settings->Keyboard settings->Keyboard mapping type:.
De echte C64 gebruikt \verb:Positional mapping: en de emulator gebruikt standaard \verb:Symbolic mapping:.
Zie ook figuur \ref{fig:changekbd} en \ref{fig:vicekbd}.

\begin{figure}
\centering
\includegraphics[width=0.75\linewidth]{images/changekbd.png}
\caption{Veranderen van keyboard layout}
\label{fig:changekbd}
\end{figure}

\begin{figure}
\centering
\includegraphics[width=\linewidth]{images/vicekbd.png}
\caption{C64 Keyboard cheatsheet}
\label{fig:vicekbd}
\end{figure}

Als u speciale tekens uit BASIC voorbeelden wilt intypen die niet op een QWERTY toetsenbord bestaan dan is het noodzakelijk om de \verb:Positional mapping: te gebruiken.
Veel opgaven kunnen ook getypt worden met de \verb:Symbolic mapping:.

\subsection{Printen}

Nu we weten hoe we tekst op het scherm plaatsen en weghalen kunnen we gaan kijken hoe we in BASIC moeten programmeren.
Laten we allereerst gaan kijken hoe we een ``Hello World'' programma kunnen maken.
Tekst kan geprint worden met het \verb:PRINT: commando.
Elk commando wordt uitgevoerd zodra u enter indrukt.
De tekst die geprint wordt moet tussen dubbele haken, namelijk \verb:":, staan.
Een voorbeeld hiervan is \verb:PRINT "HALLO":.

\problem Pas het voorbeeld aan zodat er niet \verb:HALLO: maar \verb:HELLO WORLD: geprint wordt.

\verb:PRINT: kan ook verschillende stukken tekst aan elkaar plakken.
Dit kan zowel met \verb:,: als met \verb:;: en dit kan meerdere keren herhaald worden.

\problem Wat is het verschil tussen \verb:PRINT"HALLO","OLLAH": en \verb:PRINT"HALLO";"OLLAH":?

\subsection{Rekenen}

Met BASIC kunnen we ook berekeningen uitvoeren.
De basisoperatoren $+$,$-$,$*$,$/$ en $\uparrow$ (machtsverheffing) kunnen we hiervoor gebruiken.
Merk op dat alle getallen als floating point worden opgeslagen.
Dus \verb:PRINT 2/3: zal geen $0$ printen, zoals sommige programmeurs zouden verwachten, maar $0.666666667$.

Als we nu wel integerdeling willen hebben, dat wil zeggen dat we alles achter de komma `negeren',
dan kunnen we dit doen met de functie \verb:INT():.
Bijvoorbeeld $\verb:PRINT:\ \verb:INT:(2/3)$ zal $0$ printen.

Zoals u waarschijnlijk al is opgevallen is dat de basisoperatoren dezelfde prioriteit hebben als in de wiskunde.
Dit is best luxe, want er waren rekenmachines en computers uit dezelfde tijd die de prioriteit van operatoren kompleet negeerden!
Met behulp van haakjes kunt u berekeningen in een andere volgorde uitvoeren.

\subsection{Variabelen}

Het is handig als we onze resultaten kunnen opslaan in variabelen zodat we resultaten niet steeds opnieuw moeten uitrekenen.
Daarnaast kunnen we een programma ook interactief maken.
BASIC heeft numerieke variabelen en nog andere typen variabelen (denk aan tekst en lijsten van getallen).
We kijken voor nu alleen naar de numerieke variabelen.

Vroeger werd de code van een programma eerst op papier geschreven en daarna ingetypt.
Variabelen werden daarom vaak kort en cryptisch opgeschreven, waardoor de Commodore 64 alleen korte variabelen ondersteunt.
Een variabele mag niet de naam hebben van een BASIC commando, dus \verb:PRINT: is bijvoorbeeld een ongeldige naam.
Daarnaast mag een variabele naam \emph{niet langer zijn dan twee karakters}.
Als een variabele langer is dan twee karakters worden alleen de eerste twee onthouden.
Bijvoorbeeld \verb:APPELS:, \verb:AP: en \verb:APPELFLAP: worden allemaal gezien als dezelfde variabele!

Stel, we willen gaan bijhouden hoeveel appels we hebben en we beginnen met 3 appels.
Dit doen we met \verb:A=3: en we kunnen dit uitprinten met bijvoorbeeld \verb:PRINT"APPELS",A:

We kunnen er een appel bijleggen door \verb:A=A+1: uit te voeren.
Daarna printen we weer hoeveel appels we hebben en zullen zien dat er nu 4 appels zijn.
Als u de printregel nog op het scherm heeft staan kunt u de cursor ernaar verplaatsen en weer op enter drukken zodat u de regel niet opnieuw hoeft in te typen.

Nu we de elementaire basis van BASIC onder de knie hebben kunnen we naar wat serieuzere programma's gaan kijken.

\section{Programmeren in BASIC}

Tot nu toe moesten we elke keer handmatig de BASIC commando's invullen.
Dit werkt voor kleine probeelsels, maar voor echte serieuze projecten is dit niet praktisch.
We gaan kijken hoe we ons eerste BASIC programma kunnen schrijven wat helemaal op zichzelf kan draaien zodra we hem RUNnen.

Elke regel in een BASIC programma wordt voorafgegaan met een uniek nummer.
Dit is te vergelijken met een regelnummer.
De BASIC interpreter gebruikt de regelnummers om te weten in welke volgorde het programma uitgevoerd moet worden.

Een simpel programma wat we nu kunnen maken is het volgende:

\begin{lstlisting}
10 PRINT"COMMODORE 64"
\end{lstlisting}

Als we nu RUN intypen en op enter drukken zal het programma uitgevoerd worden.
Gefeliciteerd! U heeft uw eerste echte BASIC programma gemaakt!

Nu is dit programma een beetje saai.
We willen dat het scherm helemaal vol komt te staan door oneindig keer regel 10 uit te voeren.
In BASIC kunnen we met behulp van een \verb:GOTO: naar een regelnummer springen.
Hiermee kunnen we een oneindige loop maken.
Voeg de volgende regel toe en RUN het programma.

\begin{lstlisting}
20 GOTO 10
\end{lstlisting}

Het programma kan afgebroken worden met ESCAPE (\verb:STOP: op de C=64).

\problem Voeg nu achter regel 10 een \verb:;: toe en druk op enter. Wat gebeurt er nu als we het programma RUNnen?

\problem Typ \verb:LIST: en daarna \verb:LIST10: in. Wat doet dit commando?

Elke keer als we een nieuw programma gaan schrijven willen we alle code van vorige programma's verwijderen.
Dit kunnen we doen door de emulator te resetten, maar dit is natuurlijk niet handig.
Een veel makkelijkere manier is door \verb:NEW: in te typen.
Wees hier wel voorzichtig mee, want er wordt geen waarschuwing gegeven!

\subsection{Peeks 'n Pokes}

RUN het programma \verb:BORDERKLEUR: van de slides disk (\verb:roms/workshop/slides.d64:).

\problem Wat doet het \verb:PEEK: commando op regel 20?

Omdat we maar 16 kleuren hebben negeren we de bovenste vier bits met \verb:AND15:.

\problem Wat doet het \verb:POKE: commando op regel 50?

\problem Wat bevat het magische adres \verb:VIC+32:?

\subsection{Invoer/Uitvoer}

Invoer kan opgevraagd worden met het \verb:INPUT: commando.
Bijvoorbeeld met het programma:

\begin{lstlisting}
10 INPUT A$
20 PRINT "UW INVOER: "; A$
\end{lstlisting}

Typ het programma in en RUN het programma.
Het programma zal u om gebruikersinvoer vragen en dit printen.

\problem Schrijf een BASIC programma wat een gebruikersnaam vraagt en dit print naar het scherm.

Nu is een goed moment om te kijken of u het programma naar een floppydisk kunt opslaan zodat u het programma later weer kunt laden.
Zie ook het cheatsheet voor het opslaan en laden van programma's in sectie 3.5 en 3.6.

\subsection{Programmeerhandleiding en Literatuur}

Dit is een klein voorproefje van wat er allemaal in BASIC mogelijk is.
Indien u meer programma's wilt schrijven kunt u hoofdstuk 3 en verder bekijken van de \verb:User's Guide:.
Als u meer uitdaging wilt of meer wilt weten over de details van de Commodore 64 kunt u naar de \verb:Programmer's Reference Guide: kijken.
Beide documenten zijn inbegrepen als u de release versie van de githubpagina heeft gedownload.

\section{Maak je eigen intro}
FieserwolF heeft een intro maker gemaakt waarmee u een eigen intro kunt maken!
De gebruikersinstructies worden ook door het programma zelf uitgelegd, maar hier is een kort stappenplan:

\begin{itemize}
	\item Download roms.zip als u dit nog niet gedaan heeft en pak deze uit.
	\item Autostart de disk \verb:intro_maker.d64: en wacht tot u het hoofdmenu ziet
	\item Met SPATIE kunt u dingen selecteren en met ESCAPE annuleren.
	\item Als u meer plaatjes wilt gebruiken kunt u \verb:voorbeelden.d64: aan drive 8 zetten.
\end{itemize}

Let op: als u uw creatie wilt opslaan dient u een nieuwe disk aan te sluiten en kunt u het opslaan via \verb:1 main->7 save intro:.

\section{Appendix}

\subsection{Commodore 64 programma's, spellen en demo's}

Er is heel veel software te vinden voor de Commodore 64.
Sterker nog, er wordt nog steeds veel software gemaakt!
De bekendste en volledigste website is \href{http://csdb.dk}{csdb}.

Een andere hele bekende, onder andere voor programmeerwedstrijden, is \href{https://www.pouet.net}{pouet}.
Hier is nog veel meer te vinden dan alleen Commodore 64 software.
Bijvoorbeeld 's werelds grootste en recentste programmeerwedstrijd van de C64 is X.
\href{https://www.pouet.net/party.php?which=50&when=2018}{Hier} is een overzicht van alle software die toen is uitgebracht.
En een paar voorgaande jaren: \href{https://www.pouet.net/party.php?which=50&when=2010}{2010}, \href{https://www.pouet.net/party.php?which=50&when=2012}{2012}, \href{https://www.pouet.net/party.php?which=50&when=2014}{2014}, \href{https://www.pouet.net/party.php?which=50&when=2016}{2016}

\subsection{easy6502}

Als u de BASIC opdrachten erg saai vond of benieuwd bent in programmeren in assembly dan kunt u \href{https://skilldrick.github.io/easy6502/}{deze introductie} eens proberen.

\subsection{KickAssembler}

Als u alles al afheeft of nog meer uitdaging zoekt kunt u kijken of u een programma in assembly kan schrijven met \href{http://theweb.dk/KickAssembler/Main.html#frontpage}{deze assembler}.
Er zijn voorbeelden inbegrepen bij de assembler en er is goede documentatie.

Voor assembly programmeren zijn deze bronnen ook onmisbaar:

\begin{tabular}{l|l|l}
Link & Domein & Toelichting \\
\href{http://codebase64.org}{codebase64} & D\'e Programmeerwiki & Programmeertips en -voorbeelden \\
\href{http://sta.c64.org/cbm64mem.html}{cbm64mem.html} & Memory map van C64 & Elk geheugenadres met toelichting \\
\href{https://www.c64-wiki.com}{c64-wiki} & Wikipedia voor C64 & Alles rondom Commodore \\
\end{tabular}

\subsection{Demoscene}

De workshop laat kleine aspecten zien van de demoscene die voornamelijk ontstaan is door computerclubs van de eerste Personal Computers.
Als u ge\"interesseerd bent geraakt in de demoscene of u wilt weten wat dit is zijn de volgende bronnen aan te raden om te kijken:

\begin{tabular}{l|l|l}
Link & Omschrijving & Toelichting \\
\href{https://www.youtube.com/watch?v=AdTANxS-LHg}{YT: Polderpioniers - J. Tel} & Eerste gamemuzikanten & Tweakers interview \\
\href{https://www.youtube.com/watch?v=5MexnBunH_g}{YT: The Art of Algorithms} & Geschiedenis Demoscene & Documentaire \\
\href{https://www.youtube.com/watch?v=O-1zEo7DD8w}{YT: Rev2017 Shader Finals} & Programmeerwedstrijd & Stream \\
\href{https://en.wikipedia.org/wiki/Demoscene}{Wikipedia: Demoscene} & Demoscene crash course & Wikipedia's blik \\
\href{https://id.scene.org/}{Scene portal} & Demoscene portal & Account voor demoscene sites \\
\href{https://www.youtube.com/watch?v=rFv7mHTf0nA}{Future Crew - Second Reality} & Winnaar Assembly 1993 & Een bekende jaren '90 demo \\
\end{tabular}

\end{document}
