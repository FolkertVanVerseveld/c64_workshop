\documentclass{article}

\usepackage[dutch]{babel}
\usepackage{amsmath}
\usepackage{hyperref}
\usepackage{listings}
\usepackage{graphicx}
\usepackage{eurosym}

\title{Opgaven voor workshop Commodore 64}
\author{Folkert van Verseveld}

\newcounter{problem}
\newcounter{solution}

\newcommand\problem{%
  \stepcounter{problem}%
  \textbf{\theproblem.}~%
  \setcounter{solution}{0}%
}
\parindent 0in
\parskip 1em

\begin{document}
\maketitle

\section{Inleiding}

Deze opgavenbundel bevat een verzameling aan opgaven van langzaam toenemende moeilijkheidsgraad.
De moeilijke opgaven zijn voorzien van een opmerking en deze kunt u overslaan.
In het antwoordenbundel staan de uitwerkingen van een aantal opgaven.
Een deel van de opgaven zullen we ook tijdens de workshop klassikaal doornemen en variaties van laten zien.

\subsection{VICE opzetten}
Voor deze workshop gebruiken we de Versatile Commodore Emulator (VICE).
Deze kunt u \href{http://vice-emu.sourceforge.net/}{hier} vinden.
Hieronder volgen kant-en-klare recepten voor verschillende besturingssytemen.

\subsubsection{Linux}
Het makkelijkst is om VICE met uw package manager te installeren. Bijvoorbeeld: \verb:sudo apt install vice: op Debian, Ubuntu en varianten.
Als dit niet werkt maar u heeft Wine dan kunt u het recept van Windows ook volgen.
Voor de masochisten is er natuurlijk ook altijd de \href{http://sourceforge.net/projects/vice-emu/files/releases/vice-3.1.tar.gz/download}{source code}.

\subsubsection{Windows}
Download de \href{http://sourceforge.net/projects/vice-emu/files/releases/binaries/windows/WinVICE-3.1-x64.7z/download}{windows versie}.
Andere versies kun je \href{http://vice-emu.sourceforge.net/windows.html}{hier} vinden.
U heeft \href{http://www.7-zip.org/}{7Zip} nodig om de emulator uit te pakken.
Pak vervolgens de emulator uit en start \verb:x64.exe: op.

\subsubsection{Mac}
Download de \href{http://sourceforge.net/projects/vice-emu/files/releases/binaries/macosx/vice-macosx-sdl-x86_64-10.12-3.1.dmg/download}{mac versie}.
Om de emulator te gebruiken moet de .dmg uitgepakt worden.
Let op dat de emulator vanuit een terminal gestart moet worden omdat het optiemenu ontbreekt.

Open de terminal en navigeer naar de map waar u de emulator heeft uitgepakt en start de emulator op. Bijvoorbeeld als u de emulator op uw bureaublad heeft gezet:
\begin{lstlisting}
cd /Users/gebruiker/Desktop/vice-emu/
./x64.app/Contents/MacOS/x64
\end{lstlisting}

Vervang \verb:gebruiker: met uw gebruikersnaam.

\subsection{Tips en Trucs}

Als u vastloopt met de opgaven kunt u ook het document met hints bekijken die u verder op weg helpen.
Aarzel vooral niet om dingen te proberen.
Sommige opgaven zijn juist gemaakt met het idee dat u dingen moet proberen.

Als de emulator lijkt te zijn vastgelopen dan kunt u via \verb:File->Reset->Soft: de computer herstarten.
Als dit niet werkt kunt u via \verb:File->Reset->Hard: het forceren en als dat ook niet werkt de emulator sluiten en opnieuw openen.

\section{Introductie BASIC}

De Commmodore 64 begrijpt naast machine code ook de high-level programmeertaal BASIC.
Als u de VICE emulator opstart zult u een vergelijkbaar venster zien zoals Figuur \ref{fig:vice}.

\begin{figure}
\centering
\includegraphics[width=0.5\linewidth]{images/boot.png}
\caption{VICE emulator}
\label{fig:vice}
\end{figure}

Het opstartvenster vertelt dat het uitgerust is met BASIC versie 2 en dat het een 64KiB systeem is waarvan er 38911 bytes beschikbaar zijn voor BASIC.
Dit betekent ook dat een BASIC programma dus niet groter kan zijn dan 38.9KB!

Zoals u waarschijnlijk ook is opgevallen is dat er alleen hoofdletters in het opstartscherm gebruikt zijn.
Dit komt omdat toentertijd computerontwerpers het overbodig vonden om hoofdletters en kleine letters door mekaar te kunnen gebruiken!
Het is wel mogelijk om ze allebei tegelijk te gebruiken, maar dat laten we hier buiten beschouwing.

\subsection{Typen}

\problem Probeer \verb:hallo: en \verb:HALLO: in te typen. Wat gebeurt er als u tekst typt terwijl u de shift toets ingedrukt houdt?

Met behulp van de spatie kunt u naast een spatie plaatsen ook een teken van het scherm `wegpoetsen'.

\problem Probeer verschillende stukken tekst van het scherm aan te passen of te verwijderen door er spaties overheen te plaatsen.

\problem Wat gebeurt er eigenlijk als u de cursor helemaal rechts plaats en vervolgens nog een keer pijltje naar rechts indrukt?
En wat gebeurt er als u de cursor helemaal links plaats en vervolgens nog een keer pijltje naar links indrukt?
En wat gebeurt er als u de cursor onderaan het scherm plaats en nogmaals pijltje omlaag indrukt?

\subsection{Printen}

Nu we weten hoe we tekst op het scherm plaatsen en weghalen kunnen we gaan kijken hoe we in BASIC moeten programmeren.
Laten we allereerst gaan kijken hoe we een ``Hello World'' programma kunnen maken.
Tekst kan geprint worden met het \verb:PRINT: commando.
Elk commando wordt uitgevoerd zodra u enter indrukt.
De tekst die geprint wordt moet tussen dubbele haken, namelijk \verb:":, staan.
Een voorbeeld hiervan is \verb:PRINT "HALLO":.

\problem Pas het voorbeeld aan zodat er niet \verb:HALLO: maar \verb:HELLO WORLD: geprint wordt.

\verb:PRINT: kan ook verschillende stukken tekst aan elkaar plakken.
Dit kan zowel met \verb:,: als met \verb:;: en dit kan meerdere keren herhaald worden.

\problem Wat is het verschil tussen \verb:PRINT"HALLO","OLLAH": en \verb:PRINT"HALLO";"OLLAH":?

\subsection{Rekenen}

Met BASIC kunnen we ook berekeningen uitvoeren.
De basisoperatoren $+$,$-$,$*$ en $/$ kunnen we hiervoor gebruiken.
Maar als we bijvoorbeeld \verb:12 + 12: intypen en op enter drukken, dan zien we geen resultaat.
Dit komt omdat we niet verteld hebben dat we het resultaat ook willen printen.

\problem Schrijf een BASIC commando wat $12 + 12$ uitrekent en het resultaat naar het scherm print.

\problem Schrijf een BASIC commando wat $2 - 3 * 4$ uitrekent en het resultaat naar het scherm print.

De enige operator die we nog niet met een voorbeeld hebben laten zien is de deeloperator $/$.
In BASIC worden alle getallen als floating point getallen opgeslagen, wat sommige programmeurs misschien niet zouden verwachten.
Bijvoorbeeld $\verb:PRINT:\ 2/3$ zal niet $0$, maar $0.666666667$ printen.

Als we nu wel integerdeling willen hebben, dat wil zeggen dat we alles achter de komma `negeren',
dan kunnen we dit doen met de functie \verb:INT():.
Bijvoorbeeld $\verb:PRINT:\ \verb:INT:(2/3)$ zal $0$ printen.

Zoals u waarschijnlijk al is opgevallen is dat de basisoperatoren dezelfde prioriteit hebben als in de wiskunde.
Met behulp van haakjes kunt u berekeningen in een andere volgorde uitvoeren.
%Als u een te ingewikkelde formule invult zal de foutmelding \verb:FORMULA TOO COMPLEX: geprint worden.

\problem Schrijf een BASIC commando wat $3*3*3*3$ uitrekent en het resultaat naar het scherm print.
Kunnen we dit naast herhaaldelijk vermenigvuldigen ook anders opschrijven?

\subsection{Variabelen}

Het is handig als we onze resultaten kunnen opslaan in variabelen zodat we resultaten niet steeds opnieuw moeten uitrekenen.
Daarnaast kunnen we een programma ook interactief maken.
BASIC heeft numerieke variabelen en nog andere typen variabelen (denk aan tekst en lijsten van getallen).
We kijken voor nu alleen naar de numerieke variabelen.

Vroeger werd de code van een programma eerst op papier geschreven en daarna ingetypt.
Variabelen werden daarom vaak kort en cryptisch opgeschreven, waardoor de Commodore 64 alleen korte variabelen ondersteunt.
Een variabele mag niet de naam hebben van een BASIC commando, dus \verb:PRINT: is bijvoorbeeld een ongeldige naam.
Daarnaast mag een variabele naam \emph{niet langer zijn dan twee karakters}.
Als een variabele langer is dan twee karakters worden alleen de eerste twee onthouden.
Bijvoorbeeld \verb:APPELS:, \verb:AP: en \verb:APPELFLAP: worden allemaal gezien als dezelfde variabele!

Stel, we willen gaan bijhouden hoeveel appels we hebben en we beginnen met 3 appels.
Dit doen we met \verb:A=3: en we kunnen dit uitprinten met bijvoorbeeld \verb:PRINT"APPELS",A:

We kunnen er een appel bijleggen door \verb:A=A+1: uit te voeren.
Daarna printen we weer hoeveel appels we hebben en zullen zien dat er nu 4 appels zijn.
Als u de printregel nog op het scherm heeft staan kunt u de cursor ernaar verplaatsen en weer op enter drukken zodat u de regel niet opnieuw hoeft in te typen.

Nu we de elementaire basis van BASIC onder de knie hebben kunnen we naar wat serieuzere programma's gaan kijken.

\section{Programmeren in BASIC}

Tot nu toe moesten we elke keer handmatig de BASIC commando's invullen.
Dit werkt voor kleine probeelsels, maar voor echte serieuze projecten is dit niet praktisch.
We gaan kijken hoe we ons eerste BASIC programma kunnen schrijven wat helemaal op zichzelf kan draaien zodra we hem RUNnen.

Elke regel in een BASIC programma wordt voorafgegaan met een nummer.
Dit is te vergelijken met een regelnummer.
De BASIC interpreter gebruikt de regelnummers om te weten in welke volgorde het programma uitgevoerd moet worden.

Een simpel programma wat we nu kunnen maken is het volgende:

\begin{lstlisting}
10 PRINT"COMMODORE 64"
\end{lstlisting}

Als we nu RUN intypen en op enter drukken zal het programma uitgevoerd worden.
Gefeliciteerd! U heeft uw eerste echte BASIC programma gemaakt!

Nu is dit programma een beetje saai.
We willen dat het scherm helemaal vol komt te staan door oneindig keer regel 10 uit te voeren.
In BASIC kunnen we met behulp van een \verb:GOTO: naar een regelnummer springen.
Hiermee kunnen we een oneindige loop maken.
Voeg de volgende regel toe en RUN het programma.

\begin{lstlisting}
20 GOTO 10
\end{lstlisting}

Het programma kan afgebroken worden met ESCAPE (\verb:STOP: op de C=64).

\problem Voeg nu achter regel 10 een \verb:;: toe en druk op enter. Wat gebeurt er nu als we het programma RUNnen?

\problem Typ \verb:LIST: en daarna \verb:LIST10: in. Wat doet dit commando?

Elke keer als we een nieuw programma gaan schrijven willen we alle code van vorige programma's verwijderen.
Dit kunnen we doen door de emulator te resetten, maar dit is natuurlijk niet handig.
Een veel makkelijkere manier is door \verb:NEW: in te typen.
Wees hier wel voorzichtig mee, want er wordt geen waarschuwing gegeven!

\problem Folkert heeft 6 appels gekocht van \euro 0,32 per stuk en 4 bananen van \euro 0,22 per stuk en zou de totale kosten van alle appels, alle bananen en het totaal willen uitprinten.
Schrijf een programma wat dit uitrekent en print. Vergeet niet \verb:NEW: in te typen voordat je begint met programmeren!

\end{document}
