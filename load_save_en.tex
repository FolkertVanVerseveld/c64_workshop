\section{Saving/Loading programs}

First, we are going to look how to create a floppydisk and see if we can load and save some programs, games etc.

It is important to check if the emulator is running in \verb:True drive emulation: mode, because many programs depend on this behavior.
You can check this at \verb:Settings->Drive settings->True drive emulation: (see figure \ref{fig:true_c1541}).
If you can't find this option it is probably setup up correctly and you can ignore this.

\begin{figure}
\centering
\includegraphics[width=0.75\linewidth]{images/true_c1541.png}
\caption{Enabling True drive emulation}
\label{fig:true_c1541}
\end{figure}

\subsection{Fast load programs}

The emulator enables you to autostart a program from floppy, disk or cartridge without doing everything manually.
You can autostart a program using \verb:File->Smart-attach disk/tape: or just \verb:ALT+A:.
After you selected the file you can \verb:Open: or \verb:Autostart: the program.
\verb:Autostart: will start the program immediately while \verb:Open: just connects/attaches the floppy.

\problem Autostart the demo \verb:roms/cracktros/hotline/htl-41.prg: to make sure the emulator is working properly.

In the appendix are multiple references to websites for more software you could try out as well (and also in \verb:roms/:).
%Spellen kunt u vinden in \verb:roms/games: en demo's voor programmeerwedstrijden in \verb:roms/parties:.

\subsection{Cheatsheet saving/loading programs}

The following is a simple table with BASIC commands for loading and saving programs:

\begin{tabular}{l|l}
Task & Command \\ \hline
Load program list & \verb:LOAD"$",8: \\
Load program named \verb:TEST: & \verb:LOAD"TEST",8,1: \\
Load first program from disk & \verb:LOAD"*",8,1: \\
Save program & \verb:SAVE"NAME",8:
\end{tabular}

\verb:8: is the drive letter.
With \verb:RUN: you can run a program and with \verb:LIST: you can print the program list or the BASIC code from a loaded BASIC program.

In the next sections we will look how to create floppydisk and how to \verb:SAVE: and \verb:LOAD: programs in more detail and also provide some tips and troubleshooting.

\subsection{Creating and attaching a floppydisk}

The easiest way to create a floppydisk is by navigating through the options from the menu bar \verb:File->Create and attach an empty disk->Unit #8: and
then type the program name in the \verb:Disk name: field and make sure the type is set \verb:d64: (see figure \ref{fig:create_d64}).

\begin{figure}
\centering
\includegraphics[width=0.75\linewidth]{images/create_d64.png}
\caption{Creating floppydisk for drive 8}
\label{fig:create_d64}
\end{figure}

Now you should be able to load the floppy using \verb:LOAD"$",8: and the C64 will print \verb:READY: after a few seconds.
Then you can \verb:LIST: the program list.
The result should look similar to figure \ref{fig:d64dummy}.

\begin{figure}
\centering
\includegraphics[width=0.75\linewidth]{images/d64dummy.png}
\caption{Empty floppydisk}
\label{fig:d64dummy}
\end{figure}

If this does not work, or you like using commmand-line tools, you can try out the following recipe.

\subsection{Manually create floppydisk using c1541}

You can create a floppy using a tool called c1541.
If we would like to create a disk named for example `music' we can do this on linux using:

\begin{lstlisting}
c1541 -format diskname,id d64 muziek.d64
\end{lstlisting}

Or on windows:

\begin{lstlisting}
c1541.exe -format diskname,id d64 muziek.d64
\end{lstlisting}

Or on a mac:

% XXX make sure this is correct. It may be ./x64.app/.../c1541 but I'm not sure
\begin{lstlisting}
./c1541.app/Contents/MacOS/c1541 -format diskname,id d64 muziek.d64
\end{lstlisting}

Change `music' with the name of your disk.

On windows and Linux you can attach the floppydisk to the drive even after you have booted the emulator using \verb:File->Attach disk image->Unit #8: or just \verb:ALT+8:.

\subsection{Save/Load programs on Mac}

If you want to save or load programs you have to attach the floppydisk \emph{before} starting the emulator. For example:

\begin{lstlisting}
./x64.app/Contents/MacOS/x64 path/to/your/floppy.d64
\end{lstlisting}

\subsection{Saving programs}

After typing in a BASIC program and attaching the floppydisk to drive 8 you can save the program using \verb:SAVE"NAME",8:.
If you want to overwrite a program you have to overwrite it in multiple steps.
First, we have to delete the program:

\begin{lstlisting}
OPEN1,8,15,"I":CLOSE1
OPEN1,8,15,"S:NAME":CLOSE1
\end{lstlisting}

Replace \verb:NAME: with the name of your program.
After deletion, you can resave the program using \verb:SAVE"NAME",8:.

\subsection{Loading programs}

After attaching the floppydisk you can load the program list from drive 8 using \verb:LOAD"$",8,1: and then \verb:LIST: the program list.
You can load a program using \verb:LOAD"NAME",8,1:.

If you want to load the first program from disk you can also use this: \verb:LOAD"*",8,1:.

It is possible to remove \verb:,1: from the \verb:LOAD: command, but this is discouraged as some programs may not load correctly.
Why this happens is out of scope of this workshop.
