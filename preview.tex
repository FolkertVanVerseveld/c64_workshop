%
% Naam: Folkert van Verseveld
% UvAnetID: 11839341
% Contact:
%   @: folkert.van.verseveld@gmail.com
%   @: folkert.vanverseveld@students.uva.nl
%

%%%%%%%%%%%%%%%%%%%%%%%%%%%%%%
% LATEX-TEMPLATE PRESENTATIE
%-------------------------------------------------------------------------------
% Voor informatie over presenterenn, zie
% http://practicumav.nl/presenteren/presenteren.html
% Voor readme en meest recente versie van het template, zie
% https://gitlab-fnwi.uva.nl/informatica/LaTeX-template.git
% Gebaseerd op een template van: http://www.LaTeXTemplates.com
% Licentie: CC BY-NC-SA 3.0
% (http://creativecommons.org/licenses/by-nc-sa/3.0/)
%%%%%%%%%%%%%%%%%%%%%%%%%%%%%%

%-------------------------------------------------------------------------------
%	PACKAGES EN CONFIGURATIE
%-------------------------------------------------------------------------------

\documentclass[aspectratio=43]{uva-inf-presentation}
\usepackage[dutch]{babel}
%\usepackage{qtree}
\usepackage{multicol}
\usepackage{amssymb}
\usepackage{listings}
\usepackage{subcaption}

\title{Programmeren in Beperkte Omgevingen}
% Easter egg: Commodore Business Machines logo
\course{C=}
\assignment{Just 4 Fun}
\assignmenttype{Hacking}
\authors{Folkert van Verseveld}
\uvanetids{11839341}
% XXX Tutor, mentor en docent niet echt van toepassing denk ik?
\tutor{Teun Mathijssen}
\mentor{}
\docent{Robert van Wijk}
\group{Assembly A2}

\begin{document}

\begin{titelframe}
\titlepage

\end{titelframe}

\begin{frame}
\frametitle{Inhoudsopgave}
\tableofcontents
\end{frame}

%-------------------------------------------------------------------------------
%	PRESENTATIE SLIDES
%-------------------------------------------------------------------------------

% Secties zijn er om structuur in je presentatie aan te brengen
% en worden direct in de inhoudsopgave opgenomen
\section{Introductie}
% Maak een subsection voor een serie slides met een gedeeld thema
\section{Programmeeromgeving}
\section{Programmeren in BASIC}
\section{Muziek}
\section{Graphics}
\section{Intermezzo's en Opdrachten}

\section{Conclusies}

%----------------------------------------------

\begin{frame}
\frametitle{Introductie van de Personal Computer}

\begin{itemize}
\item Mini mainframe
\item `Goedkoop'
\item `Gebruikersvriendelijk'
\item Eerste echte PCs rond 1977-1982
\end{itemize}

\end{frame}

%----------------------------------------------

\begin{frame}
\frametitle{Eerste Personal Computers}

\begin{figure}
\includegraphics[width=0.6\linewidth]{images/appleii.jpg}
\end{figure}

\end{frame}

%----------------------------------------------

\begin{frame}
\frametitle{Eerste Personal Computers}

\begin{figure}
\includegraphics[width=0.75\linewidth]{images/trs80.jpg}
\end{figure}

\end{frame}

%----------------------------------------------

\begin{frame}
\frametitle{Eerste Personal Computers}

\begin{figure}
\includegraphics[width=0.6\linewidth]{images/pet.jpg}
\end{figure}

\end{frame}

%----------------------------------------------

\begin{frame}
\frametitle{Eerste Personal Computers}

\begin{figure}
\includegraphics[width=0.6\linewidth]{images/vic20.jpg}
\end{figure}

\end{frame}

%----------------------------------------------

\begin{frame}
\frametitle{Eerste Personal Computers}

\begin{figure}
\includegraphics[width=0.6\linewidth]{images/ibm5150.jpg}
\end{figure}

\end{frame}

%----------------------------------------------

\begin{frame}
\frametitle{Eerste Personal Computers}

\begin{figure}
\includegraphics[width=0.6\linewidth]{images/zxspectrum.jpg}
\end{figure}

\end{frame}

%----------------------------------------------

\begin{frame}
\frametitle{Eerste Game Consoles}

\begin{figure}
\includegraphics[width=0.75\linewidth]{images/vcs.jpg}
\end{figure}

\end{frame}

%----------------------------------------------

\begin{frame}
\frametitle{Eerste Game Consoles}

\begin{figure}
\includegraphics[width=0.75\linewidth]{images/nes.jpg}
\end{figure}

\end{frame}

%----------------------------------------------

\begin{frame}
\frametitle{Toepassingen van PCs}

\begin{itemize}
\item Kantoor gerelateerd (administratie, spreadsheets, \dots)
\item Consumentgericht
\item (Games)
\end{itemize}

\end{frame}

%----------------------------------------------

\begin{frame}
\frametitle{Ter Vergelijking}

\begin{tabular}{|l|l|l|l|l|}
\hline Jaar & Model & Toen & Omg. & Nu \\ \hline
1977 & PET & \$795 & \$3150 & \$600-\$800 \\
1980 & VIC-20 & \$299.95 & \$874 & \$30-\$110 \\
1982 & C64 & \$595 & \$1480 & \$10-\$50 \\ \hline
1977 & Apple ][ & \$1296 & \$5130 & \$200-\$750 \\
1981 & IBM PC & \$1565 & \$4130 & \$150-\$1100 \\
1982 & ZX Spec. & \pounds 175 & \pounds 567 & \pounds 80-\pounds 120 \\ \hline
1977 & VCS & \$199 & \$768 & \$40-\$80 \\
1983 & NES & \$179 & \$433 & \$25-\$120 \\ \hline
\end{tabular}

\end{frame}

%----------------------------------------------

\begin{frame}
\frametitle{En harddisk opslag?}

\begin{itemize}
\item Niet nodig
\item Te duur (+\$500/+\$2,000)
\end{itemize}

\end{frame}

%----------------------------------------------

\begin{frame}
\frametitle{Data-opslag}

\begin{itemize}
\item Programma's in RAM, firmware in ROM
\item Power loss: programma weg!
\end{itemize}

\end{frame}

%----------------------------------------------

\begin{frame}
\frametitle{Tapes!}

\begin{figure}
\includegraphics[width=0.6\linewidth]{images/datasette.jpg}
\end{figure}

\end{frame}

%----------------------------------------------

\begin{frame}
\frametitle{Tapes!}

\begin{itemize}
\item Sequentieel magnetisch geheugen
\item Enkele MBs
\item Enkel-/dubbelzijdig
\end{itemize}

\end{frame}

%----------------------------------------------

\begin{frame}
\frametitle{Programma opzoeken}

\begin{itemize}
\item Onhandig
\item Handmatig heen en weer zoeken
\item Analoge file table
\end{itemize}

\end{frame}

%----------------------------------------------

\begin{frame}
\frametitle{Floppies!}

\begin{figure}
\includegraphics[width=0.7\linewidth]{images/c1541.jpg}
\end{figure}

\end{frame}

%----------------------------------------------

\begin{frame}
\frametitle{Floppies!}

\begin{itemize}
\item Echt floppy
\item 8 inch, 5.25 inch
\item SD/DD
\item 300-600 RPM (5-10 revs/sec)
\item 0.1MB-2.0MB
\end{itemize}

\end{frame}

%----------------------------------------------

\begin{frame}
\frametitle{Ontmoet de Succesvolste PC ooit}

\begin{figure}
\includegraphics[width=0.75\linewidth]{images/desk.jpg}
\end{figure}

\end{frame}

%----------------------------------------------

\begin{frame}
\frametitle{De Enige Echte}

\begin{figure}
\includegraphics[width=0.6\linewidth]{images/c64.jpg}
\end{figure}

\end{frame}

%----------------------------------------------

\begin{frame}
\frametitle{De Enige Echte}

\begin{figure}
\includegraphics[width=0.6\linewidth]{images/c64c.jpg}
\end{figure}

\end{frame}

%----------------------------------------------

\begin{frame}
\frametitle{Wat kan een C64?}

\begin{figure}
\includegraphics[width=0.8\linewidth]{images/c64back.png}
\end{figure}

\end{frame}

%----------------------------------------------

\begin{frame}
\frametitle{Wat kan een C64?}

\begin{figure}
\includegraphics[width=0.8\linewidth]{images/performance.png}
\end{figure}

\end{frame}

%----------------------------------------------

\begin{frame}
\frametitle{Waarom werken met Vintage PCs?}

\begin{itemize}
\item Enorme uitdaging
\item Simpel ontwerp, simpel te begrijpen
\item Literatuur
\item Programmeerwedstrijden (ja, 30 jaar lang!)
\end{itemize}

\end{frame}

%----------------------------------------------

\begin{frame}
\frametitle{C64 wordt nog steeds gebruikt!!}

\begin{figure}
\includegraphics[width=0.75\linewidth]{images/c64c_garage.jpg}
\end{figure}

\end{frame}

%----------------------------------------------

\begin{frame}
\frametitle{Er wordt nog steeds software gemaakt!}

\begin{itemize}
\item Tijdschriften, nieuwsbrieven, blogs, \dots
\item Offence \& Prosonix - Area64
\item Mayday! - Fallen Stars
\end{itemize}

\end{frame}

%----------------------------------------------

\begin{frame}
\frametitle{Programmeertalen}

\begin{itemize}
\item Microsoft BASIC
\item Machine code (assembly)
\item Pascal
\item K\&R C
\end{itemize}

\end{frame}

%----------------------------------------------

\begin{frame}
\frametitle{Hoe Programmeerden we vroeger?}

\begin{itemize}
\item Geen internet
\item Literatuur (tijdschriften, boeken, \dots)
\item Overlevering (bijeenkomsten, wedstrijden, \dots)
\end{itemize}

\end{frame}

%----------------------------------------------

\begin{frame}{Programmeren in BASIC}

\begin{itemize}
\item Makkelijk te leren, makkelijk te begrijpen
\item Al ingebouwd in C64
\item 68 BASIC commando's
\item Broncode is genummerd per regel
\item Numerieke variabelen, strings, lijsten
\item Code/Data
\end{itemize}

\end{frame}

%----------------------------------------------

\begin{frame}[fragile]{Simpel Programma - BASIC}

\begin{lstlisting}
10 PRINT "HALLO VIA"
20 GOTO 10
\end{lstlisting}

\end{frame}

%----------------------------------------------

\begin{frame}[fragile]{Simpel Programma - BASIC}

\begin{lstlisting}
10 FOR I=0 TO 16
20 PRINT 2^I, I
30 NEXT I
\end{lstlisting}

\end{frame}

%----------------------------------------------

% is the same as
\begin{frame}[fragile]{Simpel Programma - BASIC}

\begin{lstlisting}
10 FORI=0TO16
20 PRINT2^I,I
30 NEXTI
\end{lstlisting}

\end{frame}

%----------------------------------------------

% as well as
\begin{frame}[fragile]{Simpel Programma - BASIC}

\begin{lstlisting}
10FORI=0TO16:PRINT2^I,I:NEXTI
\end{lstlisting}

\end{frame}

%----------------------------------------------

\begin{frame}{Intermezzo}

\begin{itemize}
\item Ga naar https://www.github.com/FolkertVanVerseveld/workshop
\item Download het release archief en pak het uit
\item Probeer wat opdrachten te maken
\item Probeer zelf een BASIC programma te schrijven
\end{itemize}

\end{frame}

%----------------------------------------------

\begin{frame}{Assembly}

\begin{itemize}
\item Simpel, maar diepere leercurve
\item 151 offici\"ele instructies
\item Kleine voorbeelden
\end{itemize}

\end{frame}

%----------------------------------------------

\begin{frame}[fragile]{Simpel Programma - BASIC}

\begin{lstlisting}
10 POKE 1024,8
20 POKE 1025,5
30 POKE 1026,25
40 END
\end{lstlisting}

\end{frame}

%----------------------------------------------

\begin{frame}[fragile]{Simpel Programma - assembly}

\begin{lstlisting}
* = $0810
lda #$08
sta 1024
lda #$05
sta 1025
lda #$19
sta 1026
rts
\end{lstlisting}

\end{frame}

%----------------------------------------------

\begin{frame}{Simpel Programma - assembly}

\begin{tabular}{l|l|l|l}
Addr. & Mach. Code & Dissassembly & Pseudo Code \\
 & & * = \$0810 & \\
0810 & A9 08 & LDA \#\$08 & A = `H' \\
0812 & 8D 00 04 & STA \$400 & mem[1024] = A \\
0815 & A9 05 & LDA \#\$05 & A = `E' \\
0817 & 8D 01 04 & STA \$401 & mem[1025] = A \\
081A & A9 19 & LDA \#\$19 & A = `Y' \\
081C & 8D 02 04 & STA \$402 & mem[1026] = A \\
081F & 60 & RTS & return \\
\end{tabular}

\end{frame}

%----------------------------------------------

\begin{frame}{Ter Vergelijking - Instruction Set}

\begin{tabular}{|l|l|l|l|}
\hline CPU & Regs & Reg Bits & Mul/Div \\ \hline
8086 & 14 & 8/16 & Mul \& Div \\
65XX & 6 & 8 & geen \\
Z80 & 14 & 8/16 & Mul \\ \hline
\end{tabular}

\end{frame}

%----------------------------------------------

\begin{frame}{6502 registers}

\begin{tabular}{|l|l|l|}
\hline Naam & Bits & Doel \\ \hline
A & 8 & Accumulator \\
X & 8 & Index register \\
Y & 8 & Index register \\
SP & 8 & Stack Pointer \\
PC & 16 & Program Counter \\
FLAGS & 8 & Processor Status Word \\ \hline
\end{tabular}

\end{frame}

%----------------------------------------------

\begin{frame}{6502}

\begin{itemize}
\item Erg simpel en elegant
\item 256 bytes stack (d.w.z. 128 call-stack)
\item IRQ, NMI
\end{itemize}

\end{frame}

%----------------------------------------------

\begin{frame}
\frametitle{Graphics}

\begin{figure}
\includegraphics[width=0.5\linewidth]{images/palette.png}
\end{figure}

\end{frame}

%----------------------------------------------

\begin{frame}
\frametitle{Graphics}

\begin{itemize}
\item 320x200 pixels
\item 16 kleuren palette
\item Maximaal 2 of 4 kleuren per 8x8 blok.
\end{itemize}

\end{frame}

%----------------------------------------------

\begin{frame}
\frametitle{Ter Vergelijking}

\begin{tabular}{|l|l|l|}
\hline Model & Video & Kleuren \\
PET & ??? & 1 \\
VIC-20 & VIC I & 4-16??? \\
C64 & VIC II & 16 \\
Apple ][ & ??? & 1-4 \\
IBM PC & mono of CGA & 1-4 \\
ZX Spec. & geen (CPU) & 8 \\ \hline
NES & Ricoh 2C02 PPU & 52 \\ \hline
\end{tabular}

\end{frame}

%----------------------------------------------

\begin{frame}
\frametitle{Graphics Memory}

\begin{itemize}
\item 320x200 = 64,000
\item Maar 65,536 - 64,000 = 1,536 bytes over met bitmap
\item Hoe lossen we dit op?
\end{itemize}

\end{frame}

%----------------------------------------------

\begin{frame}
\frametitle{Graphics Memory}

\begin{itemize}
\item Gebruik 8x8 blokken
\item Tekst modus
\item Maar 40 * 25 = 1,000 bytes nodig
\item Met 16 colors maar 1,000 / 2 = 500 bytes meer
\item 1,500 bytes
\end{itemize}

\end{frame}

%----------------------------------------------

\begin{frame}
\frametitle{Graphics}

\begin{itemize}
\item Scherm tekstdata vanaf \$0400 (== 1024)
\item Color RAM vanaf \$D800 (== 55296)
\end{itemize}

\end{frame}

%----------------------------------------------

\begin{frame}
\frametitle{Graphic modes}

\begin{itemize}
\item Standard Character (text) Mode
\item Multicolor Character (text) Mode
\item Standard Bitmap Mode
\item Multicolor Bitmap Mode
\item Extended Background Color Mode
\end{itemize}

\end{frame}

%----------------------------------------------

\begin{frame}
\frametitle{Multicolor Character Mode}

\begin{figure}
\includegraphics[width=0.5\linewidth]{images/gary.png}
\end{figure}

\end{frame}

%----------------------------------------------

\begin{frame}
\frametitle{Standard Bitmap}

\begin{figure}
\includegraphics[width=0.5\linewidth]{images/sbm_obey.png}
\end{figure}

\begin{center}
Source: Archmage
\end{center}

\end{frame}

%----------------------------------------------

\begin{frame}
\frametitle{Standard Bitmap}

\begin{figure}
\includegraphics[width=0.5\linewidth]{images/sbm_tuksu.png}
\end{figure}

\begin{center}
Source: Duce
\end{center}

\end{frame}

%----------------------------------------------

\begin{frame}
\frametitle{Graphics Programming}

\begin{itemize}
\item Eigen grafische driver schrijven
\item 64 bytes (17 ongebruikt)
\item Dus 47 registers
\end{itemize}

\end{frame}

%----------------------------------------------

\begin{frame}
\frametitle{VIC registers}

\begin{tabular}{|l|l|}
\hline Doel & \# Registers \\ \hline
% spr[0-7]x, spr[0-7]y, sprxh, spre, sdh, sdw, smc, sprio
% 8 + 8 + 6
Sprites & 22 \\
% cse0, cse1
Sprite color & 10 \\
% cbor, cbkg, cbe0, cbe1, cbe2
Border/Foreground color & 5 \\
% ssc, sbkc
Collision Detection & 2 \\
% 1 MC mode & screen width etc.
% 1 screen height & text/bitmap etc.
% scrc0, scrc1
Screen control & 2 \\
% 1 current raster line
% crl, isr, icr
Interrupt control & 3 \\
% msr
Memory setup & 1 \\
% 2 lightpen
% lpx, lpy
Light pen & 2 \\ \hline
\end{tabular}

\end{frame}

%----------------------------------------------

\begin{frame}[fragile]{Simpel Graphics Programma - BASIC}

\begin{lstlisting}
10 VIC=53248
20 FORC=0TO15
20 POKEVIC+32,C
40 FORT=1TO128:NEXT
50 NEXTC
\end{lstlisting}

\end{frame}

%----------------------------------------------

\begin{frame}
\frametitle{Hoe plaatsen we individuele pixels?}

\begin{itemize}
\item Onhandig: Grid/blok-aligned
\item Color clash (bleeding)
\end{itemize}

\end{frame}

%----------------------------------------------

\begin{frame}
\frametitle{Bleeding bij MSX}

\begin{figure}
\includegraphics[width=0.5\linewidth]{images/bleeding.png}
\end{figure}

\end{frame}

%----------------------------------------------

\begin{frame}
\frametitle{Welkom in de wereld van Sprites!}

\begin{itemize}
\item Movable Objects (MOBs)
\item 8 sprites bestaande uit 24*21 pixels
\item 63 bytes per sprite
\item X en Y co\"ordinaat
\end{itemize}

\end{frame}

%----------------------------------------------

\begin{frame}
\frametitle{Sprite scaling}

\begin{figure}
\includegraphics[width=0.5\linewidth]{images/sprdwh.png}
\end{figure}

\end{frame}

%----------------------------------------------

\begin{frame}
\frametitle{Great Giana Sisters}

\begin{itemize}
\item Lijkt op Super Mario Bros
\item 33 stages
\end{itemize}

\end{frame}

%----------------------------------------------

\begin{frame}
\frametitle{Graphics Hacking}

\begin{itemize}
\item CRT signaleert HSYNC, VSYNC
\item Raster Interrupts
\end{itemize}

\end{frame}

%----------------------------------------------

\begin{frame}
\frametitle{Graphics Hacking - Raster bars}

\begin{itemize}
\item Horizontaal: Double Hotline - R-Type+
\item Verticaal \& Ronddraaiend: Booze Design - Uncensored
\item Wiebelen: Censor \& Oxyron - Comalight X14
\end{itemize}

\end{frame}

%----------------------------------------------

\begin{frame}
\frametitle{Graphics Hacking - Effects/hacks/VIC bugs}

\begin{itemize}
\item Borderless graphics
\item Raster split
\item Parallax scrolling
\item Andere graphic modes
\item Plasma
\item Meer dan 8 sprites
\item \dots
\end{itemize}

\end{frame}

%----------------------------------------------

\begin{frame}{Intermezzo}

\begin{itemize}
\item Ga verder met de opgaven
\item Of: schrijf een BASIC programma met wat grafische effecten
\item Geadvanceerd: ga naar: https://skilldrick.github.io/easy6502/ en kijk naar
de voorbeelden
\end{itemize}

\end{frame}


%----------------------------------------------

\begin{frame}
\frametitle{Audio}

\begin{itemize}
\item Analoog of Digitaal
\end{itemize}

\begin{figure}
\includegraphics[width=0.6\linewidth]{images/waveforms.png}
\end{figure}

\end{frame}

%----------------------------------------------

\begin{frame}
\frametitle{Analoog Audio}

\begin{itemize}
\item Niet-elektronisch
\item Visualisatie met Oscilloscoop
\end{itemize}

\end{frame}

%----------------------------------------------

\begin{frame}
\frametitle{Digitaal Audio}

\begin{itemize}
\item Benadering van Analoog
\item Stapsgewijs
\item Duidelijk zichtbaar op oude PCs
\end{itemize}

\end{frame}

%----------------------------------------------

\begin{frame}
\frametitle[fragile]{Digitaal Audio}

\begin{figure}
	\begin{subfigure}[b]{0.4\textwidth}
		\includegraphics[width=\linewidth]{images/waveforms.png}
	\end{subfigure}
	\begin{subfigure}[b]{0.5\textwidth}
		\includegraphics[width=\linewidth]{images/rc3nes.png}
	\end{subfigure}
\end{figure}

\end{frame}

%----------------------------------------------

\begin{frame}
\frametitle{Digitaal Audio - Speaker}

\begin{itemize}
\item Squaker
\item Zeer ruw
\item Aan of uit
\item Frequentie
\end{itemize}

\end{frame}

%----------------------------------------------

\begin{frame}
\frametitle{Digitaal Audio - Speaker Voorbeelden}

\begin{itemize}
\item Loop
\item Single
\item Arpeggio
\end{itemize}

\end{frame}

%----------------------------------------------

\begin{frame}
\frametitle{Digitaal Audio - NES}

\begin{tabular}{|l|l|l|l|}
\hline \# & type & volumes & frequenties \\ \hline
2 & Pulse wave & 16 & 54Hz-28kHz \\
% XXX er stond 447kHz als upper bound, maar kan toch niet kloppen?
1 & White noise & 16 & 29.3Hz-47kHz \\
1 & Triangle wave & 1 & 27Hz-56kHz \\
1 & Delta PCM & 16 & 4kHz-33kHz \\ \hline
\end{tabular}

\end{frame}
%----------------------------------------------

\begin{frame}
\frametitle{Digitaal Audio - NES}

\begin{itemize}
\item 5 kanalen
\item 8 bit voor pulse, noise en triangle
\item 1 bit voor DPCM
\item hardware pitch bending
\end{itemize}

\end{frame}

%----------------------------------------------

\begin{frame}
\frametitle{Digitaal Audio - NES voorbeeld}

\begin{itemize}
\item The Legend of Zelda theme
\end{itemize}

\end{frame}

%----------------------------------------------

\begin{frame}
\frametitle{Digitaal Audio}

\begin{figure}
\includegraphics[width=0.9\linewidth]{images/rc3nes.png}
\end{figure}

\end{frame}

%----------------------------------------------

\begin{frame}
\frametitle{C64: SID chip}

\begin{itemize}
\item 's werelds eerste audio synthesizer
\item Analoog \emph{en} Digitaal
\item Verschillende onderscheidingen
\item 6581 en 8580
\item Timbaland plagiaat
\end{itemize}

\end{frame}

%----------------------------------------------

\begin{frame}
\frametitle{Ter Vergelijking}

\begin{tabular}{|l|l|l|l|}
\hline Model & Geluid & Type & Voices \\ \hline
PET & CPU PIO & 1-bit beeper & 0 \\
VIC-20 & VIC I & 8-bit digital & 4 \\
C64 & SID & A/D 8-bit synthesizer & 3 \\ \hline
Apple ][ & Speaker & 1-bit beeper & 0 \\
IBM PC & Speaker & 1-bit beeper & 0 \\
NES & RP2A03 & 8-bit digitaal & 5 \\ \hline
\end{tabular}

\end{frame}

%----------------------------------------------

\begin{frame}
\frametitle{Audio}

\begin{itemize}
\item 3 kanalen (voices)
\item ADSR envelope
\end{itemize}

\end{frame}

%----------------------------------------------

\begin{frame}
\frametitle{ADSR}

\begin{figure}
\includegraphics[width=0.8\linewidth]{images/adsr.png}
\end{figure}

\end{frame}

%----------------------------------------------

\begin{frame}
\frametitle{Waveforms}

\begin{itemize}
\item Sawtooth
\item Square
\item Sinuso\"ide
\item Triangle
\item Noise
\end{itemize}

\end{frame}

%----------------------------------------------

\begin{frame}
\frametitle{Waveforms}

\begin{figure}
\includegraphics[width=0.75\linewidth]{images/waveforms.png}
\end{figure}

\end{frame}

%----------------------------------------------

\begin{frame}
\frametitle{SID - Technieken}

\begin{itemize}
\item Pulse Width Modulation (PWM)
\item Ring modulation (ring mod)
\end{itemize}

\end{frame}

%----------------------------------------------

\begin{frame}
\frametitle{SID - Voorbeelden}

\begin{itemize}
\item Driller (PWM)
\item Commando
\item Zoids (ring mod)
\end{itemize}

\end{frame}

%----------------------------------------------

\begin{frame}
\frametitle{Intermezzo}

\begin{block}{Muziek}
Raad welk muziekje bij welke computer/console hoort!
\end{block}

\begin{itemize}
\item IBM PC
\item NES
\item C64
\end{itemize}

\end{frame}

%----------------------------------------------

\begin{frame}
\frametitle{Intermezzo - Voorbeelden}

\begin{itemize}
\item Grand Prix Circuit (PC, C64)
\item Castlevania (PC, NES, C64)
\end{itemize}

\end{frame}

%----------------------------------------------

\begin{frame}
\frametitle{Intermezzo}

\begin{itemize}
\item IBM PC
\item NES
\item C64
\end{itemize}

\end{frame}

%----------------------------------------------

\begin{frame}
\frametitle{SID hacking}

\begin{itemize}
\item Click/pop-bug in 6581
\item 4-bit PCM digitaal
\end{itemize}

\end{frame}

%----------------------------------------------

\begin{frame}
\frametitle{SID hacking - Arkanoid}

\begin{figure}
\includegraphics[width=\linewidth]{images/farts.png}
\end{figure}

\end{frame}

%----------------------------------------------

\begin{frame}
\frametitle{SID hacking - 6581 examples}

\begin{tabular}{|l|l|l|}
\hline Year & Artist & Title \\ \hline
1987 & Martin Galway & Arkanoid \\
1987 & Chris H\"ulsbeck & Dulcedo Cogitationis \\
1988 & Jeroen Tel & Savage \\
1989 & Jeroen Tel & Stormlord \\
1989 & Jeroen Tel & Turbo Outrun \\
1990 & J. Tel and C. H\"ulsbeck & Turrican \\ \hline
\end{tabular}

\end{frame}


%----------------------------------------------

\begin{frame}
\frametitle{Programmeerwedstrijden}

\begin{itemize}
\item X (in Nederland!)
\item Revision (grootste ter wereld!)
\item Nordlicht
\item Under Construction
\item \dots
\end{itemize}

\end{frame}

%----------------------------------------------

\begin{frame}
\frametitle{X}

\begin{figure}
\includegraphics[width=0.8\linewidth]{images/x.jpg}
\end{figure}

\end{frame}

%----------------------------------------------

\begin{frame}
\frametitle{X}

\begin{itemize}
\item sinds 1995
\item Vanaf 2004 \'e\'en keer per 2 jaar
\item Vorig jaar ong. 400 bezoekers
\end{itemize}

\end{frame}

%----------------------------------------------

\begin{frame}
\frametitle{Revision}

\begin{figure}
\includegraphics[width=0.8\linewidth]{images/rev2017.jpg}
\end{figure}

\end{frame}

%----------------------------------------------

\begin{frame}
\frametitle{Revision}

\begin{itemize}
\item Sinds 2011 (daarvoor: Breakpoint)
\item Ong. 700 bezoekers
\item Grote bedrijven bieden banen aan!
\item Seminars
\item 30+ wedstrijden (demo, gfx, fotografie, \dots)
\end{itemize}

\end{frame}

%----------------------------------------------

\begin{frame}
\frametitle{Opdracht}

\begin{itemize}
\item Vorm groepen van maximaal 5
\item Maak een demo!
\end{itemize}

\end{frame}

%----------------------------------------------

\begin{frame}
\frametitle{Wil je een C64?}

\begin{itemize}
\item Dat kan!
\item Duur op ebay en marktplaats
\item Goedkoop bij rommelmarkt
\item Toetsenbord
\item HCC
\end{itemize}

\end{frame}

%----------------------------------------------

\begin{frame}
\frametitle{HCC}

\begin{itemize}
\item Nederlandse Hobby Computer Club
\item Sinds 27 april 1977
\item Diverse interessegroepen
\item Elke twee maanden in Maarssen
\item Retro (8/16 bit) computers welkom
\item Ultimate II+
\item Swappen
\end{itemize}

\end{frame}

%------------------------------------------------

\begin{frame}
\frametitle{Vervolg}

\begin{itemize}
\item Demoscene
\item Oldskool computers
\item Programmeerwedstrijden
\item Cracking 'n Hacking
\end{itemize}

\end{frame}

%------------------------------------------------

\begin{frame}
\frametitle{HVSC}

\begin{figure}
\includegraphics[width=0.6\linewidth]{images/hvsc.jpg}
\end{figure}

\begin{itemize}
\item 49542 SIDs
\item Ongeveer 1500 artiesten
\item Ongeveer 300MB
\item Dus ongeveer 6KB per SID!
\item Meer dan 16000 uur muziek!
\end{itemize}

\end{frame}

%------------------------------------------------

\begin{frame}
\frametitle{Pouet}

\begin{figure}
\includegraphics[width=0.6\linewidth]{images/pouet.jpg}
\end{figure}

\begin{itemize}
\item 92 platforms
\item 36 categorie\"en
\item Ongeveer 70,000 prods!
\end{itemize}

\end{frame}

%------------------------------------------------

\begin{frame}
\frametitle{Conclusies}

\begin{itemize}
\item PC Geschiedenis
\item Hacken op C64
\item Programmeren tot het uiterste
\item Programmeerwedstrijden
\end{itemize}

\end{frame}

%------------------------------------------------

%------------------------------------------------

\begin{frame}
\frametitle{Met dank aan}

\begin{itemize}
\item Lezingencommissie
\item VIA
\item CCC
\item HCC, lemon64, csdb, VICE
\item Hobbyisten en C64 fans
\item Scene
\end{itemize}

\end{frame}

%------------------------------------------------

\begin{frame}
\frametitle{Met dank aan}

\begin{multicols}{3}
\begin{itemize}
\item Abyss Connection
\item Abnormal
\item Albion Crew
\item Alpha Flight
\item Algotech
\item Arkanix Labs
\item Artline Designs
\item Arise
\item Artstate
\item Arsenic
\item Atlantic
\item Atlantis
\item Atw
\item Bauknecht
\item Black Sun
\item Bluez Muz
\item Bonzai
\item Booze Design
\item Byterapers
\item Beyond Force
\end{itemize}
\end{multicols}

\end{frame}

%------------------------------------------------

\begin{frame}
\frametitle{Met dank aan}

\begin{multicols}{3}
\begin{itemize}
\item Camelot
\item Cascade
\item Censor
\item Chorus
\item Chrome
\item Cosine
\item Covert Bitops
\item Creators
\item Crescent
\item Crest
\item Darklite
\item Delysid
\item Dekadence
\item Desire
\item Dmagic
\item Dual Crew
\item Elysium
\item Extend
\item Excess
\item Exon
\item Fairlight
\item Fatzone
\item Focus
\item Fossil
\end{itemize}
\end{multicols}

\end{frame}

%------------------------------------------------

\begin{frame}
\frametitle{Met dank aan}

\begin{multicols}{3}
\begin{itemize}
\item Genesis Project
\item Glance
\item Hack 'n 'trade
\item Hitman
\item Hoaxers
\item Holger
\item Hokuto Force
\item Horizon
\item House Designs
\item HVSC Crew
\item Judas
\item Laxity
\item Lepsi De
\item Level 64
\item LFT
\item Mahoney
\item Maniacs of Noise
\item Mayday
\item MDG
\end{itemize}
\end{multicols}

\end{frame}

%------------------------------------------------

\begin{frame}
\frametitle{Met dank aan}

\begin{multicols}{3}
\begin{itemize}
\item Metalvotze
\item Miracles
\item Multistyle Labs
\item The New Dimension
\item Nah-Kolor
\item Noice
\item No Name
\item Nostalgia
\item Nuance
\item Offence
\item Onslaught
\item Oxyron
\item Padua
\item Panda Design
\item Plush
\item Poo-brain
\item Prosonix
\item Radwar
\item Razor
\item Reflex
\item Resource
\item Samar
\item Scene sat
\item Shape
\item Sidrip
\end{itemize}
\end{multicols}

\end{frame}

%------------------------------------------------

\begin{frame}
\frametitle{Met dank aan}

\begin{multicols}{3}
\begin{itemize}
\item Singular
\item Slay Radio
\item Smash Designs
\item Starion
\item Style
\item SCS*TRC
\item TPUG
\item Triad
\item Tropyx
\item TRSI
\item Unicess
\item Up Rough
\item Vandalism News
\item Vibrants
\item VICE team
\item Vision
\item Wow
\item Wrath Designs
\item Xenon
\item You Lazy Bastards
\end{itemize}
\end{multicols}

\end{frame}


%------------------------------------------------

\begin{frame}
\frametitle{Bedankt voor jullie aandacht}
\Large{\centerline{Zijn er nog vragen?}}
\end{frame}

\end{document}
