\section{Opslaan/Laden van Programma's}

We gaan eerst kijken hoe we een floppydisk kunnen aanmaken en daarna kijken we naar het laden en opslaan van programma's, spellen en dergelijke.

Als eerste is het belangrijk dat we kijken of de emulator de echte trage laadtijden emuleert omdat veel programma's hiervan afhankelijk zijn.
Deze optie kunt u vinden in \verb:Settings->Drive settings->True drive emulation: (zie figuur \ref{fig:true_c1541}).
Als u deze optie niet kunt vinden dan is de drive emulation al goed ingesteld en kunt u dit negeren.

\begin{figure}
\centering
\includegraphics[width=0.75\linewidth]{images/true_c1541.png}
\caption{Aanzetten van echte floppy emulatie}
\label{fig:true_c1541}
\end{figure}

\subsection{Snel laden van programma's}

De emulator biedt gelukkig een manier om een floppy, disk of cartridge aan te sluiten en het direct te starten.
Dit kan met \verb:File->Smart-attach disk/tape: of met \verb:ALT+A:.
U kunt na het kiezen van een bestand dan op \verb:Open: of \verb:Autostart: klikken.
\verb:Autostart: zal het programma direct opstarten terwijl \verb:Open: alleen het opslagmedium aansluit.

\problem Autostart de demo \verb:roms/cracktros/hotline/htl-41.prg: om te kijken of de emulator goed werkt.

In de appendix kunt u verwijzingen vinden naar websites voor als u nog meer software wilt proberen (en ook in \verb:roms/:).
%Spellen kunt u vinden in \verb:roms/games: en demo's voor programmeerwedstrijden in \verb:roms/parties:.

\subsection{Cheatsheet opslaan/laden van programma's}

Hieronder volgt een beknopt overzicht van de BASIC commando's om programma's te laden en op te slaan:

\begin{tabular}{l|l}
Taak & Commando \\ \hline
Laad lijst van programma's & \verb:LOAD"$",8: \\
Laad programma genaamd \verb:TEST: & \verb:LOAD"TEST",8,1: \\
Laad eerste programma van disk & \verb:LOAD"*",8,1: \\
Sla programma op & \verb:SAVE"NAAM",8:
\end{tabular}

Hierbij is \verb:8: de drive letter.
Met \verb:RUN: kunt u een programma starten en met \verb:LIST: de lijst van programma's of de BASIC code van het geladen programma tonen.

In de volgende secties kijken we hoe we floppydisks kunnen maken en worden de \verb:SAVE: en \verb:LOAD: commando's uitgebreider besproken.

\subsection{Aanmaken en aansluiten van een floppydisk}

De makkelijkste manier om een floppydisk aan te maken is door deze in de werkbalk aan te maken met \verb:File->Create and attach an empty disk->Unit #8: en
vervolgens kiest u een naam voor het bestand en de \verb:Disk name: met het type \verb:d64: (zie figuur \ref{fig:create_d64}).

\begin{figure}
\centering
\includegraphics[width=0.75\linewidth]{images/create_d64.png}
\caption{Aanmaken van een floppydisk voor drive 8}
\label{fig:create_d64}
\end{figure}

Als het goed is kunt de floppydisk inladen met \verb:LOAD"$",8: waarna na een aantal seconden \verb:READY: zal verschijnen.
Nu kunt u met \verb:LIST: een directory listing tonen.
Het resultaat is vergelijkbaar met figuur \ref{fig:d64dummy}.

\begin{figure}
\centering
\includegraphics[width=0.75\linewidth]{images/d64dummy.png}
\caption{Lege floppydisk}
\label{fig:d64dummy}
\end{figure}

Als dit niet werkt, of u houdt van command-line tools, kunt u het onderstaande recept proberen.

\subsection{Handmatig aanmaken van floppydisk met c1541}

Met het floppy hulpprogramma c1541 kunt u een disk maken.
Als we bijvoorbeeld een disk genaamd `muziek' willen maken op linux kunnen we dit doen:

\begin{lstlisting}
c1541 -format diskname,id d64 muziek.d64
\end{lstlisting}

Of op windows:

\begin{lstlisting}
c1541.exe -format diskname,id d64 muziek.d64
\end{lstlisting}

Of op een mac:

% XXX make sure this is correct. It may be ./x64.app/.../c1541 but I'm not sure
\begin{lstlisting}
./c1541.app/Contents/MacOS/c1541 -format diskname,id d64 muziek.d64
\end{lstlisting}

Vervang `muziek' met de naam van de disk.

Op windows en Linux kunt u de floppydisk aan de drive aansluiten zelfs nadat u de emulator gestart hebt met \verb:File->Attach disk image->Unit #8: of met \verb:ALT+8:.

\subsection{Opslaan/Laden van programma's op Mac}

Als u programma's wil opslaan of laden moet u de floppydisk aansluiten \emph{voordat} u de emulator start. Bijvoorbeeld:

\begin{lstlisting}
./x64.app/Contents/MacOS/x64 pad/naar/uw/floppy.d64
\end{lstlisting}

\subsection{Programma's opslaan}

Nadat u een BASIC programma heeft ingevoerd en de floppydisk aan drive 8 heeft aangesloten kunt u met \verb:SAVE"NAAM",8: een programma opslaan.
Als u het programma wilt overschrijven zult u dit in meerdere stappen moeten doen.
Als eerste moet het programma verwijderd worden:

\begin{lstlisting}
OPEN1,8,15,"I":CLOSE1
OPEN1,8,15,"S:NAAM":CLOSE1
\end{lstlisting}

Vervang \verb:NAAM: met het programma wat u wilt verwijderen.
Nadat het verwijderd is, kunt u het opnieuw bewaren met \verb:SAVE"NAAM",8:.

\subsection{Programma's laden}

Nadat u de floppydisk heeft aangesloten kunt u het programma van drive 8 laden met \verb:LOAD"$",8,1: waarna u \verb:LIST: kan intypen om de lijst van alle programma's te zien.
Een bestand laden kan met \verb:LOAD"NAAM",8,1:.

Als u het eerste programma van de disk wilt laden kunt u ook dit gebruiken: \verb:LOAD"*",8,1:.

Het is ook mogelijk om \verb:,1: weg te laten bij het \verb:LOAD: commando, maar sommige programma's zullen dan niet goed ingeladen worden.
Wat dit precies betekent valt buiten het domein van deze opgaven.
