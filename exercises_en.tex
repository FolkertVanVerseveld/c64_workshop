\documentclass{article}

\usepackage{amsmath}
\usepackage{hyperref}
\usepackage{listings}
\usepackage{graphicx}
\usepackage{eurosym}

\title{Exercises for workshop Commodore 64}
\author{Folkert van Verseveld}

\newcounter{problem}
\newcounter{solution}

\newcommand\problem{%
  \stepcounter{problem}%
  \textbf{\theproblem.}~%
  \setcounter{solution}{0}%
}
\parindent 0in
\parskip 1em

\begin{document}
\maketitle

\section{Introduction}

This bundle contains a set of exercises with increasing difficulty.
The difficult exercises may be skipped.

\subsection{VICE setup}
This workshop uses the Versatile Commodore Emulator (VICE).
You can download the emulator \href{http://vice-emu.sourceforge.net/}{from this site}.
Depending on your operating system, some additional setup may be required as will be discussed in the next section.

\subsubsection{Linux}
Setting up the emulator depends on your Linux distribution.
If your Linux distro is derived from Debian (e.g. Ubuntu or Xubuntu) you can setup the emulator by running the script as root:
\verb:sudo sh vice.sh:.
If this does not work, but you have a different package manager (e.g. \verb:yum: rather than \verb:apt:), you can modify the variables in the script and (re)run it.

If this does not work either, but you have Wine installed, you can also use the Windows recipe.
The die-hards may also try to compile the \href{http://sourceforge.net/projects/vice-emu/files/releases/vice-3.1.tar.gz/download}{source code}.

\subsubsection{Windows}
Download the \href{http://sourceforge.net/projects/vice-emu/files/releases/binaries/windows/WinVICE-3.1-x64.7z/download}{windows version}.
Other versions can be downloaded from the \href{http://vice-emu.sourceforge.net/windows.html}{download section}.
You need \href{http://www.7-zip.org/}{7Zip} in order to extract the archive.
Extract all files and run \verb:x64.exe:.

\subsubsection{Mac}
Download the \href{http://sourceforge.net/projects/vice-emu/files/releases/binaries/macosx/vice-macosx-sdl-x86_64-10.12-3.1.dmg/download}{mac version}.
You need to extract the .dmg in order to use the emulator.
Please note that you have to run the emulator from a terminal because the menu bar is missing.

Open a terminal and navigate to the directory where you just extracted the emulator and run the emulator. For example, if you have extracted the emulator on your desktop:
\begin{lstlisting}
cd /Users/name/Desktop/vice-emu/
./x64.app/Contents/MacOS/x64
\end{lstlisting}

Replace \verb:name: with your user name.

\subsection{Tips and Tricks}

Please don't hesitate to try some things before giving up.
Some exercises are designed with this in mind.

If the emulator is haning you can try to reset it using \verb:File->Reset->Soft: to restart the C64.
If this does not work, you can try to hard reset it using \verb:File->Reset->Hard: or restart the emulator if that does not work either.

\emph{If you reset the emulator all BASIC programs are lost. Make sure that you save your BASIC program to disk first!
The floppydisk has to be reattached if you reset the emulator!}

You can speed up the emulator using \verb:ALT+W: and reduce program load time.
The slow loading times must be emulated because many software depends on this!
Use \verb:ALT+W: to start/stop speed up mode.

\section{Saving/Loading programs}

First, we are going to look how to create a floppydisk and see if we can load and save some programs, games etc.

It is important to check if the emulator is running in \verb:True drive emulation: mode, because many programs depend on this behavior.
You can check this at \verb:Settings->Drive settings->True drive emulation: (see figure \ref{fig:true_c1541}).
If you can't find this option it is probably setup up correctly and you can ignore this.

\begin{figure}
\centering
\includegraphics[width=0.75\linewidth]{images/true_c1541.png}
\caption{Enabling True drive emulation}
\label{fig:true_c1541}
\end{figure}

\subsection{Fast load programs}

The emulator enables you to autostart a program from floppy, disk or cartridge without doing everything manually.
You can autostart a program using \verb:File->Smart-attach disk/tape: or just \verb:ALT+A:.
After you selected the file you can \verb:Open: or \verb:Autostart: the program.
\verb:Autostart: will start the program immediately while \verb:Open: just connects/attaches the floppy.

\problem Autostart the demo \verb:roms/cracktros/hotline/htl-41.prg: to make sure the emulator is working properly.

In the appendix are multiple references to websites for more software you could try out as well (and also in \verb:roms/:).
%Spellen kunt u vinden in \verb:roms/games: en demo's voor programmeerwedstrijden in \verb:roms/parties:.

\subsection{Cheatsheet saving/loading programs}

The following is a simple table with BASIC commands for loading and saving programs:

\begin{tabular}{l|l}
Task & Command \\ \hline
Load program list & \verb:LOAD"$",8: \\
Load program named \verb:TEST: & \verb:LOAD"TEST",8,1: \\
Load first program from disk & \verb:LOAD"*",8,1: \\
Save program & \verb:SAVE"NAME",8:
\end{tabular}

\verb:8: is the drive letter.
With \verb:RUN: you can run a program and with \verb:LIST: you can print the program list or the BASIC code from a loaded BASIC program.

In the next sections we will look how to create floppydisk and how to \verb:SAVE: and \verb:LOAD: programs in more detail and also provide some tips and troubleshooting.

\subsection{Creating and attaching a floppydisk}

The easiest way to create a floppydisk is by navigating through the options from the menu bar \verb:File->Create and attach an empty disk->Unit #8: and
then type the program name in the \verb:Disk name: field and make sure the type is set \verb:d64: (see figure \ref{fig:create_d64}).

\begin{figure}
\centering
\includegraphics[width=0.75\linewidth]{images/create_d64.png}
\caption{Creating floppydisk for drive 8}
\label{fig:create_d64}
\end{figure}

Now you should be able to load the floppy using \verb:LOAD"$",8: and the C64 will print \verb:READY: after a few seconds.
Then you can \verb:LIST: the program list.
The result should look similar to figure \ref{fig:d64dummy}.

\begin{figure}
\centering
\includegraphics[width=0.75\linewidth]{images/d64dummy.png}
\caption{Empty floppydisk}
\label{fig:d64dummy}
\end{figure}

If this does not work, or you like using commmand-line tools, you can try out the following recipe.

\subsection{Manually create floppydisk using c1541}

You can create a floppy using a tool called c1541.
If we would like to create a disk named for example `music' we can do this on linux using:

\begin{lstlisting}
c1541 -format diskname,id d64 muziek.d64
\end{lstlisting}

Or on windows:

\begin{lstlisting}
c1541.exe -format diskname,id d64 muziek.d64
\end{lstlisting}

Or on a mac:

% XXX make sure this is correct. It may be ./x64.app/.../c1541 but I'm not sure
\begin{lstlisting}
./c1541.app/Contents/MacOS/c1541 -format diskname,id d64 muziek.d64
\end{lstlisting}

Change `music' with the name of your disk.

On windows and Linux you can attach the floppydisk to the drive even after you have booted the emulator using \verb:File->Attach disk image->Unit #8: or just \verb:ALT+8:.

\subsection{Save/Load programs on Mac}

If you want to save or load programs you have to attach the floppydisk \emph{before} starting the emulator. For example:

\begin{lstlisting}
./x64.app/Contents/MacOS/x64 path/to/your/floppy.d64
\end{lstlisting}

\subsection{Saving programs}

After typing in a BASIC program and attaching the floppydisk to drive 8 you can save the program using \verb:SAVE"NAME",8:.
If you want to overwrite a program you have to overwrite it in multiple steps.
First, we have to delete the program:

\begin{lstlisting}
OPEN1,8,15,"I":CLOSE1
OPEN1,8,15,"S:NAME":CLOSE1
\end{lstlisting}

Replace \verb:NAME: with the name of your program.
After deletion, you can resave the program using \verb:SAVE"NAME",8:.

\subsection{Loading programs}

After attaching the floppydisk you can load the program list from drive 8 using \verb:LOAD"$",8,1: and then \verb:LIST: the program list.
You can load a program using \verb:LOAD"NAME",8,1:.

If you want to load the first program from disk you can also use this: \verb:LOAD"*",8,1:.

It is possible to remove \verb:,1: from the \verb:LOAD: command, but this is discouraged as some programs may not load correctly.
Why this happens is out of scope of this workshop.


\section{Introduction BASIC}

The Commodore 64 understands machine code and the high-level programming language BASIC.
If you run the VICE emulator you will see a similar window to figure \ref{fig:vice}.

\begin{figure}
\centering
\includegraphics[width=0.5\linewidth]{images/boot.png}
\caption{VICE emulator}
\label{fig:vice}
\end{figure}

The default screens tells us that we are using BASIC version 2 and it has 64KiB of RAM leaving 38911 bytes for BASIC.
This means that our BASIC program cannot exceed 38.9KB!

You may have noticed that the default screen only uses upper case letters.
This is because many computer designers at that time deemed it unnessecary to support both upper and lower case letters at the same time!
It is possible to use both at the same time, but we are not going to use this.

\subsection{Typing}

You can use the spacebar to place a space character on the screen or to `erase' the current character.

\problem Try to change different portions of text on the screen and try to remove some text by overwriting them with spaces.

\problem What happens if we place the cursor to the far right and press the right arrow key?
What happens if we place the cursor to the far left and press the left arrow key?
And what happens if we place the cursor at the bottom and press the down arrow key?

\subsection{Keyboardconfiguration}

You may have noticed that the C64 keyboard layout is very different compared to a QWERTY keyboard.
If you would type \verb:SHIFT+2: and \verb:]: you will probably see \verb:@]: on the screen.
In case you see \verb:"@:, this means that the emulator uses the real C64 keyboard layout.
You can change the layout using \verb:Settings->Keyboard settings->Keyboard mapping type:.
A real C64 uses \verb:Positional mapping: and the emulator uses \verb:Symbolic mapping: by default.
See also figure \ref{fig:changekbd} and \ref{fig:vicekbd}.

\begin{figure}
\centering
\includegraphics[width=0.75\linewidth]{images/changekbd.png}
\caption{Changing keyboard layout}
\label{fig:changekbd}
\end{figure}

\begin{figure}
\centering
\includegraphics[width=\linewidth]{images/vicekbd.png}
\caption{C64 Keyboard cheatsheet}
\label{fig:vicekbd}
\end{figure}

If you need to type special characters from the BASIC examples that are not available on a QWERTY keyboard than you have to change the layout to \verb:Positional mapping:.
Many exercises can be typed in using \verb:Symbolic mapping:.

\subsection{Printing}

Now that we have learned how to write text and remove it, we can look at some BASIC programming.
Lets see how we can write a ``Hello World'' program.
You can print text using the \verb:PRINT: command.
Each BASIC command is executed after you press enter.
All text to be printed must be enclosed with double quotes, that is \verb:":.
One example is \verb:PRINT "HELLO":.

\problem Change the example to print \verb:HELLO WORLD:.

\verb:PRINT: can also join multiple portions of text using either \verb:,: or \verb:;:.
You can use these multiple times.

\problem What is the difference between \verb:PRINT"HELLO","OLLEH": and \verb:PRINT"HELLO";"OLLEH":?

\subsection{Calculations}

Using BASIC we can compute some expressions.
The basic math operators $+$,$-$,$*$, $/$, and $\uparrow$ (power) can be used in expressions.
Please note that all numbers are stored as floating point numbers.
For example, \verb:PRINT 2/3: prints $0.666666667$, rather than $0$ as some programmers would have expected.

If we want to perform integer division, this means `ignoring' the fractional part of the result,
we can use the function \verb:INT():.
For example, $\verb:PRINT:\ \verb:INT:(2/3)$ prints $0$.

As you may have noticed is that the operator precedence is the same as in mathematics.
Note that some calculators and computers from the same era did not have operator precedence at all!
You can also use braces to change the order of computation.

\subsection{Variables}

It would be nice if we could store the result of a computation in a variable so we don't have to recalculate each result.
Using variables we can also write interactive programs.
BASIC has numeric variables and a couple of other types (think about text or a list of numbers).
For now, we will only look at the numeric variables.

Back then, many programmers first wrote their program on tape before typing them in.
This is one of the many reasons why variables had very short and cryptic names and why the C64 only supports short variable names.
BASIC commands are illegal variable names. For example, \verb:PRINT: is an illegal variable name.
A variable may \emph{not be longer than two characters}.
If it is longer than two characters, all characters after the second are ignored.
For example, \verb:BANANA:, \verb:BA: and \verb:BAROMETER: are all considered to refer to the same variable!

Lets say we want to keep track of the number of apples we have.
We could do this using \verb:A=3: and we can print the number of apples with \verb:PRINT"APPELS",A:

If we want to add one apple we just have to add \verb:A=A+1:.
If you would print the number of apples you should see that we now have 4 apples.
You can navigate the cursor to the previous typed in \verb:PRINT: statement and press enter to re-execute the statement without having to type it all over again.

Now we know some fundamental theory about BASIC and lets take a look at some more serious programs.

\section{Programming using BASIC}

Until now we had to manually type each BASIC command.
While this works quite nicely for small experiments, it is inpractical for real programs.
Lets see how we can write our first BASIC program that can run by itself once we RUN it.

Each line of BASIC codoe has to be preceeded by a unique number.
It is similar to a linenumber.
The BASIC interpreter uses these linenumbers to determine in what order the program has to be executed.

Lets write our first BASIC program:

\begin{lstlisting}
10 PRINT"COMMODORE 64"
\end{lstlisting}

If you type RUN (or if you have a real C64, press the RUN/STOP key) it will start RUNning the program.
Congratulations! You have written your first real BASIC program!

This program is a little boring.
Lets change the program so it will keep printing the text over and over.
In BASIC you can use the \verb:GOTO: command to jump to a line.
Using \verb:GOTO: we can make a loop.
Lets add this line and RUN the program:

\begin{lstlisting}
20 GOTO 10
\end{lstlisting}

You can stop the program by pressing ESCAPE (\verb:STOP: on a real C=64).

\problem Add \verb:;: to line 10 and press enter. What happens if we RUN the program?

\problem Type \verb:LIST: and then \verb:LIST10:. What does this command do?

Whenever you want to write a new program it is a good idea to erase all previous code.
We can reset the emulator to accomplish this, but this is unpractical.
It is much easier to just type \verb:NEW: to start a NEW program.
Be careful with this command, because it does not ask for confirmation!

\subsection{Peeks 'n Pokes}

RUN the program \verb:BORDERKLEUR: located on the slides disk (\verb:roms/workshop/slides.d64:).

\problem What does the \verb:PEEK: commando do on line 20?

Because we have only 16 colors we ignore the upper 4 bits using \verb:AND15:.

\problem What does the \verb:POKE: commando do on line 50?

\problem What information does the magic address \verb:VIC+32: store?

\subsection{Input/Output}

You can read user input using the \verb:INPUT: command.
For example, enter the following program:

\begin{lstlisting}
10 INPUT A$
20 PRINT "YOU TYPED: "; A$
\end{lstlisting}

Typ and RUN the program.
You can enter some text that will be printed to the screen.

\problem Write a BASIC program that asks a username and prints it to the screen.

Now is a good moment to figure out how to save your program to floppydisk so you can load it when you need it.
See section 2 for a cheatsheet for save and load commands.

\subsection{Programmer's Reference Guide and Literature}

This workshop provides a glimpse of BASIC programming and shows how easy it is.
If you would like to write more programs you can start reading at chapter 3 of the \verb:User's Guide:.
If you would like more challenging stuff or want to learn more details about the Commodore 64 you can also look at the \verb:Programmer's Reference Guide:.
Both documents are included if you have downloaded the release version from the project's github page.

\section{Appendix}

\subsection{Commodore 64 programs, games and demo's}

There is a lot of software for the Commodore 64.
In fact, many software is still being written to this day!
The most famous and complete website for roms is \href{http://csdb.dk}{csdb}.

Another real well known, also for programming contests, is \href{https://www.pouet.net}{pouet}.
Lots of Commodore 64 software can be found on that website.
For example, the world's biggest and most recent programming contest for the C64 is X.
\href{https://www.pouet.net/party.php?which=50&when=2016}{Here} you can find an overview of all software that has been released at this party.

\subsection{easy6502}

If you thought the BASIC exercises were boring and are interested in programming in assembly, you can try out \href{https://skilldrick.github.io/easy6502/}{this introduction}.

\subsection{KickAssembler}

If you have completed all exercises and need more challenging problems you can try to start coding in assembly using \href{http://theweb.dk/KickAssembler/Main.html#frontpage}{this assembler}.
A couple of examples and very good documentation is included as well.

These sources are also very important for coding in assembly:

\begin{tabular}{l|l|l}
Link & Domain & Notes \\
\href{http://codebase64.org}{codebase64} & The Programmer's wiki & Tips and examples \\
\href{http://sta.c64.org/cbm64mem.html}{cbm64mem.html} & C64 Memory map & Every memory address with notes \\
\href{https://www.c64-wiki.com}{c64-wiki} & Wikipedia for C64 & Everything about Commodore \\
\end{tabular}

\subsection{Demoscene}

This workshop shows a glimpse about the demoscene that originated primarily by computer clubs of the first Personal Computers.
If you are interested in the demoscene or just want to learn what it really is, then we recommend to take a look at the following:

\begin{tabular}{l|l|l}
Link & Domain & Notes \\
\href{https://www.youtube.com/watch?v=AdTANxS-LHg}{YT: Polderpioniers - J. Tel} & First game music composers & Dutch only! \\
\href{https://www.youtube.com/watch?v=5MexnBunH_g}{YT: The Art of Algorithms} & History of Demoscene & Documentary \\
\href{https://www.youtube.com/watch?v=O-1zEo7DD8w}{YT: Rev2017 Shader Finals} & Programming Contest & Stream \\
\href{https://en.wikipedia.org/wiki/Demoscene}{Wikipedia: Demoscene} & Demoscene crash course & Wikipedia's view \\
\href{https://id.scene.org/}{Scene portal} & Demoscene portal & Account for demoscene sites \\
\href{https://www.youtube.com/watch?v=rFv7mHTf0nA}{Future Crew - Second Reality} & Winner Assembly 1993 & Very famous '90s demo \\
\end{tabular}

\end{document}
