\documentclass{article}

\usepackage{amsmath}
\usepackage{hyperref}
\usepackage{listings}
\usepackage{graphicx}
\usepackage{eurosym}

\title{Exercises for workshop Commodore 64}
\author{Folkert van Verseveld}

\newcounter{problem}
\newcounter{solution}

\newcommand\problem{%
  \stepcounter{problem}%
  \textbf{\theproblem.}~%
  \setcounter{solution}{0}%
}
\parindent 0in
\parskip 1em

\begin{document}
\maketitle

\section{Introduction}

This bundle contains a set of exercises with increasing difficulty.
The difficult exercises may be skipped.
See the bundle of answers for solutions of some exercises.
Some exercises (or variations) will also be discussed during the workshop.

\subsection{VICE setup}
This workshop uses the Versatile Commodore Emulator (VICE).
You can download the emulator \href{http://vice-emu.sourceforge.net/}{from this site}.
Depending on your operating system, some additional setup may be required as will be discussed in the next section.

\subsubsection{Linux}
The easiest method is to use your package manager to install VICE. Example \verb:sudo apt install vice: on Debian, Ubuntu and derivatives.
If this does not work, but you have Wine installed, you can also use the Windows recipe.
The die-hards may also try to compile the \href{http://sourceforge.net/projects/vice-emu/files/releases/vice-3.1.tar.gz/download}{source code}.

\subsubsection{Windows}
Download the \href{http://sourceforge.net/projects/vice-emu/files/releases/binaries/windows/WinVICE-3.1-x64.7z/download}{windows version}.
Other versions can be downloaded from the \href{http://vice-emu.sourceforge.net/windows.html}{download section}.
You need \href{http://www.7-zip.org/}{7Zip} in order to extract the archive.
Extract all files and run \verb:x64.exe:.

\subsubsection{Mac}
Download the \href{http://sourceforge.net/projects/vice-emu/files/releases/binaries/macosx/vice-macosx-sdl-x86_64-10.12-3.1.dmg/download}{mac version}.
You need to extract the .dmg in order to use the emulator.
Please note that you have to run the emulator from a terminal because the menu bar is missing.

Open a terminal and navigate to the directory where you just extracted the emulator and run the emulator. For example, if you have extracted the emulator on your desktop:
\begin{lstlisting}
cd /Users/name/Desktop/vice-emu/
./x64.app/Contents/MacOS/x64
\end{lstlisting}

Replace \verb:name: with your user name.

\textbf{Note: If you want to save and load programs, you should read the `Saving/Loading programs' section first!}

\subsection{Tips and Tricks}

If you are stuck on an exercise you can try to look at the hints document.
Please don't hesitate to try some things before giving up.
Some exercises are designed with this in mind.

If the emulator is haning you can try to reset it using \verb:File->Reset->Soft: to restart the C64.
If this does not work, you can try to hard reset it using \verb:File->Reset->Hard: or restart the emulator if that does not work either.

\section{Introduction BASIC}

The Commodore 64 understands machine code and the high-level programming language BASIC.
If you run the VICE emulator you will see a similar window to figure \ref{fig:vice}.

\begin{figure}
\centering
\includegraphics[width=0.5\linewidth]{images/boot.png}
\caption{VICE emulator}
\label{fig:vice}
\end{figure}

The default screens tells us that we are using BASIC version 2 and it has 64KiB of RAM leaving 38911 bytes for BASIC.
This means that our BASIC program cannot exceed 38.9KB!

You may have noticed that the default screen only uses upper case letters.
This is because many computer designers at that time deemed it unnessecary to support both upper and lower case letters at the same time!
It is possible to use both at the same time, but we are not going to use this.

\subsection{Typing}

\problem Try to type \verb:hello: and \verb:HELLO:. What happens if you hold shift while trying to type something?

You can use the spacebar to place a space character on the screen or to `erase' the current character.

\problem Try to change different portions of text on the screen and try to remove some text by overwriting them with spaces.

\problem What happens if we place the cursor to the far right and press the right arrow key?
What happens if we place the cursor to the far left and press the left arrow key?
And what happens if we place the cursor at the bottom and press the down arrow key?

\subsection{Printing}

Now that we have learned how to write text and remove it, we can look at some BASIC programming.
Lets see how we can write a ``Hello World'' program.
You can print text using the \verb:PRINT: command.
Each BASIC command is executed after you press enter.
All text to be printed must be enclosed with double quotes, that is \verb:":.
One example is \verb:PRINT "HELLO":.

\problem Change the example to print \verb:HELLO WORLD:.

\verb:PRINT: can also join multiple portions of text using either \verb:,: or \verb:;:. You can use these multiple times.

\problem What is the difference between \verb:PRINT"HELLO","OLLEH": and \verb:PRINT"HELLO";"OLLEH":?

\subsection{Calculations}

Using BASIC we can compute some expressions.
The basic math operators $+$,$-$,$*$ and $/$ can be used in expressions.
But if we would type \verb:12 + 12: we don't get any result.
This is because we did not instruct the C64 to also print the result.

\problem Write a BASIC command that computes $12 + 12$ and prints the result.

\problem Write a BASIC command that computes $2 - 3 * 4$ and prints the result.

The only remaining operator is $/$.
BASIC uses floating point numbers by default.
This may suprise some programmers who would probably expect integers to be the default type.
For example, $\verb:PRINT:\ 2/3$ does not print $0$, but $0.666666667$.

If we want to perform integer division, this means `ignoring' the fractional part of the result, we can use the function \verb:INT():.
For example, $\verb:PRINT:\ \verb:INT:(2/3)$ prints $0$.

As you may have noticed is that the operator precedence is the same as in mathematics.
Note that some calculators and computers from the same era did not have operator precedence at all!
You can also use braces to change the order of computation.

\problem Write a BASIC command that computes $3*3*3*3$ and prints the result.
Can we replace the multiple multiplications with something else?

\subsection{Variables}

It would be nice if we could store the result of a computation in a variable so we don't have to recalculate each result.
Using variables we can also write interactive programs.
BASIC has numeric variables and a couple of other types (think about text or a list of numbers).
For now, we will only look at the numeric variables.

Back then, many programmers first wrote their program on tape before typing them in.
This is one of the many reasons why variables had very short and cryptic names and why the C64 only supports short variable names.
BASIC commands are illegal variable names.
For example, \verb:PRINT: is an illegal variable name.
A variable may \emph{not be longer than two characters}.
If it is longer than two characters, all characters after the second are ignored.
For example, \verb:BANANA:, \verb:BA: and \verb:BAROMETER: are all considered to refer to the same variable!

Lets say we want to keep track of the number of apples we have.
We could do this using \verb:A=3: and we can print the number of apples with \verb:PRINT"APPELS",A:

If we want to add one apple we just have to add \verb:A=A+1:.
If you would print the number of apples you should see that we now have 4 apples.
You can navigate the cursor to the previous typed in \verb:PRINT: statement and press enter to re-execute the statement without having to type it all over again.

Now we know some fundamental theory about BASIC and lets take a look at some more serious programs.

\section{Programming using BASIC}

Until now we had to manually type each BASIC command.
While this works quite nicely for small experiments, it is inpractical for real programs.
Lets see how we can write our first BASIC program that can run by itself once we RUN it.

Each line of BASIC codoe has to be preceeded by a unique number.
It is similar to a linenumber.
The BASIC interpreter uses these linenumbers to determine in what order the program has to be executed.

Lets write our first BASIC program:

\begin{lstlisting}
10 PRINT"COMMODORE 64"
\end{lstlisting}

If you type RUN (or if you have a real C64, press the RUN/STOP key) it will start RUNning the program.
Congratulations! You have written your first real BASIC program!

This program is a little boring.
Lets change the program so it will keep printing the text over and over.
In BASIC you can use the \verb:GOTO: command to jump to a line.
Using \verb:GOTO: we can make a loop.
Lets add this line and RUN the program:

\begin{lstlisting}
20 GOTO 10
\end{lstlisting}

You can stop the program by pressing ESCAPE (\verb:STOP: on a real C=64).

\problem Add \verb:;: to line 10 and press enter. What happens if we RUN the program?

\problem Type \verb:LIST: and then \verb:LIST10:. What does this command do?

Whenever you want to write a new program it is a good idea to erase all previous code.
We can reset the emulator to accomplish this, but this is unpractical.
It is much easier to just type \verb:NEW: to start a NEW program.
Be careful with this command, because it does not ask for confirmation!

\problem Folkert has bought 6 apples for \$0.32 a piece and 4 banana's for \$0.22 a piece and he would like to determine the total price for all apples, all banana's and all apples and all banana's and print these to the screen.
Write such a program that computes and prints the overal cost. Don't forget to type \verb:NEW: before you start writing the program!

If you want to save the result, either read the section Saving/Loading programs or write your solution in your favorite text editor or even on paper.

\subsection{Peeks 'n Pokes}

Restart the emulator and type a `A' in the upper-left corner of the screen.
Then go to an empty line and type \verb:PEEK(1024):.

\problem Replace the `A' in the upper-left corner of the screen with `B' and rerun \verb:PEEK(1024):.
What does this command do? Try a couple of other values to figure out what the printed number represents.

Now type in \verb:POKE1024,1: and hit enter. Then replace \verb:,1: with \verb:,2: and try some other values as well.

\problem What does this command do?

\subsection{Input/Output}

You can read user input using the \verb:INPUT: command.
For example, enter the following program:

\begin{lstlisting}
10 INPUT A$
20 PRINT "YOU TYPED: "; A$
\end{lstlisting}

If you RUN the program, you can enter some text that will be printed to the screen.

\problem Write a BASIC program that asks a username and prints it to the screen.

\section{Saving/Loading programs}

\subsection{Creating and attaching a floppy disk}

You can use c1541 (the floppy drive tools program) to create a disk.
For example, if we want to create a disk named `music' we can use this on linux:

\begin{lstlisting}
c1541 -format diskname,id d64 music.d64
\end{lstlisting}

Or this on windows:

\begin{lstlisting}
c1541.exe -format diskname,id d64 music.d64
\end{lstlisting}

Or this on a mac:

% XXX make sure this is correct. It may be ./x64.app/.../c1541 but I'm not sure
\begin{lstlisting}
./c1541.app/Contents/MacOS/c1541 -format diskname,id d64 music.d64
\end{lstlisting}

Replace `music' with whatever you want to name the disk.

On windows and Linux, you can attach the drive even after you have started the emulator using \verb:File->Attach disk image->Unit #8: or using \verb:ALT+8:.

\subsection{Saving/Loading programs on Mac}

In order to be able to save and load programs, you have to attach a floppy disk \emph{before} starting the emulator like this:

\begin{lstlisting}
./x64.app/Contents/MacOS/x64 path/to/your/floppy.d64
\end{lstlisting}

\subsection{Saving programs}

After you have typed in a BASIC program and attached a floppy disk to drive 8 you can save it by typing \verb:SAVE"NAME",8:.
If you want to overwrite a program you have to do the following in multiple steps.
First, you have to remove the program like this:

\begin{lstlisting}
OPEN1,8,15,"I":CLOSE1
OPEN1,8,15,"S:NAME":CLOSE1
\end{lstlisting}

Replace \verb:NAME: with the program you want to overwrite.
After you have removed the program, you can resave it using \verb:SAVE"NAME",8:.

\subsection{Loading programs}

After you have attached the floppy disk you want to read from to drive 8 you can load a listing of the drive using \verb:LOAD"$",8: and type \verb:LIST: to see the list of files.
You can load a file using \verb:LOAD"NAME",8:.

If you want to load the first program from disk you can use the short-hand \verb:LOAD"*",8,1":.

\end{document}
