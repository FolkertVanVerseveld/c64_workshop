\documentclass{article}

\usepackage[dutch]{babel}
\usepackage{amsmath}
\usepackage{hyperref}
\usepackage{listings}
\usepackage{graphicx}

\title{Tips bij Opgaven voor workshop Commodore 64}
\author{Folkert van Verseveld}

\newcounter{problem}

\newcommand\problem{%
  \stepcounter{problem}%
  \textbf{\theproblem.}~%
}
\parindent 0in
\parskip 1em

\begin{document}
\maketitle

\section{Inleiding}

In dit document vindt u tips om de opgaven op te kunnen lossen.

\section{Introductie BASIC}

\problem Typ wat tekst in zonder de shift ingedrukt te houden en daarna de shift ingedrukt houden.

\problem Gebruik de pijltoetsen om de cursor te bewegen en probeer tekst te typen.

\problem Gebruik de pijltoetsen om de cursor te bewegen naar elke hoek van het scherm en probeer vervolgens de cursor buiten het scherm te krijgen.

\problem Gebruik het \verb:PRINT: commando.

\problem Typ een voorbeeld wat \verb:,: en \verb:;: gebruikt.

\problem Gebruik het \verb:PRINT: commando.

\problem Gebruik het \verb:PRINT: commando.

\problem Kunnen we ook de machtsverheffingsoperator gebruiken?

\section{Programmeren in BASIC}

\problem Gebruik uw antwoord bij vraag 5.

\problem Typ de gevraagde commando's in en kijk wat er gebeurt.

\problem Het aantal appels en bananen kunnen we elk in een variabele stoppen net als de prijs. Het printen van de resultaten is een kwestie van uitrekenen en uitwerken.
Het is makkelijker om regelnummers met stappen van 5 of 10 op te volgen, want hierdoor kunt u er later nog regels tussen stoppen.

\end{document}
