%
% Name: Folkert van Verseveld
% UvAnetID: 11839341
% Contact:
%   @: folkert.van.verseveld@gmail.com
%   @: folkert.vanverseveld@students.uva.nl
%

\documentclass[aspectratio=43]{uva-inf-presentation}
\usepackage{multicol}
\usepackage{amssymb}
\usepackage{listings}
\usepackage{subcaption}

\title{Programming in Restricted Enviroments}
% Easter egg: Commodore Business Machines logo
\course{C=}
\assignment{Just 4 Fun}
\assignmenttype{Hacking}
\authors{Folkert van Verseveld}
\uvanetids{11839341}
% XXX Tutor, mentor and docent are not necessary?
\tutor{Teun Mathijssen}
\mentor{}
\docent{Robert van Wijk}
\group{Assembly A2}

\begin{document}

\begin{titelframe}
\titlepage

\end{titelframe}

\begin{frame}
\frametitle{Table of Contents}
\tableofcontents
\end{frame}


\section{Introduction}
\section{Programming Environment}
\section{Intermezzo's and Exercises}

\section{Conclusions}

%----------------------------------------------

\begin{frame}
\frametitle{Restricted Environments}

\begin{itemize}
\item Very first `computers'
\item Primarily built for Army, Universities or Government
\item Spanning several rooms using enormous amount of electricity
\end{itemize}

\end{frame}

%----------------------------------------------

\begin{frame}
\frametitle{Applications of very first `computers'}

\begin{itemize}
\item Calculation tables (sin, cos)
\item Administration (statistics, taxes)
\end{itemize}

\end{frame}

%----------------------------------------------

\begin{frame}
\frametitle{Calculation tables}

\begin{figure}
\includegraphics[width=0.75\linewidth]{images/rekentabellen.jpg}
\end{figure}

\end{frame}

%----------------------------------------------

\begin{frame}
\frametitle{Military Applications}

\begin{itemize}
\item Artillery (projectiles)
\item Cryptography (enigma)
\item OXO
\end{itemize}

\end{frame}

%----------------------------------------------

\begin{frame}
\frametitle{Bombe}

\begin{figure}
\includegraphics[width=0.75\linewidth]{images/bombe.jpg}
\end{figure}

\end{frame}

%----------------------------------------------

\begin{frame}
\frametitle{EDSAC}

\begin{figure}
\includegraphics[width=0.75\linewidth]{images/edsac.jpg}
\end{figure}

\end{frame}

%----------------------------------------------

\begin{frame}
\frametitle{EDSAC I/O}

\begin{figure}
\includegraphics[width=0.75\linewidth]{images/edsac_crts.jpg}
\end{figure}

\end{frame}

%----------------------------------------------

\begin{frame}
\frametitle{OXO}

\begin{figure}
\includegraphics[width=0.75\linewidth]{images/oxo.jpg}
\end{figure}
% http://www.dcs.warwick.ac.uk/~edsac/

\end{frame}

%----------------------------------------------

\begin{frame}
\frametitle{Mainframes}

\begin{itemize}
\item Multi-user environment
\item Batch
\item Reserve computing time
\item Not for practical use
\item Very expensive: \$100,000+ / \$1,000,000+
\item Not user friendly
\end{itemize}

\end{frame}

%----------------------------------------------

\begin{frame}
\frametitle{Mini-mainframes}

\begin{figure}
\includegraphics[width=0.45\linewidth]{images/pdp8.jpg}
\end{figure}

\end{frame}

%----------------------------------------------

\begin{frame}
\frametitle{Mini-mainframes}

\begin{itemize}
\item Expensive, but less than millions: \$10000+
\item Acceptable performance
\item `Average' customers
\end{itemize}

\end{frame}

%----------------------------------------------

\begin{frame}
\frametitle{Moore's law}

\begin{itemize}
\item Calculators are getting faster and cheaper
\item The cheaper, the more applications
\end{itemize}

\end{frame}

%----------------------------------------------

\begin{frame}
\frametitle{Introduction of Personal Computer}

\begin{itemize}
\item Mini mainframe
\item `Cheap'
\item `User friendly'
\item Real PCs around 1977-1982
\end{itemize}

\end{frame}

%----------------------------------------------

\begin{frame}
\frametitle{First Personal Computers}

\begin{figure}
\includegraphics[width=0.6\linewidth]{images/appleii.jpg}
\end{figure}

\end{frame}

%----------------------------------------------

\begin{frame}
\frametitle{First Personal Computers}

\begin{figure}
\includegraphics[width=0.75\linewidth]{images/trs80.jpg}
\end{figure}

\end{frame}

%----------------------------------------------

\begin{frame}
\frametitle{First Personal Computers}

\begin{figure}
\includegraphics[width=0.6\linewidth]{images/pet.jpg}
\end{figure}

\end{frame}

%----------------------------------------------

\begin{frame}
\frametitle{First Personal Computers}

\begin{figure}
\includegraphics[width=0.6\linewidth]{images/vic20.jpg}
\end{figure}

\end{frame}

%----------------------------------------------

\begin{frame}
\frametitle{First Personal Computers}

\begin{figure}
\includegraphics[width=0.6\linewidth]{images/ibm5150.jpg}
\end{figure}

\end{frame}

%----------------------------------------------

\begin{frame}
\frametitle{First Personal Computers}

\begin{figure}
\includegraphics[width=0.6\linewidth]{images/zxspectrum.jpg}
\end{figure}

\end{frame}

%----------------------------------------------

\begin{frame}
\frametitle{First Game Consoles}

\begin{figure}
\includegraphics[width=0.75\linewidth]{images/vcs.jpg}
\end{figure}

\end{frame}

%----------------------------------------------

\begin{frame}
\frametitle{First Game Consoles}

\begin{figure}
\includegraphics[width=0.75\linewidth]{images/nes.jpg}
\end{figure}

\end{frame}

%----------------------------------------------

\begin{frame}
\frametitle{Applications of PCs}

\begin{itemize}
\item Office related (administration, spreadsheets, \dots)
\item Aimed at broader consumer audience
\item (Games)
\end{itemize}

\end{frame}

%----------------------------------------------

\begin{frame}
\frametitle{Personal Computers by Comparison}

\begin{tabular}{|l|l|l|l|l|}
\hline Year & Model & Introd. & Converted & Now \\ \hline
1977 & PET & \$795 & \$3150 & \$600-\$800 \\
1980 & VIC-20 & \$299.95 & \$874 & \$30-\$110 \\
1982 & C64 & \$595 & \$1480 & \$10-\$50 \\ \hline
1977 & Apple ][ & \$1296 & \$5130 & \$200-\$750 \\
1981 & IBM PC & \$1565 & \$4130 & \$150-\$1100 \\
1982 & ZX Spec. & \pounds 175 & \pounds 567 & \pounds 80-\pounds 120 \\ \hline
1983 & NES & \$179 & \$433 & \$25-\$120 \\ \hline
\end{tabular}

\end{frame}

%----------------------------------------------

\begin{frame}
\frametitle{What about Hard Disks?}

\begin{itemize}
\item Don't need them
\item Too expensive (+\$500/+\$2000)
\end{itemize}

\end{frame}

%----------------------------------------------

\begin{frame}
\frametitle{Data Storage}

\begin{itemize}
\item Programs in RAM, firmware in ROM
\item Power loss: program gone!
\end{itemize}

\end{frame}

%----------------------------------------------

\begin{frame}
\frametitle{Tapes!}

\begin{itemize}
\item Sequential magnetic memory
\item Couple of MBs
\item Single/Double density
\end{itemize}

\end{frame}

%----------------------------------------------

\begin{frame}
\frametitle{Searching for Program}

\begin{itemize}
\item Clumsy
\item Manually seek forward and rewind tape
\item Analogue file table
\end{itemize}

\end{frame}

%----------------------------------------------

\begin{frame}
\frametitle{Floppies!}

\begin{itemize}
\item Real floppy
\item 8 inch, 5.25 inch
\item SD/DD
\item 300-600 RPM (5-10 revs/sec)
\item 0.1MB-2.0MB
\end{itemize}

\end{frame}

%----------------------------------------------

\begin{frame}
\frametitle{Meet the most Successful PC of all time}

\begin{figure}
\includegraphics[width=0.75\linewidth]{images/desk.jpg}
\end{figure}

\end{frame}

%----------------------------------------------

\begin{frame}
\frametitle{The Only Real One - Breadbin}

\begin{figure}
\includegraphics[width=0.6\linewidth]{images/c64.jpg}
\end{figure}

\end{frame}

%----------------------------------------------

\begin{frame}
\frametitle{The Only Real One - C64C}

\begin{figure}
\includegraphics[width=0.6\linewidth]{images/c64c.jpg}
\end{figure}

\end{frame}

%----------------------------------------------

\begin{frame}
\frametitle{Modern Version (emulator)}

\begin{figure}
\includegraphics[width=0.5\linewidth]{images/boot.png}
\end{figure}

\end{frame}

%----------------------------------------------

\begin{frame}
\frametitle{Hardware Specifications}

\begin{itemize}
\item Introduced in 1982
\item 64 kilobytes of RAM
\item MOS 6510 clocked at 0.985MHz (PAL)
\item 2 joystick ports
\item cartridge port (games!), serial I/O, user port
\item 16 colors
\end{itemize}

\end{frame}

%----------------------------------------------

\begin{frame}
\frametitle{CPU and Memory by Comparison}

\begin{tabular}{|l|l|l|l|l|}
\hline Model & RAM (KB) & CPU & MHz \\ \hline
PET & 8 & MOS 6502 & 1 \\
VIC-20 & 5 & MOS 6502 & 1.11 \\
C64 & 64 & MOS 6510 & 0.985 \\ \hline
Apple ][ & 4 & MOS 6502 & 1 \\
IBM PC & 16 & i8086 & 4.04 \\
ZX Spec. & 16 & Z80 & 3.5 \\ \hline
NES & 2 & Ricoh 2A03 & 1.6 \\ \hline
\end{tabular}

\end{frame}

%----------------------------------------------

\begin{frame}
\frametitle{MOS 6502}

\begin{itemize}
\item Really cheap and successful
\item 6502, 6510, ricoh 2a03, \dots
\item PET, VIC-20, C64, NES, \dots
\item Embedded systems
\end{itemize}

\end{frame}

%----------------------------------------------

\begin{frame}
\frametitle{MOS 6502}

\begin{figure}
\includegraphics[width=\linewidth]{images/terminator.jpg}
\end{figure}

\end{frame}

%----------------------------------------------

\begin{frame}
\frametitle{MOS 6502}

\begin{figure}
\includegraphics[width=0.6\linewidth]{images/bender.jpeg}
\end{figure}

\end{frame}

%----------------------------------------------

\begin{frame}
\frametitle{What can you do on a C64?}

\begin{itemize}
\item Administration (spreadsheets, word, \dots)
\item Scanning, printing (with dox matrix printer)
\item Games!
\item Internet (using 50Kbps modem)
\item Hardware upgrades
\end{itemize}

\end{frame}

%----------------------------------------------

\begin{frame}
\frametitle{Why Bother with Vintage PCs?}

\begin{itemize}
\item Great Challenge
\item Simple design, easy to understand
\item Literature
\item Programming Contests (yes, more than 30 years!)
\end{itemize}

\end{frame}

%----------------------------------------------

\begin{frame}
\frametitle{C64 is still in use at this garage!!}

\begin{figure}
\includegraphics[width=0.75\linewidth]{images/c64c_garage.jpg}
\end{figure}

\end{frame}

%----------------------------------------------

\begin{frame}
\frametitle{Supported Programming Languages}

\begin{itemize}
\item Microsoft BASIC
\item Machine code (assembly)
\end{itemize}

\end{frame}

%----------------------------------------------

\begin{frame}
\frametitle{How did we write Programs back then?}

\begin{itemize}
\item No internet (BBS after some time)
\item Literature (magazines, books, \dots)
\item Mouth-to-mouth (meetings, competitions, parties, \dots)
\end{itemize}

\end{frame}

%----------------------------------------------

\begin{frame}
\frametitle{What do you get if you bought a C64?}

\begin{itemize}
\item User's Guide approx. 300 pages
\item Programmer's reference guide for BASIS and Assembly
\item Reference to programs, clubs, magazines
\end{itemize}

\end{frame}

%----------------------------------------------

\begin{frame}
\frametitle{User's Guide}

\begin{itemize}
\item Connecting peripherals
\item Configurations and Commands
\item BASIC programming (approx. 60 commands)
\item Examples
\item Tips and tricks
\end{itemize}

\end{frame}

%----------------------------------------------

\begin{frame}
\frametitle{Programmer's Reference Guide}

\begin{itemize}
\item Memory layout
\item Assembly instruction set (151 documented)
\item Graphics modes
\item Sound effects
\item Hardware expansions
\end{itemize}

\end{frame}

%----------------------------------------------

\begin{frame}
\frametitle{Literature}

\begin{figure}
\includegraphics[width=0.75\linewidth]{images/c64schema.jpg}
\end{figure}

\end{frame}

%----------------------------------------------

\begin{frame}
\frametitle{Literature}

\begin{figure}
\includegraphics[width=0.75\linewidth]{images/c64schema2.jpg}
\end{figure}

\end{frame}

%----------------------------------------------

\begin{frame}[fragile]{Simple Program - BASIC}

\begin{lstlisting}
10 PRINT "HALLO VIA"
20 GOTO 10
\end{lstlisting}

\end{frame}

%----------------------------------------------

\begin{frame}[fragile]{Simple Program - BASIC}

\begin{lstlisting}
10 POKE 1024,8
20 POKE 1025,5
30 POKE 1026,25
\end{lstlisting}

\end{frame}

%----------------------------------------------

\begin{frame}[fragile]{Simple Program - assembly}

\begin{lstlisting}
lda #$08
sta $400
lda #$05
sta $401
lda #$19
sta $402
rts
\end{lstlisting}

\end{frame}

%----------------------------------------------

\begin{frame}{Assembly}

\begin{itemize}
\item Simple, but steep learning curve
\item 151 official instructions
\item Small examples
\end{itemize}

\end{frame}

%----------------------------------------------

\begin{frame}{Assembly}

\begin{tabular}{l|l|l}
lda \#\$10 & A = 16 & Write \$10 to A \\
sta \$400 & mem[1000] = A & Write A to \$400 \\
rts & return; & End of procedure \\
\end{tabular}

\end{frame}

%----------------------------------------------

\begin{frame}{Simple Program}

\begin{itemize}
\item Go to: https://skilldrick.github.io/easy6502/
\item Look at the examples.
\item Try to create your own program!
\end{itemize}


\end{frame}

%----------------------------------------------

\begin{frame}
\frametitle{Audio}

\begin{itemize}
\item Analogue or Digital
\end{itemize}

\begin{figure}
\includegraphics[width=0.6\linewidth]{images/waveforms.png}
\end{figure}

\end{frame}

%----------------------------------------------

\begin{frame}
\frametitle{Analoog Audio}

\begin{itemize}
\item Not-electronic
\item Visualization using Oscilloscope
\end{itemize}

\end{frame}

%----------------------------------------------

\begin{frame}
\frametitle{Digital Audio}

\begin{itemize}
\item Approximation of analogue
\item Amplitude jumps in levels
\item Clearly visible with oscilloscope on old PCs
\end{itemize}

\end{frame}

%----------------------------------------------

\begin{frame}
\frametitle[fragile]{Digital Audio}

\begin{figure}
	\begin{subfigure}[b]{0.4\textwidth}
		\includegraphics[width=\linewidth]{images/waveforms.png}
	\end{subfigure}
	\begin{subfigure}[b]{0.5\textwidth}
		\includegraphics[width=\linewidth]{images/rc3nes.png}
	\end{subfigure}
\end{figure}

\end{frame}

%----------------------------------------------

\begin{frame}
\frametitle{Digitaal Audio - Speaker}

\begin{itemize}
\item Squaker
\item Very rough
\item On or off
\item Frequency
\end{itemize}

\end{frame}

%----------------------------------------------

\begin{frame}
\frametitle{Digital Audio - Speaker Examples}

\begin{itemize}
\item Loop
\item Single
\item Arpeggio
\end{itemize}

\end{frame}

%----------------------------------------------

\begin{frame}
\frametitle{Digital Audio - NES}

\begin{tabular}{|l|l|l|l|}
\hline \# & Type & Volumes & Frequencies \\ \hline
2 & Pulse wave & 16 & 54Hz-28kHz \\
1 & White noise & 16 & 29.3Hz-47kHz \\
1 & Triangle wave & 1 & 27Hz-56kHz \\
1 & Delta PCM & 16 & 4kHz-33kHz \\ \hline
\end{tabular}

\end{frame}

%----------------------------------------------

\begin{frame}
\frametitle{Digital Audio - NES}

\begin{itemize}
\item 5 channels
\item 8 bit for pulse, noise en triangle
\item 1 bit for DPCM
\item hardware pitch bending
\end{itemize}

\end{frame}

%----------------------------------------------

\begin{frame}
\frametitle{Digital Audio - NES example}

\begin{itemize}
\item The Legend of Zelda theme
\end{itemize}

\end{frame}

%----------------------------------------------

\begin{frame}
\frametitle{Digital Audio}

\begin{figure}
\includegraphics[width=0.9\linewidth]{images/rc3nes.png}
\end{figure}

\end{frame}

%----------------------------------------------

\begin{frame}
\frametitle{C64: SID chip}

\begin{itemize}
\item World's first audio synthesizer
\item Analogue \emph{and} Digitaal
\item Designer received different awards
\item 6581 and 8580
\item Timbaland plagiarism
\end{itemize}

\end{frame}

%----------------------------------------------

\begin{frame}
\frametitle{Audio Hardware by Comparison}

\begin{tabular}{|l|l|l|l|}
\hline Model & Hardware & Type & Voices \\ \hline
PET & CPU PIO & 1-bit beeper & 0 \\
VIC-20 & VIC I & 8-bit digital & 4 \\
C64 & SID & A/D 8-bit synthesizer & 3 \\ \hline
Apple ][ & Speaker & 1-bit beeper & 0 \\
IBM PC & Speaker & 1-bit beeper & 0 \\
NES & RP2A03 & 8-bit digital & 5 \\ \hline
\end{tabular}

\end{frame}

%----------------------------------------------

\begin{frame}
\frametitle{Audio}

\begin{itemize}
\item 3 channels (voices)
\item ADSR envelope
\end{itemize}

\end{frame}

%----------------------------------------------

\begin{frame}
\frametitle{ADSR}

\begin{figure}
\includegraphics[width=0.8\linewidth]{images/adsr.png}
\end{figure}

\end{frame}

%----------------------------------------------

\begin{frame}
\frametitle{Waveforms}

\begin{itemize}
\item Sawtooth
\item Square
\item Sinusoid
\item Triangle
\item Noise
\end{itemize}

\end{frame}

%----------------------------------------------

\begin{frame}
\frametitle{Waveforms}

\begin{figure}
\includegraphics[width=0.75\linewidth]{images/waveforms.png}
\end{figure}

\end{frame}

%----------------------------------------------

\begin{frame}
\frametitle{SID - Examples}

\begin{itemize}
\item Driller
\item Commando
\end{itemize}

\end{frame}

%----------------------------------------------

\begin{frame}
\frametitle{Intermezzo}

\begin{block}{Muziek}
Guess which music belongs to which machine!
\end{block}

\begin{itemize}
\item IBM PC
\item NES
\item C64
\end{itemize}

\end{frame}

%----------------------------------------------

\begin{frame}
\frametitle{Intermezzo - Examples}

\begin{itemize}
\item Grand Prix Circuit (PC, C64)
\item Castlevania (PC, NES, C64)
\end{itemize}

\end{frame}

%----------------------------------------------

\begin{frame}
\frametitle{Intermezzo}

\begin{itemize}
\item IBM PC
\item NES
\item C64
\end{itemize}

\end{frame}

%----------------------------------------------

\begin{frame}
\frametitle{SID hacking}

\begin{itemize}
\item Click/pop-bug in 6581
\item 4-bit PCM digital
\end{itemize}

\end{frame}

%----------------------------------------------

\begin{frame}
\frametitle{Graphics}

\begin{figure}
\includegraphics[width=0.5\linewidth]{images/palette.png}
\end{figure}

\end{frame}

%----------------------------------------------

\begin{frame}
\frametitle{Graphics}

\begin{itemize}
\item 320x200 pixels
\item 16 color palette
\item At most 2 or 4 colors per 8x8 block.
\end{itemize}

\end{frame}

%----------------------------------------------

\begin{frame}
\frametitle{Video Hardware by Comparison}

\begin{tabular}{|l|l|l|}
\hline Model & Video & Colors \\
PET & ??? & 1 \\
VIC-20 & VIC I & 4-16??? \\
C64 & VIC II & 16 \\
Apple ][ & ??? & 1-4 \\
IBM PC & mono of CGA & 1-4 \\
ZX Spec. & geen (CPU) & 8 \\ \hline
NES & Ricoh 2C02 PPU & 52 \\ \hline
\end{tabular}

\end{frame}

\begin{frame}
\frametitle{Graphic modes}

\begin{itemize}
\item Standard Character (text) Mode
\item Multicolor Character (text) Mode
\item Standard Bitmap Mode
\item Multicolor Bitmap Mode
\item Extended Background Color Mode
\end{itemize}

\end{frame}

%----------------------------------------------

\begin{frame}
\frametitle{Multicolor Character Mode}

\begin{figure}
\includegraphics[width=0.5\linewidth]{images/gary.png}
\end{figure}

\end{frame}

%----------------------------------------------

\begin{frame}
\frametitle{Standard Bitmap}

\begin{figure}
\includegraphics[width=0.5\linewidth]{images/sbm_obey.png}
\end{figure}

\begin{center}
Source: Archmage
\end{center}

\end{frame}

%----------------------------------------------

\begin{frame}
\frametitle{Standard Bitmap}

\begin{figure}
\includegraphics[width=0.5\linewidth]{images/sbm_tuksu.png}
\end{figure}

\begin{center}
Source: Duce
\end{center}

\end{frame}

%----------------------------------------------

\begin{frame}
\frametitle{Programming Contests}

\begin{itemize}
\item X (biggest C64 party in the Netherlands!)
\item Revision (largest demoparty in the world!)
\item Nordlicht
\item Under Construction
\item \dots
\end{itemize}

\end{frame}

%----------------------------------------------

\begin{frame}
\frametitle{X}

\begin{figure}
\includegraphics[width=0.8\linewidth]{images/x.jpg}
\end{figure}

\end{frame}

%----------------------------------------------

\begin{frame}
\frametitle{X}

\begin{itemize}
\item Since 1995
\item Since 2004 once per 2 years
\item Last year approx. 400 visitors
\end{itemize}

\end{frame}

%----------------------------------------------

\begin{frame}
\frametitle{Revision}

\begin{figure}
\includegraphics[width=0.8\linewidth]{images/rev2017.jpg}
\end{figure}

\end{frame}

%----------------------------------------------

\begin{frame}
\frametitle{Revision}

\begin{itemize}
\item Since 2011 (previously known as Breakpoint)
\item Approx. 700 visitors
\item Big companies are looking for jobs at the party!
\item Seminars
\item 30+ competitions (demo, gfx, photography, \dots)
\end{itemize}

\end{frame}

%----------------------------------------------

\begin{frame}
\frametitle{Exercise}

\begin{itemize}
\item Form groups (maximum of 5)
\item Create a demo!
\end{itemize}

\end{frame}

%----------------------------------------------

\begin{frame}
\frametitle{Do you want a C64?}

\begin{itemize}
\item Yes you can!
\item Expensive on ebay etc.
\item Cheap at flea market
\item Keyboard
\item HCC
\end{itemize}

\end{frame}

%----------------------------------------------

\begin{frame}
\frametitle{HCC}

\begin{itemize}
\item Dutch Hobby Computer Club
\item Since 27 april 1977
\item Multiple groups of interest
\item Meeting once every two months in Maarssen (Netherlands)
\item Retro (8/16 bit) computers
\item Ultimate II+
\item Swapping
\end{itemize}

\end{frame}

%------------------------------------------------

\begin{frame}
\frametitle{Further Reading/Learning}

\begin{itemize}
\item Demoscene
\item Oldskool computers
\item Programming Contests
\item Cracking 'n Hacking
\end{itemize}

\end{frame}

%------------------------------------------------

\begin{frame}
\frametitle{Conclusions}

\begin{itemize}
\item PC History
\item Hacking on C64
\item Programming to the Limit
\item Programming Contests
\end{itemize}

\end{frame}

%------------------------------------------------

%------------------------------------------------

\begin{frame}
\frametitle{Bibliography}

\begin{multicols}{2}
\begin{itemize}
\item archive.org
\item bolo.ch
\item c64-wiki.com
\item codebase64.org
\item commodore.hcc.nl
\item csdb.dk
\item dcs.warwick.ac.uk
\item demoparty.net
\item derbian.webs.com
\item devil.iki.fi
\item famitracker.com
\item freeinfosociety.com
\item granishmusicproduction. com
\item history-computer.com
\item id.scene.org
\item ieee.org
\end{itemize}
\end{multicols}

\end{frame}

%------------------------------------------------

\begin{frame}
\frametitle{Bibliography}

\begin{multicols}{2}
\begin{itemize}
\item musictechstudent.co.uk
\item oldcomputers.net
\item pagetable.com
\item pouet.net
\item revision-party.net
\item scs-trc.net
\item sta.c64.org
\item vice.sourceforge.net
\item wikipedia.org
\end{itemize}
\end{multicols}

\end{frame}
%------------------------------------------------

\begin{frame}
\frametitle{Special Thanks to}

\begin{itemize}
\item Lezingencommissie
\item VIA
\item CCC
\item HCC, lemon64, csdb, VICE
\item Hobbyisten and C64 fans
\item Scene
\item \'Ecole Polytechnique F\'ed\'erale de Lausanne
% trs80 image:
\item Mus\'ee Bolo
\end{itemize}

\end{frame}

%------------------------------------------------

\begin{frame}
\frametitle{Special Thanks to - HCC and Sceners}

\begin{itemize}
\item Duncan
\item fgenesis
\item fieserWolf
\item Ron van Schaik
\item Wilfred Bos
\end{itemize}

\end{frame}

%------------------------------------------------

\begin{frame}
\frametitle{Special Thanks to - Preview Testers}

\begin{itemize}
\item Bert
\item Boaz
\item Brian
\item Jelle
\item Kees
\item Otto
\item Sander
\item Teun
\item Tjaard
\end{itemize}

\end{frame}

%------------------------------------------------

\begin{frame}
\frametitle{Special Thanks to - gfx}

\begin{multicols}{2}
\begin{itemize}
% breadbin comic
\item Bart van Tieghem
% vectrex
\item Evan-Amos
\item Bill Bertram
% CRT raster scan
\item Ian Harvey
\item Jozef Galanda
% bleeding MSX
\item Marco L
% C1541 drive
\item Nathan Beach
% waveforms
\item Omegatron
% trs80 image author:
\item Rama
\item Ruben de Rijcke
% C64 back size
\item Sammler
% great giana sisters - sprite vergroesserung
\item Werner
\end{itemize}
\end{multicols}

\end{frame}

%------------------------------------------------

\begin{frame}
\frametitle{Special Thanks to Sceners}

\begin{multicols}{3}
\begin{itemize}
\item Abyss Connection
\item Abnormal
\item Albion Crew
\item Alpha Flight
\item Algotech
\item Arkanix Labs
\item Artline Designs
\item Arise
\item Artstate
\item Arsenic
\item Atlantic
\item Atlantis
\item Atw
\item Bauknecht
\item Black Sun
\item Bluez Muz
\item Bonzai
\item Booze Design
\item Byterapers
\item Beyond Force
\end{itemize}
\end{multicols}

\end{frame}

%------------------------------------------------

\begin{frame}
\frametitle{Special Thanks to Sceners}

\begin{multicols}{3}
\begin{itemize}
\item Camelot
\item Cascade
\item Censor
\item Chorus
\item Chrome
\item Cosine
\item Covert Bitops
\item Creators
\item Crescent
\item Crest
\item Darklite
\item Delysid
\item Dekadence
\item Desire
\item Dmagic
\item Dual Crew
\item Elysium
\item Extend
\item Excess
\item Exon
\item Fairlight
\item Fatzone
\item Focus
\item Fossil
\end{itemize}
\end{multicols}

\end{frame}

%------------------------------------------------

\begin{frame}
\frametitle{Special Thanks to Sceners}

\begin{multicols}{3}
\begin{itemize}
\item Genesis Project
\item Glance
\item Hack 'n 'trade
\item Hitman
\item Hoaxers
\item Holger
\item Hokuto Force
\item Horizon
\item House Designs
\item HVSC Crew
\item Judas
\item Laxity
\item Lepsi De
\item Level 64
\item LFT
\item Mahoney
\item Maniacs of Noise
\item Mayday
\item MDG
\end{itemize}
\end{multicols}

\end{frame}

%------------------------------------------------

\begin{frame}
\frametitle{Special Thanks to Sceners}

\begin{multicols}{3}
\begin{itemize}
\item Metalvotze
\item Miracles
\item Multistyle Labs
\item The New Dimension
\item Nah-Kolor
\item Noice
\item No Name
\item Nostalgia
\item Nuance
\item Offence
\item Onslaught
\item Oxyron
\item Padua
\item Panda Design
\item Plush
\item Poo-brain
\item Prosonix
\item Radwar
\item Razor
\item Reflex
\item Resource
\item Samar
\item Scene sat
\item Shape
\item Sidrip
\end{itemize}
\end{multicols}

\end{frame}

%------------------------------------------------

\begin{frame}
\frametitle{Special Thanks to Sceners}

\begin{multicols}{3}
\begin{itemize}
\item Singular
\item Slay Radio
\item Smash Designs
\item Starion
\item Style
\item SCS*TRC
\item TPUG
\item Triad
\item Tropyx
\item TRSI
\item Unicess
\item Up Rough
\item Vandalism News
\item Vibrants
\item VICE team
\item Vision
\item Wow
\item Wrath Designs
\item Xenon
\item You Lazy Bastards
\end{itemize}
\end{multicols}

\end{frame}


%------------------------------------------------

\begin{frame}
\frametitle{Thank you for your attention}
\Large{\centerline{Are there any questions?}}
\end{frame}

\end{document}
